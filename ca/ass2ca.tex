\documentclass{article}
\usepackage[T1]{fontenc}
\usepackage{lmodern}
\usepackage[utf8]{inputenc}
\usepackage[british]{babel}
\usepackage{geometry}
\usepackage{color}
\usepackage{amsthm}
\usepackage{amsmath,amssymb}
\usepackage{graphicx}
\usepackage{mathtools}
\usepackage{listings}
\usepackage{newlfont}
\usepackage{tikz-cd}

\newcommand{\numberset}{\mathbb}
\newcommand{\N}{\numberset{N}}
\newcommand{\Z}{\numberset{Z}}
\newcommand{\R}{\numberset{R}}
\newcommand{\Q}{\numberset{Q}}
\newcommand{\C}{\numberset{C}}
\newcommand{\K}{\numberset{K}}
\newcommand{\F}{\numberset{F}}
\newcommand{\n}{\mathcal{N}}
\newcommand{\aid}{\mathfrak{a}}
\newcommand{\bid}{\mathfrak{b}}
\newcommand{\pid}{\mathfrak{p}}
\newcommand{\qid}{\mathfrak{q}}
\newcommand{\mi}{\mathfrak{m}}
\newcommand{\I}{\mathbb{I}}
\newcommand{\V}{\mathbb{V}}


\newcommand{\exercise}[1]{\noindent {\bf Exercise #1}}

\DeclareMathOperator{\Spec}{Spec}
\DeclareMathOperator{\mSpec}{mSpec}
\DeclareMathOperator{\cont}{cont}
\DeclareMathOperator{\Ima}{Im}

\begin{document}

\title{Commutative Algebra - Assignment 2}

\author{Matteo Durante, 2303760, Leiden University\\Waifod@protonmail.com}

\maketitle


\exercise{1}

$(a)$ Consider the ideals $\qid_1=<X^2-7>^2,\qid_2=<X^2+7>\subset\Z[X]$.

Since $\qid_2=\pid_2$ is generated by an irreducible polynomial, it is prime and hence primary. Furthermore, the associated prime is itself.

On the other hand, let $gh\in\qid_1$. Then, $gh=(X^2-7)^2t$ and, if $g\not\in\qid_1$, $(x^2-7)^2\not|g$, thus, being $X^2-7$ irreducible and $\Z[X]$ a UFD, $(X^2-7)|h$. It follows that $(X^2-7)^2|h^2$, and therefore $h^2\in\qid_1$.

The prime associated to $\qid_1$ is $\pid_1=r(\qid_1)=<X^2-7>$, for we have that $X^2-7$ is irreducible and $(X^2-7)\in r(\qid_1)$, which must therefore contain $<X^2-7>$, which in turn contains $<X^2-7>^2$.

Notice that $\pid_1+\pid_2=<X^2-7,14>$, thus neither $\pid_1$ is contained nor contains $\pid_2$ and a prime ideal $\pid$ containing both has to contain either $2$ or $7$ (and just one prime, for otherwise it would be equal to $\Z[X]$). Observe now that $(f)=\qid_1\cap\qid_2$, with $(f)\subsetneq\qid_1,\qid_2$. We have found a suitable minimal primary decomposition. Furthermore, given that the set of associated primes is $\{\pid_1,\pid_2\}$, they are both isolated, hence we have by~\cite[cor. 4.11]{atm} its uniqueness.

$(b)$ We can better describe the ideals in $A$ by looking at the corresponding ideals in $\Z[X]$ containing $(f)$, thanks to the fact that the elements of $A$ are just equivalence classes of those in $\Z[X]$ and we have a 1:1 correspondence between the two sets of ideals. More specifically, $\Spec(A)\cong V(f)$ through the morphism of spectra induced by $\Z[X]\xrightarrow{\pi} A$. It follows that it is sufficient to describe the irreducible components of $V(f)$ and solve the problem there. Then, the generators of a prime in $A$ will be just the classes of the generators of the corresponding one in $\Z[X]$.

Notice that, if $(f)=\qid_1\cap\qid_2\subset\pid$, then $\qid_1\subset\pid$ or $\qid_2\subset\pid$ by~\cite[prop. 1.11(ii)]{atm}, and hence $\pid_1=r(\qid_1)\subset\pid$ or $\pid_2=r(\qid_2)\subset\pid$. It follows that $V(f)=V(\pid_1)\cup V(\pid_2)$. Furthermore, being $\pid_1,\pid_2$ primes, $V(\pid_1),V(\pid_2)$ are irreducible. We have already given a system of generators for each $\pid_i$, which are s.t. $\Spec(A)=V(\pi(\pid_1))\cup V(\pi(\pid_2))=V(X^2-7)\cup V(X^2+7)$ is a decomposition in irreducible components.

$(c)$ Suppose $\pi(\pid_i)\subset\pid\in\mSpec(A)$, $2,7\not\in\pid$, and hence $\pi(\pid_i) A_{\pid}\subset\pid A_{\pid},\pi(\pid_j) A_{\pid}\not\subset\pid A_{\pid}$ for $j\neq i$. By previous observations and the 1:1 order-preserving correspondence between the prime ideals in $A_{\pid}$ and the ones in $A$ contained in $\pid$, $\pi(\pid_i)A_{\pid}$ is a prime ideal contained in every other prime ideal of $A_{\pid}$. Indeed, they are of the form $\pid'A_{\pid}$, where $\pid'\in\Spec(A)=V(\pi(\pid_1))\cup V(\pi(\pid_2))$, and if $\pi(\pid_i)\not\subset\pid'$, and hence $\pi(\pid_j)\subset\pid'$, then $\pi(\pid_j)\subset\pid$, against the assumption. Remember that $nil(A_{\pid})=\bigcap \pid'A_{\pid}=\pi(\pid_i) A_{\pid}$. This is a prime ideal in $A_{\pid}$, hence $A_{\pid}/nil(A_{\pid})\cong A_{\pid}/\pi(\pid_i)A_{\pid}\cong (A/\pi(\pid_i))_{\pid}$ is an integral domain. It is a local ring by construction, with $\pid(A/\pi(\pid_i))_{\pid}$ being the maximal ideal. Remember that it is a Noetherian ring by~\cite[cor. 7.4]{atm}, as $A/\pi(\pid_1)$ is by~\cite[cor. 7.1]{atm}. Furthermore, to justify the last isomorphism, notice that the sequence $0\rightarrow\pi(\pid_i)\rightarrow A\rightarrow A/\pi(\pid_i)\rightarrow 0$ is exact, hence $0\rightarrow\pi(\pid_i) A_{\pid}\rightarrow A_{\pid}\rightarrow (A/\pi(\pid_i))_{\pid}\rightarrow 0$ is also exact by~\cite[prop. 3.3]{atm}.

We shall focus on $\pi(\pid_1)\subset\pid$, which corresponds to a maximal ideal in $\Z[X]$ of the form $<p,f(X)>$ s.t. $f|X^2-7 \mod p$ for $p\neq 2,7$. The proof in the case where $\pi(\pid_2)\subset\pid$ is essentially identical. In what comes after, we may use $\pid$ to refer either to the maximal ideal in $A/\pi(\pid_1)\cong\Z[X]/\pid_1$ or the one in $\Z[X]$.

There are two cases in $A/\pi(\pid_1)$: either $X^2-7$ is irreducible$\mod p$, and hence $\pid=<p,X^2-7>=<p>$, or $X^2-7=(X-a)(X+a) \mod p$, in which case $\pid$ can be either $<p,X-a>$ or $<p,X+a>$.

An element in $(A/\pi(\pid_1))_{\pid}$ is of the form $f_1(X)/f_2(X)$, where $f_2(X)\not\in\pid$.

Considering the case where $\pid=<p,X-a>$, we have that $X^2-7=X^2-a^2+pk$, hence $pk=(X^2-7)-(X^2-a^2)$.

If $p\not |\ k$, since $k\not\in\pid$, then, considered $h(X)/g(X)\in\pid(A/\pi(\pid_1))_{\pid}$:
\begin{align*}
  \frac{h(X)}{g(X)} & = \frac{ps(X)+(X-a)f(X)}{g(X)} \\
  & = \frac{pks(X)+(X-a)kf(X)}{kg(X)} \\
  & = (X-a)\frac{kf(X)-(X+a)s(X)}{kg(X)}\in <X-a>(A/\pi(\pid_1))_{\pid}
\end{align*}

It follows that $\pid (A/\pi(\pid_1)_{\pid})=<X-a>(A/\pi(\pid_1))_{\pid}$.

Notice that $X+a\not\in\pid$, for otherwise $2a\in\pid$ and therefore $p|2a$, i.e., given that $p\neq 2$, $p|a$, thus $X^2-7=X^2\mod p$ and $p|7$, which is absurd. If $p\not |\ k-1$, then:
\begin{align*}
  \frac{h(X)}{g(X)} & = \frac{ps(X)+(X-a)f(X)}{g(X)} \\
  & = \frac{p(X+a)s(X)+(X^2-a^2)f(X)}{(X+a)g(X)} \\
  & = p\frac{(X+a)s(X)-kf(X)}{(X+a)g(X)}\in <p>(A/\pi(\pid_1))_{\pid}
\end{align*}

Hence, $\pid (A/\pi(\pid_1))_{\pid}=<p>(A/\pi(\pid_1))_{\pid}$.

Let $\aid$ be a non-zero ideal of a local Noetherian domain $A$ with maximal $\pid$ ideal generated by $g\neq 0$. We will show that it is a PID.

Define $n:=\min \{m\geq 1\ |\ \exists x\in\aid\textit{ s.t. } x\in\pid^m\setminus\pid^{m+1}\}$. We will prove that $n<\infty$. Let $\bid=\bigcap_{m\geq 1}\pid^m$ and fix a primary decomposition of $\bid\pid$ (which is possible by~\cite[thm. 7.13]{atm}). We only have to show that $\bid$ is contained in every primary ideal $\qid$ of the decomposition. Let $\pid'=\sqrt{\qid}$. If $\pid'\neq\pid$, pick $x\in\pid\setminus\pid'$. It is not nilpotent in $A/\qid$, hence it is not a zero-divisor either. However, $x\bid\subset\pid\bid\subset\qid$, hence $\bid\subset\qid$. If $\pid'=\pid=(g)$, being $g\in\sqrt{\qid}$, for some $m>0$ we have $\bid\subset\pid^m\subset\qid$, hence we are done. It follows that $\pid\bigcap_{m\geq 1}\pid^m=\bigcap_{m\geq 1}\pid^m=0$ by~\cite[prop. 2.6]{atm}, thus, for some $x\in\aid$, there is a $m>0$ s.t. $x\not\in\pid^{m+1}$, hence $n<\infty$. By the minimality of $n$, $\aid\subset\pid^n$ and there is a $x\in\aid\setminus\pid^{n+1}$. Since $\pid^n/\pid^{n+1}$ is a 1-dimensional $A/\pid$-vector space (*), $x$ forms a basis, hence, by~\cite[cor. 2.7]{atm}, $<x>\subset\aid\subset\pid^n=<x>$.

Since $(A/\pi(\pid_1))_{\pid}$ satisfies all of the hypothesis, it is a PID, hence it is a UFD and therefore integrally closed.

(*) Notice that $\pid/\pid^2\cong\pid^n/\pid^{n+1}$ by the $A/\pid$-isomorphism induced by $\cdot g^{n-1}$. We will show that $g$ forms a $A/\pid$-basis of the $A/\pid$-vector space $\pid/\pid^2$. Indeed, an element there is of the form $ag$, where $a=0$ or $a\in A\setminus\pid$, hence we are done as $\pid\neq\pid^2$, for otherwise $\pid$ would be $=0$ by~\cite[prop. 2.6]{atm}.








\begin{thebibliography}{9}
\bibitem{atm}
		M.F. Atiyah, I.G. Macdonald,
		\textit{Introduction to Commutative Algebra},
		CRC Press,
		1994.
\end{thebibliography}


\end{document}
