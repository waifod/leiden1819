\documentclass{article}
\usepackage[T1]{fontenc}
\usepackage{lmodern}
\usepackage[utf8]{inputenc}
\usepackage[british]{babel}
\usepackage{geometry}
\usepackage{color}
\usepackage{amsthm}
\usepackage{amsmath,amssymb}
\usepackage{graphicx}
\usepackage{mathtools}
\usepackage{listings}
\usepackage{newlfont}
\usepackage{tikz-cd}

\newcommand{\numberset}{\mathbb}
\newcommand{\N}{\numberset{N}}
\newcommand{\Z}{\numberset{Z}}
\newcommand{\R}{\numberset{R}}
\newcommand{\Q}{\numberset{Q}}
\newcommand{\C}{\numberset{C}}
\newcommand{\K}{\numberset{K}}
\newcommand{\F}{\numberset{F}}
\newcommand{\n}{\mathcal{N}}
\newcommand{\aid}{\mathfrak{a}}
\newcommand{\bid}{\mathfrak{b}}
\newcommand{\pid}{\mathfrak{p}}
\newcommand{\qid}{\mathfrak{q}}
\newcommand{\mi}{\mathfrak{m}}
\newcommand{\I}{\mathbb{I}}
\newcommand{\V}{\mathbb{V}}
\newcommand{\exercise}[1]{\noindent {\bf Exercise #1}}

\newcommand{\Pic}{\operatorname{Pic}}

\begin{document}

\title{Algebraic Number Theory - Assignment 4}

\author{Matteo Durante, 2303760, Leiden University}

\maketitle


\exercise{4}

We see that $f=x^3-x^2-6x-12$ is the polynomial defining the desired number ring extension of $\Z$, as $\beta$ is a root and it is irreducible in $\Z[x]$ because it doesn't have any root in $\Z$ (we only have to check the divisors in $\Z$ of the constant term). Its derivative is $f'=3x^2-2x-6$.

Seeing that the discriminant of $f$ is $\Delta(f)=\frac{(-1)^{3(3-1)/2}R(f,f')}{1}=-4332=-2^2\cdot 3\cdot 19^2$, it has multiple roots only modulo $2,3$ and $19$, hence it is regular above all other primes.

Now, we see that $f(13)=0\mod 19$, hence, noticing that $(x-13)^3=x^3-39x^2+507x-2197=x^3-x^2-6x-12\mod 19$, and therefore $\pid_1=(19,\beta-13)$, $e_1=3$, we look at the division of $f$ by $x-13$ in $\Z[x]$. There, we get $f=(x^2+12x+150)(x-13)+1938$ and, since $1938=2\cdot 3\cdot 17\cdot 19$, the only prime ideal above $19$ is regular.

The analysis of the primes above $3$ has already been carried out in the notes, hence we will move on to the ones above $2$.

Seeing that $f=x^3+x^2=x^2(x+1)\mod 2$, we can already say that these primes are exactly $\pid_1=(2,\beta)$ and $\pid_2(2,1+\beta)$. Furthermore, since $e_2=1$, the second one is necessarily regular. On the other hand, $e_1=2$ and since $f=x(x^2-x-6)-12$ we get that $r_1=-12=-2^2\cdot 3$, hence $\pid_1$ is singular.

Now, we see that $\frac{1}{2}(\beta^2-\beta-6)=-(1+\alpha)\in r(\pid)$ by~\cite[cor. 3.2]{stev}, thus, since $\Z[\beta]\subset r(\pid)$, $\Z[\alpha,\beta]=\tilde{R}\subset r(\pid)$.

This proves that $\pid_1$ is an integral $\tilde{R}$-ideal. Now, since all non-zero ideals in this ring are invertible because it is a Dedekind Domain, we get that $\pid_1$ is an invertible $\tilde{R}$-ideal and hence proper, which concludes the proof (indeed, $\Q(\beta)=\Q(\alpha,\beta)$ and anyway the multiplier ring is invariant under field extension).


~\\
\exercise{8}

Given that $B_{\pid}\subset Q(B)$, it is a number ring.

Let $\qid$ be a non-zero prime in $B$. Then, being $A$ a subring of $B$, $\qid\cap A$ is a non-zero prime of $A$.

Indeed, it is trivially an ideal and it is proper because $1\not\in\qid\cap A\subset\qid$. Furthermore, there is a prime element $q\in\qid\cap\Z\subset\qid\cap A$. If the ideal was not prime, then there would be $a,b\in A\setminus\qid\subset B\setminus\qid$ s.t. $ab\in\qid\cap A\subset\qid$, which is absurd.

Now, the primes in $B_{\pid}:=(A\setminus\pid)^{-1}B$ are those which correspond to the ones in $B$ disjoint from $A\setminus\pid$. Let $\qid\subset B$ be a non-zero prime. Since the intersection with $A$ is a non-zero prime, either $\qid\cap A=\pid$ or $\qid\cap(A\setminus\pid)\neq\emptyset$. Indeed, we can't have $\qid\cap A\subsetneq\pid$ or $\supsetneq\pid$ because we are in a domain with Krull dimension 1. In the former case, $\qid$ corresponds to a prime in $B_{\pid}$, while in the latter it becomes the whole ring in $B_{\pid}$.

Since a prime ideal $\qid$ corresponding to a non-zero prime in $B_{\pid}$ must contain $\pid$, it contains a prime $p\in\pid$. Given that there are finitely many primes in $B$ with this property (this follows from the fact that $(p)$ has finite index in $B$), $B_{\pid}$ has finitely many prime ideals and hence finitely many maximal ones.

\begin{thebibliography}{9}
		\bibitem{stev}
				P. Stevenhagen,
				\textit{Number Rings},
				2017.
\end{thebibliography}

\end{document}
