\documentclass{article}
\usepackage[T1]{fontenc}
\usepackage{lmodern}
\usepackage[utf8]{inputenc}
\usepackage[british]{babel}
\usepackage{geometry}
\usepackage{color}
\usepackage{amsthm}
\usepackage{amsmath,amssymb}
\usepackage{graphicx}
\usepackage{mathtools}
\usepackage{listings}
\usepackage{newlfont}
\usepackage{tikz-cd}

\newcommand{\numberset}{\mathbb}
\newcommand{\N}{\numberset{N}}
\newcommand{\Z}{\numberset{Z}}
\newcommand{\R}{\numberset{R}}
\newcommand{\Q}{\numberset{Q}}
\newcommand{\C}{\numberset{C}}
\newcommand{\K}{\numberset{K}}
\newcommand{\F}{\numberset{F}}
\newcommand{\n}{\mathcal{N}}
\newcommand{\aid}{\mathfrak{a}}
\newcommand{\bid}{\mathfrak{b}}
\newcommand{\pid}{\mathfrak{p}}
\newcommand{\qid}{\mathfrak{q}}
\newcommand{\mi}{\mathfrak{m}}
\newcommand{\I}{\mathbb{I}}
\newcommand{\V}{\mathbb{V}}

\newcommand{\exercise}[1]{\noindent {\bf Exercise #1}}

\begin{document}

\title{Algebraic Number Theory - Assignment 2}

\author{Matteo Durante, 2303760, Leiden University}

\maketitle


Please consider exercises 10 and 24.


~\\
\exercise{10}
\begin{align*}
		(H:I):J & = \{x\in\K\ |\ xJ\subset H:I\} \\
		& = \{x\in\K\ |\ \forall j\in J\ xj\in H:I\} \\
		& = \{x\in\K\ |\ \forall j\in J\ \forall i\in I\ xij\in H\} \\
		& = \{x\in\K\ |\ xIJ\subset H\} \\
		& = H:(IJ) \\ \\
		(\bigcap_k I_k):J & = \{x\in\K\ |\ xJ\subset\bigcap_k I_k\} \\
		& = \{x\in\K\ |\ \forall k\ xJ\subset I_k\} \\
		& = \bigcap_k\{x\in\K\ |\ xJ\subset I_k\} \\
		& = \bigcap_k (I_k:J) \\ \\
		I:(\sum_k J_k) & = \{x\in\K\ |\ x(\sum_k J_k)\subset I\} \\
		& = \{x\in\K\ |\ \sum_k xJ_k\subset I\} \\
		& = \{x\in\K\ |\ \forall k\ xJ_k\subset I\} \\
		& = \bigcap_k \{x\in\K\ |\ xJ_k\subset I\} \\
		& = \bigcap_k (I:J_k)
\end{align*}

Indeed, let $x\in\K$ be s.t. $x(\sum_k J_k)\subset I$. Then, for every finite set of indexes and any choice of elements $f_{k_i}\in J_{k_i}$, $x(f_{k_1}+\cdots+f_{k_n})=xf_{k_i}+\cdots+xf_{k_n}\in I$, hence $\sum_k xJ_k\subset I$. The proof in the opposite direction follows the steps backwords.


~\\
\exercise{12}

We know that, given a domain $R$ and fractional $R$-ideals $I$ and $J$, $r(I)=I:I$ and $I=IR$.

Being a fractional ideal, $I$ is an $R$-module and the same goes for $r(I)$, hence $\forall x\in R$ we have that $xI\subset I$, therefore $R\subset r(I)$.
Now, let $x\in r(I)$. Then, $xI\subset I$, ergo $xII^{-1}\subset II^{-1}$, i.e. $xR\subset R$. In particular, $x=x1\in R$, therefore $R=r(I)$.

Now, let $R,\ R'$ be subrings (subdomains) of $\K=Q(R)=Q(R')$ (field) such that $I$ is an invertible $R$-ideal and an invertible $R'$-ideal. Earlier we proved that, given these conditions, we should have $r(I)=R$ and $r(I)=R'$. Since the definition of $r(I)=\{x\in\K\ |\ xI\subset I\}$ is independent from the subring we are considering, we get that $R=R'$.


~\\
\exercise{13}

First, we consider the ring homomorphism $\phi:\Z[\sqrt{-19}]\rightarrow\F_2$ defined as $\phi(a+b\sqrt{-19})=a+b$. Let's prove that it is a homomorphism:
\begin{align*}
		\phi((a+b\sqrt{-19})+(c+d\sqrt{-19})) & = \phi((a+c)+(b+d)\sqrt{-19}) \\
		& = (a+c)+(b+d) \\
		& =\phi(a+b\sqrt{-19})+\phi(c+d\sqrt{-19}) \\
		\phi((a+b\sqrt{-19})(c+d\sqrt{-19})) & =\phi((ac-19bd)+(ad+bc)\sqrt{-19}) \\
		& = ac-19bd+ad+bc \\
		& = ac+bd+ad+bc \\
		& =(a+b)(c+d) \\
		& =\phi(a+b\sqrt{-19})\phi(c+d\sqrt{-19})
\end{align*}

Notice that $\ker\phi=(2,1+\sqrt{-19})$, hence it is a maximal ideal. We observe that $\frac{1-\sqrt{-19}}{2}\in r(\mi)$ because $2\frac{1-\sqrt{-19}}{2}=1-\sqrt{-19}\in\mi$ and $(1+\sqrt{-19})\frac{1-\sqrt{-19}}{2}=10\in\mi$, therefore $\frac{1-\sqrt{-19}}{2}\in r(\mi)\neq R$.

Observe that $\mi^2=(2^2,2(1+\sqrt{-19}),(1+\sqrt{-19})^2)=(4,2+2\sqrt{-19},-18+2\sqrt{-19})=(4,2+2\sqrt{-19})=2\mi$, hence, if it did have an inverse $J$, setting $R=\Z[\sqrt{-19}]$, we would have $I=IR=I^2J=2IJ=2R$, which is obviously false.

Let $2R=\pid\qid$ with $\pid,\ \qid$ prime ideals. Then, $2\mi=\mi^2\subset\pid\qid\subset\pid\cup\qid$, hence $\mi^2$ is contained in $\pid$ or $\qid$; let's say $\mi^2\subset\pid$. Then, being $\pid$ prime, $\mi\subset\pid$, thus $\mi=\pid$. Now we have that $2R=\mi\qid$, hence $2\mi=\mi^2\qid$, which implies that $\qid=R$. We have arrived at a contradiction.


~\\
\exercise{24}

Let $\alpha\not\subset\pid_i\ \forall i\leq n$. We will prove that $\alpha\not\subset\bigcup_{i=1}^n\pid_i$.

For $n=1$, the thesis is trivial.

Let's assume it is true for $n-1\ n>1$. Then, for any choice of $n-1$ indexes among those $n$, we may find an element $x_j\in(\alpha\setminus\bigcup_{i=1,i\neq j}^n\pid_i)$.

If there is an index $j$ s.t. $x_j\not\in\pid_j$, we are done.

Otherwise, having $x_j\in\pid_j\ \forall j$, consider $y=\sum_{j=1}^n\Pi_{i=1,i\neq j}^n x_i$. We have that $y\in\alpha$ and $y\not\in\pid_j\ \forall j$.

Indeed, if this was not the case, then $\Pi_{i=1,i\neq j}^n x_i\in\pid_j$ for some $j$, hence $x_i\in\pid_j$ for some $i\neq j$ because $\pid_j$ is prime, which is absurd.

\end{document}
