\documentclass{article}
\usepackage[T1]{fontenc}
\usepackage{lmodern}
\usepackage[utf8]{inputenc}
\usepackage[british]{babel}
\usepackage{geometry}
\usepackage{color}
\usepackage{amsthm}
\usepackage{amsmath,amssymb}
\usepackage{graphicx}
\usepackage{mathtools}
\usepackage{listings}
\usepackage{newlfont}
\usepackage{tikz-cd}

\newcommand{\numberset}{\mathbb}
\newcommand{\N}{\numberset{N}}
\newcommand{\Z}{\numberset{Z}}
\newcommand{\R}{\numberset{R}}
\newcommand{\Q}{\numberset{Q}}
\newcommand{\C}{\numberset{C}}
\newcommand{\K}{\numberset{K}}
\newcommand{\F}{\numberset{F}}
\newcommand{\n}{\mathcal{N}}
\newcommand{\aid}{\mathfrak{a}}
\newcommand{\bid}{\mathfrak{b}}
\newcommand{\pid}{\mathfrak{p}}
\newcommand{\qid}{\mathfrak{q}}
\newcommand{\mi}{\mathfrak{m}}
\newcommand{\I}{\mathbb{I}}
\newcommand{\V}{\mathbb{V}}

\newcommand{\exercise}[1]{\noindent {\bf Exercise #1}}

\newcommand{\Id}{\operatorname{Id}}
\newcommand{\Pic}{\operatorname{Pic}}
\newcommand{\Tr}{\operatorname{Tr}}

\begin{document}

\title{Algebraic Number Theory - Assignment 6}

\author{Matteo Durante, 2303760, Leiden University}

\maketitle


\exercise{5}

Consider~\cite[ex. 2.10]{stev}. There, we have $\pid=(2,1+\sqrt{-19})\subset\Z[\sqrt{-19}]=R$, which has index 2, and $\pid^2=2\pid=(4,2+2\sqrt{-19})$.

On the other end, considered the ideal $2R$. By the epimorphism $\Z[X]\rightarrow R$ sending $X$ to $\sqrt{-19}$, we get that it corresponds to the ideal $(2,X^2+19)\subset\Z[X]$. Observing the classes in the quotient ring $\Z[X]/(2,X^2+19)\cong R/2R$, we see that these can be represented by polynomials whose degree is $<2$ and having director coefficient and constant term $<2$. Furthermore, each polynomial like this represents a different class, thus these rings have 4 elements and 4 is the index of $2R$.

If the index map was multiplicative, then $8=|R:\pid||R:2R|=|R:2\pid|=|R:\pid^2|=|R:\pid|^2=4$, which is absurd.

I guess that the failure of multiplicativity comes from the fact that $\pid$ is a singular prime, but is this condition sufficient?


\newpage

\exercise{18}

Given $f=X^3-aX-b\in\K[X]$, $f'=3X^2-a$. The roots of $f'$ are $\lambda_1=\frac{\sqrt{3a}}{3}$ and $\lambda_2=-\lambda_1$.

Applying the usual properties of the resultant, we get that:
\begin{align*}
		\Delta(f) & = (-1)^{3(3-1)/2}R(f,f') \\
		& = -(-1)^{3\cdot 2}R(f',f) \\
		& = -3^3\Pi_{i=1}^2 f(\lambda_i) \\
		& = -27\left(\left(\frac{\sqrt{3a}}{3}\right)^3-a\frac{\sqrt{3a}}{3}-b\right) \left( \left(-\frac{\sqrt{3a}}{3}\right)^3-a\left(-\frac{\sqrt{3a}}{3}\right)-b\right) \\
		& = -27\left((-b)+\left(\left(\frac{\sqrt{3a}}{3}\right)^3-a\frac{\sqrt{3a}}{3}\right)\right)\left((-b)-\left(\left(\frac{\sqrt{3a}}{3}\right)^3-a\frac{\sqrt{3a}}{3}\right)\right) \\
		& = 27\left(\left(\frac{\sqrt{3a}}{3}\right)^3-a\frac{\sqrt{3a}}{3}\right)^2-27b^2 \\
		& = 27\left(\frac{\sqrt{3a}}{3}\right)^2\left(\left(\frac{\sqrt{3a}}{3}\right)^2-a\right)^2-27b^2 \\
		& = 9a\left(-\frac{2a}{3}\right)^2-27b^2 \\
		& = 4a^3-27b^2
\end{align*}

In the same way, considered $g=X^n+a\in\K[X], n>0$, we have $g'=nX^{n-1}$.

Let $n>1$. Then, only root of $g'$ is $0$, with multiplicity $n-1$:
\begin{align*}
		\Delta(g) & = (-1)^{n(n-1)/2}R(g,g') \\
		& = (-1)^{n(n-1)/2}(-1)^{n(n-1)}R(g',g)\textit{ and, since $n(n-1)$ is even, $(-1)^{n(n-1)}=1$} \\
		& = (-1)^{n(n-1)/2}n^n\Pi_{i=1}^{n-1} g(0) \\
		& = (-1)^{n(n-1)/2}n^n a^{n-1}
\end{align*}

If $n=1$, then the only root of $g$, that is $a$, lies in $\K$, thus $g=f^a_{\K}$ and $\Delta(1)=\Delta(1,a,\ldots,a^{n-1})=\Delta(f^a_{\K})=\Delta(g)$.

Since $\Delta(x_1,\ldots,x_n)=\det(\Tr_{B/A}(x_ix_j))^n_{i,j=1}$ in a free $A$-algebra $B$ of rank $n$, where $x_1,\ldots,x_n\in B$, being $\K$ naturally a free $\K$-algebra of rank 1, we get that $\Delta(1)=\det(\Tr_{\K/\K}(1\cdot 1))^1_{i,j=1}=\Tr_{\K/\K}(1)$.

But $\Tr_{\K/\K}(1)=\Tr(M_1)=\Tr(\Id_1)=1$, thus $\Delta(g)=1=(-1)^{1(1-1)/2}1^1a^{1-1}$, where this equality holds even for $a=0$ as long as we accept the heresy that $0^0=1$.


\begin{thebibliography}{9}
\bibitem{stev}
		P. Stevenhagen,
		\textit{Number Rings},
		2017.
\end{thebibliography}

\end{document}
