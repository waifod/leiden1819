\documentclass{article}
\usepackage[T1]{fontenc}
\usepackage{lmodern}
\usepackage[utf8]{inputenc}
\usepackage[british]{babel}
\usepackage{geometry}
\usepackage{color}
\usepackage{amsthm}
\usepackage{amsmath,amssymb}
\usepackage{graphicx}
\usepackage{mathtools}
\usepackage{listings}
\usepackage{newlfont}
\usepackage{tikz-cd}

\newcommand{\numberset}{\mathbb}
\newcommand{\N}{\numberset{N}}
\newcommand{\Z}{\numberset{Z}}
\newcommand{\R}{\numberset{R}}
\newcommand{\Q}{\numberset{Q}}
\newcommand{\C}{\numberset{C}}
\newcommand{\K}{\numberset{K}}
\newcommand{\F}{\numberset{F}}
\newcommand{\n}{\mathcal{N}}
\newcommand{\aid}{\mathfrak{a}}
\newcommand{\bid}{\mathfrak{b}}
\newcommand{\pid}{\mathfrak{p}}
\newcommand{\qid}{\mathfrak{q}}
\newcommand{\mi}{\mathfrak{m}}
\newcommand{\I}{\mathbb{I}}
\newcommand{\V}{\mathbb{V}}

\newcommand{\exercise}[1]{\noindent {\bf Exercise #1}}

\newcommand{\Ima}{\operatorname{Im}}
\newcommand{\Id}{\operatorname{Id}}
\newcommand{\Pic}{\operatorname{Pic}}
\newcommand{\Tr}{\operatorname{Tr}}
\newcommand{\Gal}{\operatorname{Gal}}
\newcommand{\sgn}{\operatorname{sgn}}


\begin{document}

\title{Algebraic Number Theory - Assignment 11}

\author{Matteo Durante, 2303760, Leiden University}

\maketitle

\exercise{5}

We see that we are asked to find the smallest unit $>1$ and of norm 1 in $\Z[\sqrt{61}]$.

Let's consider the number field $\K\cong\Q(\sqrt{61})\cong\Q[X]/(f)$, $f=X^2-61$, and the number ring $R=\Z[\sqrt{61}]$.

Noticing that $61\equiv 1\mod 4$, we have by~\cite[thm. 3.10]{stev} that $\mathcal{O}_{\K}\cong\Z[\frac{1+\sqrt{61}}{2}]\cong\Z[X]/(g)$, where $g=X^2-X-15$. It is an order of rank 2 and we shall set $\alpha:=\frac{1+\sqrt{61}}{2}$.

Since $[\K:\Q]=2$ and $f$ has only real roots, $\mathcal{O}_{\K}$ has only two real embeddings, hence by~\cite[thm. 5.13]{stev} we have that $\mathcal{O}^*_{\K}\cong<-1>\times<\eta_0>$, where $\sigma(\eta_0)>1$ for the embedding representing the ring of integers as $\Z[\alpha]$. Let's fix this embedding.

We will now compute $\Pic(\mathcal{O}_{\K})$.

Notice that $\Delta(g)=61$, thus $M_{\K}=\sqrt{61}/2$. We only have to check the primes above 2 and 3.

Since $g$ has no roots in $\F_2$, it is irreducible in $\F_2[X]$ and the only prime above 2 is precisely 2.

On the other hand, $g\equiv X^2-X=X(X-1)\mod 3$, thus 3 splits and we have $\pid_3=(3,\alpha),\qid_3=(3,1-\alpha)$, $\pid_3\qid_3=(3)$.

If we can prove that one among $\pid_3$ and $\qid_3$ is principal, then we are done showing that $\Pic(\mathcal{O}_{\K})=0$ because $[\pid_3]=[\qid_3]^{-1}$.

However, noticing that $3+\alpha=\frac{7+\sqrt{61}}{2}\in(3,\alpha)$ has norm 3 like $\pid_3$, we have $\pid_3=(\frac{7+\sqrt{61}}{2})$.

In the same way, we get that $\qid_3=(\frac{7-\sqrt{61}}{2})$.

It follows that, since $\mathcal{O}_{\K}$ is a Dedekind ring with trivial Picard group, it is a PID by~\cite[ex. 2.39]{stev}.

$g(0)=-15$, thus $\pid_3\pid_5=(\alpha)$, where $\pid_5=(5,\alpha)=(\frac{9-\sqrt{61}}{2})$, for $5=\frac{9-\sqrt{61}}{2}\frac{9+\sqrt{61}}{2}$ and $\alpha=\frac{9-\sqrt{61}}{2}\frac{7+\sqrt{61}}{2}$.

$g(6)=15$, thus $\pid_3\qid_5=(6-\alpha)$, where $\qid_5=(5,1-\alpha)=(\frac{9+\sqrt{61}}{2})$.

$g(10)=75$, thus $\qid_3\pid_5^2=(10-\alpha)$.

Remembering that $\qid_3=3\pid_3^{-1}$ and $\qid_5=5\pid_5^{-1}$, we have the following relations: $\pid_3\pid_5=(\alpha)$, $\pid_3\pid_5^{-1}=\frac{(6-\alpha)}{5}$, $\pid_3^{-1}\pid_5^2=\frac{(10-\alpha)}{3}$.

The ideal generated by $\eta=\alpha^a(\frac{6-\alpha}{5})^b(\frac{10-\alpha}{3})^c$ will be then factorized as $\pid_3^{a+b-c}\pid_5^{a-b+2c}$. Setting the exponents $=0$, we consider a solution of the system of equations: $(1,-3,-2)$.

It follows that $\eta=\frac{39+5\sqrt{61}}{2}\in\mathcal{O}_{\K}^*$, $\eta>1$. We still need to show that it is a fundamental unit of this ring.

Remember that $\eta_0=t+u\alpha\in\mathcal{O}_{\K}^*$, $\eta_0>1$, is a fundamental unit. Then, $\eta_0^n=\eta$ for some $n\geq 1$, where $t,u>0$ because $\eta$ has positive coefficients w.r.t. the basis $\{1,\alpha\}$.

Now, $N(\eta_0)=t^2+tu-15u^2=1$. If $u=1$, $t^2+t-16=0$ has no natural solutions, thus $u>1$.

It follows that $\eta>2\alpha$, thus, since $1\leq n=\frac{\log(\eta)}{\log(\eta_0)}<\frac{\log(\eta)}{\log(2\alpha)}<2$, $n=1$ and $\eta$ is a fundamental unit.

Now, to find the fundamental unit of this ring, we still have to find out which is the lowest $n>0$ s.t. $\eta^n\in\Z[\sqrt{61}]$, for $\mathcal{O}_{\K}$ is integral over $\Z[\sqrt{61}]$ and, by~\cite[ex. 5.20]{stev}, $\Z[\sqrt{61}]^*=\Z[\sqrt{61}]\cap\mathcal{O}_{\K}^*$.

First of all, notice that $n>1$ because $\eta$ doesn't lie in $\Z[\sqrt{61}]=\Z+2\mathcal{O}_{\K}$. However, by~\cite[5.16]{stev}, the index of $\Z[\sqrt{61}]^*$ in $\mathcal{O}_{\K}^*$ divides the order of $(\mathcal{O}_{\K}/2\mathcal{O}_{\K})^*\cong\F_4^*$. From this we get that $n=3$, thus $\Z[\sqrt{61}]^*\cong<-1>\times<\eta^3>$, where $\eta^3=29718+3805\sqrt{61}$.

We only have to check which ones are the elements of norm 1 in this latest unit group and s.t. $\sqrt{61}$ has a positive coefficient. Since $N(\eta^3)=-1$ and the norm is multiplicative, these are precisely the even powers of $\eta^3$, thus the smallest integral solution of our Pell equation s.t. $y>0$ is given by the coefficients of $\eta^6=1766319049+226153980\sqrt{61}$.


~\\
\exercise{6}

We see that we are asked to find the smallest unit $>1$ and of norm 1 in $\Z[\sqrt{109}]$.

Let's consider the number field $\K\cong\Q(\sqrt{109})\cong\Q[X]/(f)$, $f=X^2-109$, and remember the number ring $R=\Z[\sqrt{109}]$.

Noticing that $109\equiv 1\mod 4$, we have by~\cite[thm. 3.10]{stev} that $\mathcal{O}_{\K}\cong\Z[\frac{1+\sqrt{109}}{2}]\cong\Z[X]/(g)$, where $g=X^2-X-27$. It is an order of rank 2 and we shall set $\alpha:=\frac{1+\sqrt{109}}{2}$.

Since $[\K:\Q]=2$ and $f$ has only 2 real roots, $\mathcal{O}_{\K}$ has only two real embeddings, hence by~\cite[thm. 5.13]{stev} we have that $\mathcal{O}_{\K}^*\cong<-1>\times<\eta>$, where $\sigma(\eta)>1$ for the embedding representing the ring of integers as $\Z[\alpha]$. Let's fix this embedding.

We will now compute $\Pic(\mathcal{O}_{\K})$.

Notice that $\Delta_{\K}=109$, thus $M_{\K}=\sqrt{109}/2$. We only have to check the primes above 2, 3 and 5.

Since $g$ has no root in $\F_2$, it is irreducible in $\F_2[X]$ and the only prime above 2 is precisely $(2)$.

Furthermore, $g\equiv X^2-X=X(X-1)\mod 3$, thus 3 splits and we have $\pid_3=(3,\alpha),\qid_3=(3,1-\alpha)$.

In the same way, $g\equiv X^2-6X+8=(X-4)(X-2)\mod 5$, thus 5 splits and we have $\pid_5=(5,4-\alpha),\qid_5=(5,2-\alpha)$.

If we can show that one ideal above 3 and one above 5 is principal, then we are done showing that $\Pic(\mathcal{O}_{\K})=0$ because $[\pid_p]=[\qid_p]^{-1}$.

Observe now that $2\cdot 3-\alpha=\frac{11-\sqrt{109}}{2}\in\pid_3$ has norm 3 like $\pid_3$, hence $\pid_3=(\frac{11-\sqrt{109}}{2})$.

Furthermore, noticing that $7\cdot 5-4(4-\alpha)=21+2\sqrt{109}\in\pid_5$ has norm 5 like $\pid_5$, we have $\pid_5=(21+2\sqrt{109})$.

It follows that, since $\mathcal{O}_{\K}$ is a Dedekind ring with trivial Picard group, it is a PID by~\cite[ex. 2.39]{stev}.

$g(4)=-15$, thus $\qid_3\pid_5=(4-\alpha)$.

$g(9)=45$, thus $\pid_3^2\pid_5=(9-\alpha)$.

$g(27)=675$, thus $\pid_3^3\qid_5^2=(27-\alpha)$.

Remembering that $\qid_p=p\cdot\pid_p^{-1}$, we have the following relations: $\pid_3^{-1}\pid_5=\frac{(4-\alpha)}{3}, \pid_3^2\pid_5=(9-\alpha), \pid_3^3\pid_5^{-2}=\frac{(27-\alpha)}{25}$.

The ideal generated by $\eta=(\frac{4-\alpha}{3})^a(9-\alpha)^b(\frac{27-\alpha}{25})^c$ will be then factorized as $\pid_3^{-a+2b+3c}\pid_5^{a+b-2c}$. Setting the exponents $=0$, we consider a solution of the system of equations: $(-7,1,-3)$.

It follows that $\eta=\frac{261+25\sqrt{109}}{2}\in\mathcal{O}_{\K}^*$, $\eta>1$. We still need to show that it is a fundamental unit of this ring.

Remember that $\eta_0=t+u\alpha\in\mathcal{O}_{\K}^*$, $\eta_0>1$, is our fundamental unit. Then, $\eta_0^n=\eta$ for some $n\geq 1$, where $t,u>0$ because $\eta$ has positive coefficients w.r.t. the basis $\{1,\alpha\}$.

Now, $N(\eta_0)=t^2+tu-27u^2=1$ and, looking at $t^2+tu-(27u^2+1)$, we shall see this as a polynomial in $t$.

We can check that for $u\in\{1,2,3,4\}$ there are no natural $t$ satisfying the equation by finding the roots of our polynomial.

It follows that $u\geq 5$ and therefore $\eta_0>5\alpha$, thus, since $1\leq n=\frac{\log(\eta)}{\log(\eta_0)}<\frac{\log(\eta)}{\log(5\alpha)}<2$, $n=1$ and $\eta$ is a fundamental unit.

Now, to find the fundamental unit of this ring, we still have to find out which is the lowest $n>0$ s.t. $\eta^n\in\Z[\sqrt{109}]$, for $\mathcal{O}_{\K}$ is integral over $\Z[\sqrt{109}]$ and, by~\cite[ex. 5.20]{stev}, $\Z[\sqrt{109}]^*=\Z[\sqrt{109}]\cap\mathcal{O}_{\K}^*$.

First of all, notice that $n>1$ because $\eta$ doesn't lie in $\Z[\sqrt{109}]=\Z+2\mathcal{O}_{\K}$. However, by~\cite[5.16]{stev}, the index of $\Z[\sqrt{109}]^*$ in $\mathcal{O}_{\K}^*$ divides the order of $(\mathcal{O}_{\K}/2\mathcal{O}_{\K})^*\cong\F_4^*$. From this we get that $n=3$, thus $\Z[\sqrt{109}]^*\cong<-1>\times<\eta^3>$, where $\eta^3=8890182+851525\sqrt{109}$.

We only have to check which ones are the elements of norm 1 in this latest unit group and s.t. $\sqrt{109}$ has a positive coefficient. Since $N(\eta^3)=-1$ and the norm is multiplicative, these are precisely the even powers of $\eta^3$, thus the smallest integral solution of our Pell equation s.t. $y>0$ is given by the coefficients of $\eta^6=158070671986249+15140424455100\sqrt{109}$.


\begin{thebibliography}{9}
  \bibitem{stev}
		P. Stevenhagen,
		\textit{Number Rings},
		2017.
\end{thebibliography}

\end{document}
