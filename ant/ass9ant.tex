\documentclass{article}
\usepackage[T1]{fontenc}
\usepackage{lmodern}
\usepackage[utf8]{inputenc}
\usepackage[british]{babel}
\usepackage{geometry}
\usepackage{color}
\usepackage{amsthm}
\usepackage{amsmath,amssymb}
\usepackage{graphicx}
\usepackage{mathtools}
\usepackage{listings}
\usepackage{newlfont}
\usepackage{tikz-cd}

\newcommand{\numberset}{\mathbb}
\newcommand{\N}{\numberset{N}}
\newcommand{\Z}{\numberset{Z}}
\newcommand{\R}{\numberset{R}}
\newcommand{\Q}{\numberset{Q}}
\newcommand{\C}{\numberset{C}}
\newcommand{\K}{\numberset{K}}
\newcommand{\F}{\numberset{F}}
\newcommand{\n}{\mathcal{N}}
\newcommand{\aid}{\mathfrak{a}}
\newcommand{\bid}{\mathfrak{b}}
\newcommand{\pid}{\mathfrak{p}}
\newcommand{\qid}{\mathfrak{q}}
\newcommand{\mi}{\mathfrak{m}}
\newcommand{\I}{\mathbb{I}}
\newcommand{\V}{\mathbb{V}}

\newcommand{\exercise}[1]{\noindent {\bf Exercise #1}}

\newcommand{\Ima}{\operatorname{Im}}
\newcommand{\Id}{\operatorname{Id}}
\newcommand{\Pic}{\operatorname{Pic}}
\newcommand{\Tr}{\operatorname{Tr}}
\newcommand{\Gal}{\operatorname{Gal}}
\newcommand{\sgn}{\operatorname{sgn}}


\begin{document}

\title{Algebraic Number Theory - Assignment 9}

\author{Matteo Durante, 2303760, Leiden University}

\maketitle

~\\
\exercise{14}

First of all, notice that, given $k\in\Z$, both $X-k$ and $X^2-(2k+1)X+k(k+1)$ are monic polynomials in $\Z[X]$ with discriminant $=1$ up to sign, therefore we only have to check the other implication.

Notice that such a polynomial must have distinct roots, for otherwise $\Delta(f)=0$.

If it is irreducible in $\Z[X]$ and it has a non-integer root $\alpha$ (which therefore does not lie in $\Q$), then it is irreducible in $\Q[X]$ and, since $\Q[X]/(f)\cong\Q(\alpha)$ is a non-trivial finite field extension of $\Q$, we have $\Z\ni|\Delta(f)|\geq|\Delta_{\Q(\alpha)}|\neq 1$, hence the roots of our polynomial must lie in $\Z$.

If it is the product of two (distinct monic) polynomials in $\Z[X]$, $f=f_1f_2$, then, calling $\alpha_i$ the roots of $f$ and $\alpha_{ki}$ the roots of $f_k$, we get $\Delta(f)=\Pi_{i<j} (\alpha_i-\alpha_j)^2=\Pi_{i<j} (\alpha_{1i}-\alpha_{1j})^2\cdot\Pi_{i<j} (\alpha_{2i}-\alpha_{2j})^2\cdot\Pi_{i,j} (\alpha_{1i}-\alpha_{2j})^2=\Delta(f_1)\Delta(f_2)Res(f_1,f_2)^2$.

Since $Res(f_1,f_2)\in\Z$, to have $|\Delta(f)|=1$ we need $|\Delta(f_1)|=|\Delta(f_2)|=1$ (they already lie in $\Z$). Iterating the procedure, we get that the absolute values of the discriminants of the irreducible factors have to be $=1$, hence $f$ only has integer roots.

Continuing, since $\Delta(f)=\Pi_{i<j} (\alpha_i-\alpha_j)^2$ and $\alpha_i\in\Z$, we need $|\alpha_i-\alpha_j|=1$ for every $i,j$, where $i\neq j$. This means that it can't have more than two roots.

It follows that either it has only one root, and hence $f=X-k$, or it has only two (consecutive) roots, i.e. $f=(X-k)(X-(k+1))=X^2-(2k+1)X+k(k+1)$, for some $k\in\Z$.


~\\
\exercise{20}

Let $A$ be an integral ring extension of $R$.

If $x\in R^*$, then $x^{-1}\in R\subset A$, hence $x\in A^*$ and $R^*\subset A^*\cap R$.

We only have to show the opposite inclusion.

Let $x\in A^*\cap R$. Being $x^{-1}$ integral over $R$, it is the zero of a monic polynomial $f=\sum_{i=0}^n r_iX^i\in R[X]$. It follows that $x^{n-1}(\sum_{i=0}^n r_ix^{-i})=(\sum_{i=0}^{n-1} x^{n-(i+1)}r_i)+x^{-1}=0$, hence $x^{-1}=-\sum_{i=0}^{n-1} x^{n-(i+1)}r_i\in R$ and $x\in R^*$.

We notice that the integrality condition is necessary, for $\Q^*\cap\Z=\Z\setminus\{0\}\neq\Z^*$.

\begin{thebibliography}{9}
  \bibitem{stev}
		P. Stevenhagen,
		\textit{Number Rings},
		2017.
\end{thebibliography}

\end{document}
