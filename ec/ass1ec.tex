\documentclass{article}
\usepackage[T1]{fontenc}
\usepackage{lmodern}
\usepackage[utf8]{inputenc}
\usepackage[british]{babel}
\usepackage{geometry}
\usepackage{color}
\usepackage{amsthm}
\usepackage{amsmath,amssymb}
\usepackage{graphicx}
\usepackage{mathtools}
\usepackage{listings}
\usepackage{newlfont}
\usepackage{tikz-cd}
\usepackage{rotating}
\usepackage[backend=biber]{biblatex}
\addbibresource{~/Documents/math/references.bib}

\newcommand{\numberset}{\mathbb}
\newcommand{\N}{\numberset{N}}
\newcommand{\Z}{\numberset{Z}}
\newcommand{\R}{\numberset{R}}
\newcommand{\Q}{\numberset{Q}}
\newcommand{\K}{\numberset{K}}
\newcommand{\F}{\numberset{F}}
\newcommand{\n}{\mathcal{N}}
\newcommand{\aid}{\mathfrak{a}}
\newcommand{\bid}{\mathfrak{b}}
\newcommand{\pid}{\mathfrak{p}}
\newcommand{\qid}{\mathfrak{q}}
\newcommand{\mi}{\mathfrak{m}}
\newcommand{\I}{\mathbb{I}}
\newcommand{\V}{\mathbb{V}}
\newcommand{\A}{\mathbb{A}}
\newcommand{\Ps}{\mathbb{P}}
\newcommand{\exercise}[1]{\noindent {\bf Exercise #1}}

\DeclareMathOperator{\Ima}{Im}
\DeclareMathOperator{\coker}{coker}
\DeclareMathOperator{\Id}{Id}
\DeclareMathOperator{\GL}{GL}

\begin{document}

\title{Elliptic Curves - Assignment 1}

\author{Matteo Durante, s2303760, Leiden University}

\maketitle


~\\
\exercise{2}

$(b)$ Consider the following system of equations:
\[
    \begin{cases}
        y^2=x^3+2x^2 \\
        y=\lambda x
    \end{cases}
    \begin{cases}
        x^3+(2-\lambda^2)x^2=x^2(x+(2-\lambda^2))=0 \\
        y=\lambda x
    \end{cases}
\]

The second order equation in $x$ has solutions given by $0$ and $\lambda^2-2$ and, 
since $\sqrt{2}\in\R\setminus\Q$, $2-\lambda^2\neq 0$ for $\lambda\in\Q$, thus 
the only solution $\neq (0,0)$ of the system of equations is 
$P_{\lambda}=(\lambda^2-2,\lambda^3-2\lambda)$.

$(c)$ Notice that, as $\lambda\in\Q$ varies, we get every point of $C$ (except 
for those with $x=0$) as a solution of the previous system of equations.

Indeed notice that, if $x=0$, then $y=0$ for any point in $C$. This means that, 
given $P=(a,b)\in C\setminus (0,0)$, $a\neq 0$. Then, since $a,b\in\Q$, $b/a\in\Q$
and thus $(a,b)=P_{\lambda}$ for a unique $\lambda=b/a\in\Q$.

Now, since each $\lambda\in\Q$ locates a unique $P_{\lambda}\in C\setminus (0,0)$ 
(the one s.t. $b/a=\lambda$), we may parametrize bijectively the $\Q$-rational 
points in $C$ through the following function:
\begin{align*}
    f: \Ps^1_{\Q} &\rightarrow C \\
    (\lambda:i) &\mapsto\begin{cases}
        ((\lambda/i)^2-2,(\lambda/i)^3-2(\lambda/i))\textit{ if }i\neq 0 \\
        (0,0)\textit{ otherwise}
    \end{cases}
\end{align*}


~\\
\exercise{3}

$(b)$ Consider the polynomial $g(x,y)=f(x)-y^2$, $f(x)\in\K[x]$. It is s.t. 
$\nabla{g}=(f'(x),-2y)$. Since an affine curve $C$ is s.t. $\dim C=1$, it is 
smooth at $P\in C$ if and only if $\nabla g(P)\neq (0,0)$, i.e. if and only if
it has rank $2-1=1$.

Now, given $P\in C$, $\nabla g(P)=(0,0)$ if and only if $f'(p_1)=-2y(p_2)=0$, 
which combined with $g(P)=0$ is equivalent to $f(p_1)=f'(p_1)=0,p_2=0$, i.e. 
$p_1\in\overline{\K}$ is a multiple root of $f(x)$ and the second coordinate is 0. 
This means that such a curve presents a singular point if and only if $f(x)$ has 
a multiple root over $\overline{\K}$.

$(c)$ We know that $f(x)=x^3+ax+b$ defines a smooth curve $C$ if and only if it 
is separable. i.e. it doesn't have a multiple root. This is equivalent to 
$\Delta(f)\neq 0$. Remember that $\Delta(f)=(-1)^{\frac{3\cdot 2}{2}}Res(f,f')=
-Res(f,f')=-Res(f',f)$.

Let $char(\K)=3$. Then, $f'(x)=a$.

If $a=0$, $f(x)=x^3+b=(x+\sqrt[3]{b})^3$ has a triple
root, $\sqrt[3]{b}$, and $4a^3+27b^2=4\cdot 0+0\cdot b^2=0$.

If $a\neq 0$, $Res(f',f)=a^3\neq 0$ and $4a^3+27b^2=4a^3\neq 0$.

Let $char(\K)\neq 2,3$. Then, $f'(x)=3x^2+a$ has roots $\pm\sqrt{-\frac{a}{3}}$. It 
follows that $Res(f',f)=3^3\cdot f(\sqrt{-\frac{a}{3}})\cdot f(-\sqrt{-\frac{a}{3}})=
3^3(-\frac{a}{3}\sqrt{-\frac{a}{3}}+a\sqrt{-\frac{a}{3}}+b)(\frac{a}{3}
\sqrt{-\frac{a}{3}}-a\sqrt{-\frac{a}{3}}+b)=4a^3+27b^3$, hence $f(x)$ has a multiple 
root if and only if $4a^3+27b^2=0$.


~\\
\exercise{6}

$(b)$ Let $a\in \K^*$. Then, since $v$ is a group homomorphism, $v(a^{-1})=-v(a)$, 
thus for any $a\in R_v\setminus\{0\}$ we have that $v(a)=0$ implies $v(a^{-1})=0$ 
and therefore $a^{-1}\in R_v$, i.e. $a\in R_v^*$, while $v(a)>0$ implies $v(a^{-1})<0$ 
and $a^{-1}\not\in R_v$.

Observe that $v(-a)=v(a)+v(-1)=v(a)$ for every $a\in\K$.

Let $a,b\in R_v\setminus\{0\}$ and suppose $v(a)\geq v(b)$. Then, 
$v(ab^{-1})=v(a)-v(b)\geq 0$, thus $ab^{-1}\in R_v$ and, since 
$a=a(b^{-1}b)=(ab^{-1})b$, $a\in (b)$. This implies that $R_v$ is a PID, as every 
non-zero ideal is generated by its element of lowest norm, which exists because 
$v(\aid)\subset\N$ is bounded below for every non-zero ideal $\aid$ of $R_v$.

Consider now $\mi=\{0\}\cup\{a\in R_v\ |\ v(a)>0\}$ and take $a,b\in\mi$, $c\in R_v$. 
Since $v(a-b)\geq\min\{v(a),v(-b)\}>0$ and $v(ac)=v(a)+v(c)\geq v(a)>0$, $a-b,ac\in\mi$. 
It follows that $\mi$ is a proper ideal of $R_v$, hence it is principal. 
Furthermore, it contains every non-invertible element of $R_v$, which will then 
be local with maximal ideal $\mi$.

Let $\pi\in\mi$ be s.t. $v(\pi)=1$. Any element $a\in\mi$ has norm $\geq 1$, 
thus by what we observed $a\in (\pi)$ and we are done.

$(c)$ As stated earlier, every non-zero ideal $\aid\subset R_v$ is principal and 
generated by its element of lowest norm. Let $(a)=\aid$. Then, for some 
$n\in\Z_{n\geq 0}$, $v(a)=n=v(\pi^n)$ and therefore, by previous observations, 
$a\in(\pi^n)$, but at the same time $\pi^n\in(a)$. It follows that $\aid=(\pi^n)$.

\printbibliography

\end{document}
