\documentclass{article}
\usepackage[T1]{fontenc}
\usepackage{lmodern}
\usepackage[utf8]{inputenc}
\usepackage[british]{babel}
\usepackage{geometry}
\usepackage{color}
\usepackage{amsthm}
\usepackage{amsmath,amssymb}
\usepackage{graphicx}
\usepackage{mathtools}
\usepackage{listings}
\usepackage{newlfont}
\usepackage{tikz-cd}
\usepackage{rotating}
\usepackage[backend=biber]{biblatex}
\addbibresource{~/math/references.bib}

\newcommand{\numberset}{\mathbb}
\newcommand{\N}{\numberset{N}}
\newcommand{\Z}{\numberset{Z}}
\newcommand{\R}{\numberset{R}}
\newcommand{\Q}{\numberset{Q}}
\newcommand{\K}{\numberset{K}}
\newcommand{\F}{\numberset{F}}
\newcommand{\n}{\mathcal{N}}
\newcommand{\aid}{\mathfrak{a}}
\newcommand{\bid}{\mathfrak{b}}
\newcommand{\pid}{\mathfrak{p}}
\newcommand{\qid}{\mathfrak{q}}
\newcommand{\mi}{\mathfrak{m}}
\newcommand{\I}{\mathbb{I}}
\newcommand{\V}{\mathbb{V}}
\newcommand{\A}{\mathbb{A}}
\newcommand{\Ps}{\mathbb{P}}
\newcommand{\exercise}[1]{\noindent {\bf Exercise #1}}

\DeclareMathOperator{\Ima}{Im}
\DeclareMathOperator{\coker}{coker}
\DeclareMathOperator{\Id}{Id}
\DeclareMathOperator{\GL}{GL}
\DeclareMathOperator{\divr}{div}

\begin{document}

\title{Elliptic Curves - Assignment 2}

\author{Matteo Durante, s2303760, Leiden University}

\maketitle


~\\
\exercise{1.a}

Let $v_{\phi(P)}(f)=n$ and write $f=t_{\phi(P)}^ng$ for some $g\in\overline{\K}(C_2)$ s.t.
$\phi^*(g)(P)=g(\phi(P))\neq 0$ (actually, $g\in\overline{\K}[C_2]_{\phi(P)}$,
$\phi^*(g)\in\overline{\K}[C_1]_P$), 
$t_{\phi(P)}\in\overline{\K}(C_2)$ a uniformizer of $C_2$
at $\phi(P)$. Then, $\phi^*(f)=\phi^*(t_{\phi(P)})^n\phi^*(g)$ and $v_{P}
(\phi^*(g))=0$, hence we get the following:
$$v_P(\phi^*(f))=v_P(\phi^*(t_{\phi(P)})^n\phi^*(g))=n\cdot
v_P(\phi^*(t_{\phi(P)}))+v_P(\phi^*(g))=v_{\phi(P)}(f)\cdot e_{\phi}(P)$$


~\\
\exercise{2}

Disclaimer: I will be using a result from the previous assignment to say that
$Y$ and $Y-1$ are uniformizers at specific points of the algebraic variety.

~\\
$(a)$ Notice that for a point $(x:y:z)\in C$ to be a zero of $f$ it has to
satisfy $y=0$ and, of course, $y^2=xz$, hence the only possible zeroes of $f$
are $Q=(1:0:0)$ and $P=(0:0:1)$. On the other hand, a pole of $f$ satisfies
$z=0$ and $y^2=xz$, thus the only possible pole is $Q=(1:0:0)$.

Now we shall study $f$ at $P$ and $Q$.

Considering the affine patch of $C$ given by $C_z=C\cap U_2$, i.e. $z=1$, the
curve has equation $h_z=Y^2-X=0$ and $f_z:=f|_{C_z}$ is represented by $Y$. On this affine
patch, the point $P$ has coordinates $(0,0)$ and, since
$\partial{h_z}/\partial{X}=-1\neq 0$, $Y$ is a uniformizer of $C_z$ at $P$.

Now, since $f_z=1\cdot Y^1$ and 1 is regular and non-zero at $P$, $v_P(f)=v_P(f_z)=1$.

On the other hand, considering the affine patch of $C$ given by $C_x=C\cap
U_0$, i.e. $x=1$, the curve has equation $h_x=Y^2-Z=0$ and $f_x:=f|_{C_x}$ is
represented by $Y/Z=1/Y$. On this affine patch, $Q$ has coordinates $(0,0)$ and,
since $\partial{h_x}/\partial{Z}=-1\neq 0$, $Y$ is a uniformizer of $C_x$ at $Q$.

Now, since $f_x=1\cdot Y^{-1}$ and 1 is regular and non-zero at $Q$, 
$v_Q(f)=v_Q(f_x)=-1$.

It follows that $\divr(f)=P-Q$.

$(b)$ Notice that for a point $(x:y:z)\in C$ to be a zero of $g$ it has to
satisfy $x=0$ and, of course, $y^2=xz$, hence the only possible zero of $g$
is $P=(0:0:1)$. On the other hand, a pole of $g$ satisfies
$z=0$ and $y^2=xz$, thus the only possible pole is $Q=(1:0:0)$.

Now we shall study $g$ at $P$ and $Q$ using the same affine patches and
uniformizers as before.

We see that, on $C_z$, $g_z$ can be represented as $X=Y^2=1\cdot Y^2$. Again,
since 1 is regular and non-zero at $P$, $v_P(g)=v_P(g_z)=2$.

In the same way, on $C_x$, $g_x$ can be represented as $1/Z=1/Y^2=1\cdot Y^{-2}$ and,
since 1 is regular and non-zero at $Q$, $v_Q(g)=v_Q(g_x)=-2$.

It follows that $\divr(g)=2P-2Q$.

$(c)$ We shall consider the function $s=\frac{Y-Z}{Y}\in\K(C)$.

Notice that for a point $(x:y:z)\in C$ to be a zero of $s$ it has to satisfy
$y-z=0$ and, of course, $y^2=xz$, hence the only possible zeroes are $Q=(1:0:0)$
and $R=(1:1:1)$. On the other hand, a pole of $s$ satisfies $y=0$ and $y^2=xz$,
thus the only possible poles are $P=(0:0:1)$ and $Q=(1:0:0)$.

Now we shall study $s$ at $P,\ Q$ and $R$ using for the first two the same
affine patches and uniformizers as before.

We see that, on $C_z$, $s_z$ can be represented as $\frac{Y-1}{Y}=(Y-1)\cdot
Y^{-1}$ and, since $Y-1$ is regular and non-zero at $P$, $v_P(s)=v_P(s_z)=-1$.

In the same way, on $C_x$, $s_x$ can be represented as
$\frac{Y-Y^2}{Y}=1-Y=(1-Y)\cdot Y^0$ and, since $1-Y$ is regular
and non-zero at $Q$, $v_Q(s)=v_Q(s_x)=0$.

Remaining on $C_x$, we see that $R$ has coordinates $(1,1)$ on this affine
patch. Since $\partial{h_x}/\partial{Z}=-1$, $Y-1$ is a uniformizer of $C_x$ at
$R$.

Now, since $s_x=1-Y=-1\cdot (Y-1)^1$ and $-1$ is regular and
non-zero at $R$, $v_R(s)=v_R(s_x)=1$.

We can conclude that $\divr(s)=R-P$.


~\\
\exercise{5}

Let $f\in\K(C)$ be s.t. $\divr(f)=D-D'$ and consider the function
$\mathcal{L}(D)\xrightarrow{\phi}\mathcal{L}(D')$ given by $g\mapsto fg$. We
want to prove that this is an isomorphism between the two vector spaces.

First of all, we prove that it is well defined. Indeed, if $g\in\mathcal{L}(D)$
is s.t. $\divr(g)+D\geq 0$, then
$\divr(fg)+D'=\divr(f)+\divr(g)+D'=D-D'+\divr(g)+D'=\divr(g)+D\geq 0$.

It is a $\K$-linear application, for $\phi(0)=f\cdot 0=0$ and, given $g,h\in
\mathcal{L}(D)$, $\lambda,\mu\in\K$ we have that $\phi(\lambda\cdot g+\mu\cdot 
h)=f(\lambda\cdot g+\mu\cdot h)=\lambda\cdot fg+\mu\cdot fh=\lambda\cdot\phi(g)+
\mu\cdot\phi(h)$.

It is invertible, for we can define another $\K$-linear application
$\mathcal{L}(D')\xrightarrow{\psi}\mathcal{L}(D)$ as $g\mapsto g/f$
(indeed, $1/f\in\K(C)$ and, remembering that $\divr(1/f)=-\divr(f)$, we can
check that the map is a well defined
$\K$-linear application in same way as we did earlier), which is s.t. $\phi\psi=
\Id_{\mathcal{L}(D')}$ and $\psi\phi=\Id_{\mathcal{L}(D)}$.

It follows that the two $\K$-vector spaces are isomorphic, hence they have the
same dimension.


\printbibliography

\end{document}


