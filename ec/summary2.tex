\documentclass{article}
\usepackage[T1]{fontenc}
\usepackage{lmodern}
\usepackage[utf8]{inputenc}
\usepackage[british]{babel}
\usepackage{geometry}
\usepackage{color}
\usepackage{amsthm}
\usepackage{amsmath,amssymb}
\usepackage{graphicx}
\usepackage{mathtools}
\usepackage{listings}
\usepackage{newlfont}
\usepackage{tikz-cd}
\usepackage{rotating}
\usepackage[backend=biber]{biblatex}
\addbibresource{~/math/references.bib}

\newcommand{\numberset}{\mathbb}
\newcommand{\N}{\numberset{N}}
\newcommand{\Z}{\numberset{Z}}
\newcommand{\R}{\numberset{R}}
\newcommand{\Q}{\numberset{Q}}
\newcommand{\K}{\numberset{K}}
\newcommand{\F}{\numberset{F}}
\newcommand{\n}{\mathcal{N}}
\newcommand{\aid}{\mathfrak{a}}
\newcommand{\bid}{\mathfrak{b}}
\newcommand{\pid}{\mathfrak{p}}
\newcommand{\qid}{\mathfrak{q}}
\newcommand{\mi}{\mathfrak{m}}
\newcommand{\I}{\mathbb{I}}
\newcommand{\V}{\mathbb{V}}
\newcommand{\A}{\mathbb{A}}
\newcommand{\Ps}{\mathbb{P}}
\newcommand{\exercise}[1]{\noindent {\bf Exercise #1}}

\DeclareMathOperator{\im}{im}
\DeclareMathOperator{\coker}{coker}
\DeclareMathOperator{\Id}{Id}
\DeclareMathOperator{\GL}{GL}
\DeclareMathOperator{\Mat}{Mat}
\DeclareMathOperator{\Ext}{Ext}
\DeclareMathOperator{\Tor}{Tor}
\DeclareMathOperator{\Hom}{Hom}
\DeclareMathOperator{\rk}{rk}
\DeclareMathOperator{\dv}{div}
\DeclareMathOperator{\Dv}{Div}
\DeclareMathOperator{\Pic}{Pic}

\theoremstyle{plain}
\newtheorem{thm}{Theorem}
\newtheorem{lem}[thm]{Lemma}
\newtheorem{prop}[thm]{Proposition}
\newtheorem{cor}[thm]{Corollary}

\theoremstyle{definition}
\newtheorem{defn}[thm]{Definition}

\begin{document}

\title{Elliptic Curves - Summary}

\author{Matteo Durante, s2303760, Leiden University}

\maketitle

\begin{defn}
    Let $\K$ be a field.
    \begin{itemize}
        \item The affine $n$-space over $\K$ is
            $\mathbb{A}^n=\mathbb{A}^n(\overline{K})=\overline{K}^n$.
        \item The rational $\K$-points of $\mathbb{A}^n$ are $\mathbb{A}^n(\K)=
            \K^n$.
        \item For a set $S\subset\overline{\K}[x_0,\ldots,x_n]$ we define
            $\V(S)=\{P\in\mathbb{A}^n\ |\ f(P)=0\textit{ for all }f\in S\}$. In
            particular, if $I=(S)$, then $\V(S)=\V(I)$.
        \item An algebraic variety over $\overline{\K}$ is a set $\V(I)$ for some
            prime ideal $I$ of $\overline{\K}[x_0,\ldots,x_n]$.
        \item For a set $V\subset\mathbb{A}^n$, let
            $\I(V)=\{f\in\overline{\K}[x_0,\ldots,x_n]\ |\ f(P)=0\textit{ for 
            all }P\in V\}$.
    \end{itemize}
\end{defn}

\begin{thm}
    There is a 1:1 correspondence between the varieties in $\mathbb{A}^n$ and
    the prime ideals of $\overline{\K}[x_0,\ldots,x_n]$ given by
    $V\mapsto\I(V),\ I\mapsto\V(I)$.
\end{thm}

\begin{defn}
    The affine coordinate ring of a variety $V\subset\mathbb{A}^n$ is
    $\overline{\K}[V]=\overline{\K}[x_1,\ldots,x_n]/\I(V)$.
\end{defn}

\printbibliography

\end{document}
