\documentclass{article}
\usepackage[T1]{fontenc}
\usepackage{lmodern}
\usepackage[utf8]{inputenc}
\usepackage[british]{babel}
\usepackage{geometry}
\usepackage{color}
\usepackage{amsthm}
\usepackage{amsmath,amssymb}
\usepackage{graphicx}
\usepackage{mathtools}
\usepackage{listings}
\usepackage{newlfont}
\usepackage{tikz-cd}
\usepackage{rotating}
\usepackage[backend=biber]{biblatex}
\addbibresource{~/math/references.bib}

\newcommand{\numberset}{\mathbb}
\newcommand{\N}{\numberset{N}}
\newcommand{\Z}{\numberset{Z}}
\newcommand{\R}{\numberset{R}}
\newcommand{\Q}{\numberset{Q}}
\newcommand{\K}{\numberset{K}}
\newcommand{\F}{\numberset{F}}
\newcommand{\n}{\mathcal{N}}
\newcommand{\aid}{\mathfrak{a}}
\newcommand{\bid}{\mathfrak{b}}
\newcommand{\pid}{\mathfrak{p}}
\newcommand{\qid}{\mathfrak{q}}
\newcommand{\mi}{\mathfrak{m}}
\newcommand{\I}{\mathbb{I}}
\newcommand{\V}{\mathbb{V}}
\newcommand{\A}{\mathbb{A}}
\newcommand{\Ps}{\mathbb{P}}
\newcommand{\exercise}[1]{\noindent {\bf Exercise #1}}

\DeclareMathOperator{\Ima}{Im}
\DeclareMathOperator{\coker}{coker}
\DeclareMathOperator{\Id}{Id}
\DeclareMathOperator{\GL}{GL}
\DeclareMathOperator{\Mat}{Mat}
\DeclareMathOperator{\Ext}{Ext}
\DeclareMathOperator{\Tor}{Tor}


\begin{document}

\title{Elliptic Curves - Assignment 3}

\author{Matteo Durante, s2303760, Leiden University}

\maketitle


~\\
\exercise{3}

$(a)$ To do this, it is sufficient to assign integer values between 0 and 6 to
$X$ and find the square roots of $X^3+2$ modulo 7. We will then have to add the
point at infinity, i.e. the one satisfying $Y^2Z=X^3+2Z^3$ with $Z=0$ and s.t.
at least one between $X$ and $Y$ is $\neq 0$.
\[
    \begin{tabular}{|l|l|l|l|l|l|l|l|}\hline
        $X$ & 0 & 1 & 2 & 3 & 4 & 5 & 6 \\ \hline
        $Y$ & 3,4 & - & - & 1,6 & - & 1,6 & 1,6 \\ \hline
    \end{tabular}
\]

It follows that the complete list of points of the elliptic curve $E$ given by
the affine equation considered is
$(0:3:1),\ (0:4:1),\ (3:1:1),\ (3:6:1),\ (5:1:1),\ (5:6:1),\ (6:1:1),\ (6:6:1),\ 
(0:1:0)$. We will omit the last coordinate and denote the point at infinity by $O$.

~\\
$(b)$ Since $E(\F_7)$ has 9 elements, every element will have an order dividing
9. 

We know that $a_1=a_2=a_3=a_4=0,\ a_6=2$, hence $b_2=b_4=b_8=0,\ b_6=8\equiv 1$ in
$\F_7$~\cite[III.1]{Sil09}. By~\cite[prop. 2.3]{Sil09}, for every
point $P\in E(\F_7)$ we get the following:
$$x([2]P)=\frac{x^4_P+5x_P}{4x_P^3+1},\quad -(x_P,y_P)=(x_P,-y_P).$$

Thanks to this we get that, for every point $P\in E$, $x_P=x_{[2]P}$, thus either
$[2]P=P$ or $[2]P=-P$. It follows that every point has order 1 or 3 and, since
there is no element of order 9, $E(\F_7)$ is not cyclic.


~\\
\exercise{6}

\begin{proof}
    $(a)$ Let $E/\K$ be an elliptic curve defined over $\K$ s.t. $P=(0,0)\in E$
    is a point of order $\geq 4$. We know that it is given by a Weierstrass
    equation of the form $y^2+a_1xy+a_3y=x^3+a_2x^2+a_4x+a_6$, $a_i\in\K$ for
    every $i$.

    Since $P$ lies on it, $a_6=0$.

    Let $g(x,y)=y^2+a_1xy+a_3y-x^3-a_2x^2-a_4x$.

    Since $P$ does not have order 2, we know that the line tangent to $E$ at $P$
    is not vertical. Also, since $\nabla(g)=(-a_4,a_3)$, it has equation
    $a_3y=a_4x$ and by the previous observation $a_3\neq 0$. We can therefore do
    the substitution $y=y'+\frac{a_4}{a_3}x$, which turns our Weirstrass
    equation into $y^2+b_1xy+b_3y=x^3+b_2x^2$, $b_3=a_3\neq 0$, and changes the
    equation of the previously mentioned tangent line to $y=0$. Notice that it
    has not moved $P$.

    If the line tangent to $E$ at $P$ didn't meet any other point, then the
    third point on $E$ met by it would be $P$ itself. Let $Q$ be the third point
    on $E$ and the line passing through $O$ and $P$. We have that $[2]P=Q\neq
    O$. We want to determine $P+Q$, but this is obvious because the line passing
    through $P$ and $Q$ is again the one through $O$ and $P$, hence $[3]P=O$,
    which is absurd because it has order $\geq 4$ by assumption.

    We have shown that this tangent meets another point, $Q\neq O,P$. Since it
    has equation $y=0$, this means that $x^3+b_2x^2$ has a root $-b_2\neq 0$.

    We can then do another change of variables,
    $y=(\frac{b_3}{b_2})^3y',\ x=(\frac{b_3}{b_2})^2x'$. Dividing then the
    equation we now have by $(\frac{b_3}{b_2})^6$, we get the following:
    $$y^2+\frac{b_1b_2}{b_3}xy+\frac{b_2^3}{b_3^2}y=x^3+\frac{b_2^3}{b_3^2}x^2$$
    
    Setting $u=\frac{b_1b_2}{b_3},v=\frac{b_2^3}{b_3^2}$, we finally get the
    equation $y^2+uxy+vy=x^3+vx^2$.
\end{proof}

$(b)$ Let's look again at the previous setting and suppose that $P$ has order 5.
Remember that, up to isomorphism, our curve can be described by
$y^2+uxy+vy=x^3+vx^2$.

We have that $u\neq 1$, for otherwise $P$ would have order 4.

The tangent line at $-2P=(-v,0)$ can be described by the equation
$y=\frac{v}{1-u}(x+v)$ and, substituting this in the equation of $E$, we get
an equation of degree 3 for the coordinate $x$ of $4P=-2(-2P)$. By solving
it, we get that this coordinate is given by $\frac{v^2+uv-v}{u^2-2u+1}$.
Since $P$ has order 5 by assumption, we have $4P=-P=(0,-v)$, thus 
$\frac{v^2+uv-v}{u^2-2u+1}=0$.

It follows that $v(v-u+1)=0$ and, since $v\neq 0$, for otherwise $P=-P$, we
have that $u=1+v$. Substituting this into the equation of $E$, we get that
$y^2+(1+v)xy+vy=x^3+vx^2$, which gives us a one-parameter family of elliptic
curves with a rational point of order 5.

\printbibliography

\end{document}


