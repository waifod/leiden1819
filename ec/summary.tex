\documentclass{article}
\usepackage[T1]{fontenc}
\usepackage{lmodern}
\usepackage[utf8]{inputenc}
\usepackage[british]{babel}
\usepackage{geometry}
\usepackage{color}
\usepackage{amsthm}
\usepackage{amsmath,amssymb}
\usepackage{graphicx}
\usepackage{mathtools}
\usepackage{listings}
\usepackage{newlfont}
\usepackage{tikz-cd}
\usepackage{rotating}
\usepackage[backend=biber]{biblatex}
\addbibresource{~/math/references.bib}

\newcommand{\numberset}{\mathbb}
\newcommand{\N}{\numberset{N}}
\newcommand{\Z}{\numberset{Z}}
\newcommand{\R}{\numberset{R}}
\newcommand{\Q}{\numberset{Q}}
\newcommand{\K}{\numberset{K}}
\newcommand{\F}{\numberset{F}}
\newcommand{\n}{\mathcal{N}}
\newcommand{\aid}{\mathfrak{a}}
\newcommand{\bid}{\mathfrak{b}}
\newcommand{\pid}{\mathfrak{p}}
\newcommand{\qid}{\mathfrak{q}}
\newcommand{\mi}{\mathfrak{m}}
\newcommand{\I}{\mathbb{I}}
\newcommand{\V}{\mathbb{V}}
\newcommand{\A}{\mathbb{A}}
\newcommand{\Ps}{\mathbb{P}}
\newcommand{\exercise}[1]{\noindent {\bf Exercise #1}}

\DeclareMathOperator{\im}{im}
\DeclareMathOperator{\coker}{coker}
\DeclareMathOperator{\Id}{Id}
\DeclareMathOperator{\GL}{GL}
\DeclareMathOperator{\Mat}{Mat}
\DeclareMathOperator{\Ext}{Ext}
\DeclareMathOperator{\Tor}{Tor}
\DeclareMathOperator{\Hom}{Hom}
\DeclareMathOperator{\rk}{rk}
\DeclareMathOperator{\dv}{div}
\DeclareMathOperator{\Dv}{Div}
\DeclareMathOperator{\Pic}{Pic}

\theoremstyle{plain}
\newtheorem{thm}{Theorem}
\newtheorem{lem}[thm]{Lemma}
\newtheorem{prop}[thm]{Proposition}
\newtheorem{cor}[thm]{Corollary}

\theoremstyle{definition}
\newtheorem{defn}[thm]{Definition}

\begin{document}

\title{Elliptic Curves - Summary}

\author{Matteo Durante, s2303760, Leiden University}

\maketitle

\begin{thm}[Mordell]
    Given an elliptic curve $E/\Q$, $\rk(E(\Q))=\rk(E(\Q)/2E(\Q))<\infty$.
\end{thm}

\begin{thm}
    Given a curve $C$ and a rational map $C\xrightarrow{\phi}W\subset\Ps^n$, if
    $C$ is smooth at $P\in C$, then $\phi$ is regular at $P$. If $C$ is smooth,
    then $\phi$ is a morphism.
\end{thm}

\begin{cor}
    Let $C_1\xrightarrow{\phi}C_2$ be a morphism of smooth curves. If
    $\deg(\phi)=1$, then it is an isomorphism.
\end{cor}

\begin{prop}
    Given any smooth projective curve $C$, a morphism $C\rightarrow\Ps^1$ is
    either constant or surjective.
\end{prop}

\begin{prop}
    Let $C_1\xrightarrow{\phi}C_2$ be a non-constant morphism. Then:
    \begin{itemize}
        \item for every $Q\in C_2$, $\deg(\phi)=\sum_{P\in\phi^{-1}(Q)}e_\phi(P)$;
        \item If $C_2\xrightarrow{\psi}C_3$ is another morphism,
            $e_{\psi\circ\phi}(P)=e_\phi(P)\cdot e_{\psi}(\phi(P))$.
    \end{itemize}
\end{prop}

\begin{prop}
    For all but finitely many $Q\in C_2$, $\#\phi^{-1}(Q)=\deg_s(\phi)$. If we
    are working over $\Q$, $=\deg(Q)$.
\end{prop}

\begin{prop}
    Let $C$ be a smooth curve, $f\in\overline{\K}(C)^\times$. Then there are
    finitely many points $P\in C$ s.t. $ord_P(f)\neq 0$.

---

BEWARE: from now on, $\K$ will always be an algebraically closed field, $C$ a
smooth projective curve over $\K$.

---

\begin{prop}
    Given a smooth projective curve over $\K$, we have for any $f\in\K(C)$:
    \begin{itemize}
        \item $\dv(f)=0\Leftrightarrow f\in\K^\times$;
        \item $\deg(\dv(f))=0$
    \end{itemize}
\end{prop}

\begin{prop}
    $\Omega_C$ is a 1-dimensional $\K(C)$-vector space and a morphism
    $C_1\xrightarrow{\phi}C_2$ induces a map
    $\Omega_{C_2}\xrightarrow{\phi^*}\Omega_{C_1}$ defined as $\phi^*(f\cdot
    dx)=\phi^*(f)\cdot d(\phi^*(x))$. Also, $\phi$ is separable if and only if
    $\phi^*\neq 0$.
\end{prop}

\begin{thm}[Riemann-Roch]
    Given $D\in\Dv(C)$, $l(D)-l(K_C-D)=\deg(D)-g+1$.
\end{thm}

\begin{prop}
    Let $E$ be a smooth projective curve of genus 1 and defined over $\K$ not
    algebraically closed. Also, fixed $O\in E(\K)$, there is an isomorphism
    $C\xrightarrow{\phi}C\subset\Ps^1_{\overline{\K}}$ with $\phi(O)=(0:1:0)$
    and $C$ given by $y^2+a_1xy+a_2y=x^3+a_3x^2+a_4x+a_5$, which is the General
    Weierstrass equation.
\end{prop}

\begin{prop}
    Given $C$ and fixed $O\in E(\K)$, there is a map $C(\K)\rightarrow\Pic(C)$,
    $P\mapsto [P-O]$, which gives a bijection $C(\K)\leftrightarrow\Pic^0(C)$.
\end{prop}

\begin{prop}
    Let $\char(\K)\neq2,3$. If $C$ is given by a Weierstrass equation, then
    there exists a change of variables which reduces it to $y^2=x^3+ax+b$. Also,
    any isomorphism of elliptic curves is given by $x=u^2x',\ y=u^3y'$ for some
    $u\in\K^\times$.
\end{prop}

\begin{prop}
    \begin{itemize}
        \item Given any Weierstrass curve $E$ over a field $\K$ not necessarily
            algebraically closed, it is:\begin{enumerate}
                \item smooth $\Leftrightarrow\Delta\neq 0$; also,
                    $E(\K)\cong\Pic^0_\K(E)$;
                \item a node $\Leftrightarrow\Delta=0\neq C_4$; also,
                    $E^{ns}(\overline{K})\cong\overline{K}^\times$;
                \item a cusp $\Leftrightarrow\Delta=C_4=0$; also,
                    $E^{ns}(\K)\cong(\K,+)$.
            \end{enumerate}
        \item Two elliptic curves $E,\ E'$ over $\K$ are isomorphic if and only
            if $j(E)=j(E')$.
        \item For all $j_0\in\K$, there exists an elliptic curve $E$ over $\K$
            s.t. $j(E)=j_0$.
    \end{itemize}
\end{prop}

\begin{thm}
    Let $E$ be a Weierstrass curve over $\Q$ and $n\in\Z_{>0}$ s.t. $p\mid n$.
    Then, we have an injection $E(\Q)[n]\hookrightarrow\tilde{E}(\F_p)$. Also,
    the order of any point in $E(\Q)^{tors}$ divides
    $p^k\cdot \#\tilde{E}(\F_p)$ for some $k\in\N$.
\end{thm}

\begin{cor}
    Given any elliptic curve $E$ over $\Q$, $E(\Q)^{tors}$ is a finite subgroup
    of $E(\Q)$.
\end{cor}

\begin{thm}[Nagell-Lutz]
    Let $E/\Q$ be an elliptic curve given in short Weierstrass form by
    $y^2=x^3+ax+b$, $a,b\in\Z$. Suppose that $P=(x_P,y_P)\in E(\Q)^{tors}$.
    Then, $x_P,y_P\in\Z$ and either $y_P=0$, in which case $P$ has order 2, or
    $y_P^2|4a^3+27b^2$.
\end{thm}

\begin{thm}[Mazur]
    Given an elliptic curve $E/\Q$, we have that $E^{tors}(\Q)$ is either
    isomorphic to $\Z/n\Z$, where $1\leq n\leq 10$ or $n=12$, or to
    $\Z/2n\Z\oplus\Z/2\Z$, where $1\leq n\leq 4$.
\end{thm}

\begin{prop}
    Let $f$ be a non-zero elliptic function on a complex lattice $\Lambda$, $D$
    a fundamental domain for $\Lambda$ s.t. $f$ has no zeroes/poles on the
    boundary of $D$. Then:
    \begin{itemize}
        \item $\sum_{\gamma\in D} res_{\gamma}(f)=0$;
        \item $\sum_{\gamma\in D} ord_\gamma(f)=0$;
        \item $\sum_{\gamma\in D}ord_\gamma(f)\cdot\gamma=0\mod\Lambda$.
    \end{itemize}
\end{prop}

\printbibliography

\end{document}
