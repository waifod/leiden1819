\documentclass{article}
\usepackage[T1]{fontenc}
\usepackage{lmodern}
\usepackage[utf8]{inputenc}
\usepackage[british]{babel}
\usepackage{geometry}
\usepackage{color}
\usepackage{amsthm}
\usepackage{amsmath,amssymb}
\usepackage{graphicx}
\usepackage{mathtools}
\usepackage{listings}
\usepackage{newlfont}
\usepackage{tikz-cd}
\usepackage{rotating}
\usepackage[backend=biber]{biblatex}
\addbibresource{~/math/references.bib}

\newcommand{\numberset}{\mathbb}
\newcommand{\N}{\numberset{N}}
\newcommand{\Z}{\numberset{Z}}
\newcommand{\R}{\numberset{R}}
\newcommand{\Q}{\numberset{Q}}
\newcommand{\K}{\numberset{K}}
\newcommand{\F}{\numberset{F}}
\newcommand{\n}{\mathcal{N}}
\newcommand{\aid}{\mathfrak{a}}
\newcommand{\bid}{\mathfrak{b}}
\newcommand{\pid}{\mathfrak{p}}
\newcommand{\qid}{\mathfrak{q}}
\newcommand{\mi}{\mathfrak{m}}
\newcommand{\I}{\mathbb{I}}
\newcommand{\V}{\mathbb{V}}
\newcommand{\A}{\mathbb{A}}
\newcommand{\Ps}{\mathbb{P}}
\newcommand{\exercise}[1]{\noindent {\bf Exercise #1}}

\DeclareMathOperator{\im}{im}
\DeclareMathOperator{\coker}{coker}
\DeclareMathOperator{\Id}{Id}
\DeclareMathOperator{\GL}{GL}
\DeclareMathOperator{\Mat}{Mat}
\DeclareMathOperator{\Ext}{Ext}
\DeclareMathOperator{\Tor}{Tor}
\DeclareMathOperator{\Hom}{Hom}


\begin{document}

\title{Algebraic Topology II - Assignment 7}

\author{Matteo Durante, s2303760, Leiden University}

\maketitle

\exercise{2}

\begin{proof}
    $(a)$ It is sufficient to notice that, for any element
    $[f]\in\pi_n(S^n)\cong\Z$, we have by definition that
    $h_{S^n}([f])=f_*[S^n]=\deg(f)\cdot [S^n]$. Since
    $[\Id_{S^n}]\in\pi_n(S^n)$ is s.t. $\Id_{S^n}$ has degree 1 because it
    induces the identity isomorphism on $H_n(S^n)\cong\Z$, we have then the
    surjectivity.
\end{proof}

\begin{proof}
    $(b)$ We shall make use of the fact that, given a pointed map
    $X\xrightarrow{f}Y$, the induced natural map on the fibration sequences
    $\Omega X\rightarrow PX\rightarrow X,\ \Omega Y\rightarrow PY\rightarrow Y$
    gives natural maps $E^r_X\xrightarrow{f_*} E^r_Y$, i.e. s.t. the
    following diagram is commutative for every $r\in\N_{>0},\ i,j\in\N$:
    \[
        \begin{tikzcd}
            E^r_{X,i,j}\arrow{r}{d^r_X}\arrow{d}{f_*}
            & E^r_{X,i-r,i+1-r}\arrow{d}{f_*} \\
            E^r_{Y,i,j}\arrow{r}{d^r_Y}
            & E^r_{Y,i-r,i+1-r}
        \end{tikzcd}
    \]
    We already have the naturality of the isomorphism
    $\pi_1(X)^{ab}\cong H_1(X)$.

    We know that the isomorphism $\pi_{n-1}(\Omega X)\cong\pi_n(X)$ given by the
    connecting homomorphism in the long exact sequence of the fibration sequence
    $\Omega X\rightarrow PX\rightarrow X$ is natural.

    Notice that, since $\pi_n(X)$ is abelian for $n>1$, in the case $n=2$ have
    then a natural isomorphism $H_1(\Omega X)\cong\pi_1(\Omega X)^{ab}\cong
    \pi_2(X)^{ab}=\pi_2(X)$.
    
    Also, by what we stated earlier the differential
    $E^2_{20}=H_2(X)\xrightarrow{d_2}E^2_{01}=H_1(\Omega X)$ is natural.
    By the arguments provided in~\cite[thm. 11.6]{HM19}, it is also an
    isomorphism.

    Reversing the natural isomorphism $H_2(X)\rightarrow
    H_1(\Omega X)$ and composing it with $\pi_2(X)\rightarrow
    H_1(\Omega X)$ we get then the desired natural isomorphism
    $\pi_2(X)\rightarrow H_2(X)$.

    Supposing now the result true for some $m>1$, we will prove the general
    case.

    Let $X$ be a space satisfying the hypothesis of~\cite[thm. 11.6]{HM19} for
    $n=m+1$. We know that $\pi_n(X)\cong\pi_{n-1}(\Omega X)\cong
    H_{n-1}(\Omega X)$ naturally by inductive hypothesis since $\pi_k(\Omega X)=
    \pi_{k+1}(X)=0$ for $k<n$.

    Again, by what we stated earlier the differential
    $E^2_{n0}=H_n(X)\xrightarrow{d_n}E^2_{0,n-1}=H_{n-1}(\Omega X)$ is natural.
    By the arguments provided in~\cite[thm. 11.6]{HM19}, it is also an
    isomorphism.

    Reversing the natural isomorphisms $H_n(X)\rightarrow H_{n-1}(\Omega X)$ and
    composing it with $\pi_n(X)\rightarrow H_{n-1}(\Omega X)$ we get then the
    desired natural isomorphism $\pi_n(X)\rightarrow H_n(X)$.
\end{proof}

\begin{proof}
    $(c)$ The two maps $g_{S^n},\ h_{S^n}$ trivially agree up to sign, for they
    are isomorphisms from $\pi_n(S^n)\cong\Z$ to $H_n(S^n)\cong\Z$.

    Observe that the isomorphism induced by $h_X$ is natural, for given a
    pointed map $X\xrightarrow{f} Y$ we have that
    $h_Y(f_*[\alpha])=h_Y([f\circ\alpha])=(f\circ\alpha)_*[S^n]=
    f_*(\alpha_*[S^n])=f_*(h_X([\alpha]))$.

    Let's assume $g_{S^n}=h_{S^n}$.

    Now, given any element $[f]\in\pi_n(X)$, considering the map given by a
    representative $S^n\xrightarrow{f}X$ and making use of the naturality of the
    maps $g_{S^n},\ g_X,\ h_{S^n},\ h_X$, we have the following commutative
    diagrams:
    \[
        \begin{tikzcd}
            \pi_n(S^n)\arrow{r}{g_{S^n}}\arrow{d}{f_*}
            & H_n(S^n)\arrow{d}{f_*} \\
            \pi_n(X)\arrow{r}{g_X}
            & H_n(X)
        \end{tikzcd}
        \quad
        \begin{tikzcd}
            \pi_n(S^n)\arrow{r}{h_{S^n}}\arrow{d}{f_*}
            & H_n(S^n)\arrow{d}{f_*} \\
            \pi_n(X)\arrow{r}{h_X}
            & H_n(X)
        \end{tikzcd}
    \]
    We have then that:
    \begin{align*}
        g_X([f]) &=g_X(f_*[\Id_{S^n}]) \\
        &=f_*(g_{S^n}([\Id_{S^n}])) \\
        &=f_*(h_{S^n}([\Id_{S^n}])) \\
        &=h_X(f_*[\Id_{S^n}]) \\
        &=h_X([f])
    \end{align*}
    The discussion of the case $g_{S^n}=-h_{S^n}$ is essentially analogous and
    leads to $g_X=-h_X$.

    Now, since $g_X=\pm h_X$ on every $n-1$ connected pointed space for $n>1$
    and $h_X$ is an isomorphism by~\cite[thm. 11.6]{HM19}, we have that $g_X$ is
    also an isomorphism, hence the thesis.
\end{proof}


~\\
\exercise{3}

\begin{proof}    
    By the usual argument about cellular maps, $\pi_t(X)=0$ for $t<n$.
    
    Since $X$ is pointed and simply connected, by~\cite[thm. 12.1]{HM19} and the
    computation of $H_*(X)$ we will provide,
    all of the homotopy groups of $X$ are abelian and finitely generated, hence
    they can be described as $\pi_t(X)=\Z^r\oplus\pi_t(X)^{tors}$ for some
    $r\in\N$. Also, $\pi_t(X)\otimes\Q=\Q^r$. We will then work with the
    Hurewicz theorem$\mod\mathcal{C}$, where $\mathcal{C}$ is the class of
    torsion abelian groups.
    
    Let's compute $H_t(X)$ for all $t,\ n,\ k$.

    Using the description of $X$ as a finite CW-complex, we see that its
    homology corresponds to the homology of the cellular chain complex
    $(C_\bullet,\partial)$, where $C_0=C_n=C_{n+1}=\Z$, $C_t=0$ for
    $t\neq 0,n,n+1$ and $C_{n+1}\xrightarrow{\partial_n}C_n$ is given by
    $m\mapsto km$. It follows that $H_n(X)=\Z/k\Z\in\mathcal{C},\ H_0(X)=\Z,\
    H_t(X)=0$ for $t\neq 0,n$.

    By Hurewicz, $\pi_n(X)=H_n(X)=\Z/k\Z$.

    We also have that $P_nX$ is a $K(\Z/k\Z,n)$. We may then consider the
    fibration sequence $X\langle n\rangle\rightarrow X\rightarrow K(\Z/k\Z,n)$,
    which gives us the following one: $\Omega
    K(\Z/k\Z,n)=K(\Z/k\Z,n-1)\rightarrow X\langle n\rangle\rightarrow X$.

    By~\cite[lemma 13.16]{HM19}, $H_t(K(\Z/k\Z,m))\in\mathcal{C}$ for all
    $t\in\N,\ m\in\N_{>0}$ and by~\cite[lemma 13.15]{HM19} the same goes for
    $H_t(X\langle n\rangle)$, which in particular gives
    $H_{n+1}(X\langle n\rangle)=\pi_{n+1}(X\langle
    n\rangle)=\pi_{n+1}(X)\in\mathcal{C}$.

    Assume now that $H_t(X\langle i-1\rangle)\in\mathcal{C}$ for all
    $t\in\N_{>0}$ for some $i>n$. We will show that $H_t(X\langle
    i\rangle)\in\mathcal{C}$ for all $t\in\N_{>0}$ as well.

    Consider the fibration sequence $F\rightarrow X\langle i\rangle\rightarrow
    X\langle i-1\rangle$, where $F$ is the homotopy fiber. By looking at the
    long exact sequence of the homotopy groups, we see that $F$ is a
    $K(\pi_{i-1}(X),i-1)$, hence $H_t(F)\in\mathcal{C}$ for all $t\in\N_{>0}$
    by~\cite[lemma 13.16]{HM19}. Again, by~\cite[lemma 13.15]{HM19}, this
    implies that $H_t(X\langle i\rangle)\in\mathcal{C}$ for all $t\in\N_{>0}$.
    
    It follows that $H_{i+1}(X\langle i\rangle)=\pi_{i+1}(X\langle
    i\rangle)=\pi_{i+1}(X)\in\mathcal{C}$, thus we can conclude that
    $\pi_i(X)\otimes\Q=0$ for all $i>0$.
\end{proof}

\printbibliography

\end{document}
