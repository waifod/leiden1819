\documentclass{article}
\usepackage[T1]{fontenc}
\usepackage{lmodern}
\usepackage[utf8]{inputenc}
\usepackage[british]{babel}
\usepackage{geometry}
\usepackage{color}
\usepackage{amsthm}
\usepackage{amsmath,amssymb}
\usepackage{graphicx}
\usepackage{mathtools}
\usepackage{listings}
\usepackage{newlfont}
\usepackage{tikz-cd}
\usepackage{rotating}
\usepackage[backend=biber]{biblatex}
\addbibresource{~/math/references.bib}

\newcommand{\numberset}{\mathbb}
\newcommand{\N}{\numberset{N}}
\newcommand{\Z}{\numberset{Z}}
\newcommand{\R}{\numberset{R}}
\newcommand{\Q}{\numberset{Q}}
\newcommand{\K}{\numberset{K}}
\newcommand{\F}{\numberset{F}}
\newcommand{\n}{\mathcal{N}}
\newcommand{\aid}{\mathfrak{a}}
\newcommand{\bid}{\mathfrak{b}}
\newcommand{\pid}{\mathfrak{p}}
\newcommand{\qid}{\mathfrak{q}}
\newcommand{\mi}{\mathfrak{m}}
\newcommand{\I}{\mathbb{I}}
\newcommand{\V}{\mathbb{V}}
\newcommand{\A}{\mathbb{A}}
\newcommand{\Ps}{\mathbb{P}}
\newcommand{\exercise}[1]{\noindent {\bf Exercise #1}}

\DeclareMathOperator{\im}{im}
\DeclareMathOperator{\coker}{coker}
\DeclareMathOperator{\Id}{Id}
\DeclareMathOperator{\GL}{GL}
\DeclareMathOperator{\Mat}{Mat}
\DeclareMathOperator{\Ext}{Ext}
\DeclareMathOperator{\Tor}{Tor}
\DeclareMathOperator{\Hom}{Hom}


\begin{document}

\title{Algebraic Topology II - Assignment 7}

\author{Matteo Durante, s2303760, Leiden University}

\maketitle

\exercise{2}

\begin{proof}
    $(a)$ It is sufficient to notice that, for any element
    $[f]\in\pi_n(S^n)\cong\Z$, we have by definition that
    $h_{S^n}([f])=f_*([\alpha])=\deg(f)\cdot [\alpha]$. Since
    $[\Id_{S^n}]\in\pi_n(S^n)$ is s.t. $\Id_{S^n}$ has degree 1 because it
    induces the identity isomorphism on $H_n(S^n)\cong\Z$, we have then the
    surjectivity.
\end{proof}

\begin{proof}
    $(c)$ The two maps trivially agree up to sign, for they are isomorphisms
    from $\pi_n(S^n)\cong\Z$ to $H_n(S^n)\cong\Z$.
\end{proof}


~\\
\exercise{3}

\begin{proof}
    First of all, we shall compute $H_n(X)\otimes\Q$ for all $n$ and $k$.

    Using the description of $X$ as a finite CW-complex, we see that its
    homology corresponds to the homology of the chain complex
    $(C_\bullet,\partial)$, where $C_0=\Z$, $C_n=\Z$, $C_{n+1}$ and
    $C_{n+1}\xrightarrow{\partial_n}C_n$ is given by $m\mapsto km$. It follows
    that $H_n(X)=\Z/k\Z,\ H_0(X)=\Z,\ H_m(X)=0$ for $m\neq 0,n$ and
    $H_0(X)\otimes\Q=\Q,\ H_t(X)\otimes\Q=0$ for all other $t$.

    By the usual argument about cellular maps, $\pi_m(X)=0=\pi_m(X)\otimes\Q$
    for $m<n$. By Hurewicz, $\pi_n(X)=H_n(X)=\Z/k\Z$ and $\pi_n(X)\otimes\Q=0$.

    We also have that $P_nX$ is a $K(\Z/k\Z,n)$. We may then consider the
    fibration sequence $X\langle n\rangle\rightarrow X\rightarrow K(\Z/k\Z,n)$,
    which then gives us the following one: $\Omega
    K(\Z/k\Z,n)=K(\Z/k\Z,n-1)\rightarrow X\langle n\rangle\rightarrow X$.

    We want to compute $\pi_{n+1}(X)=\pi_{n+1}(X\langle
    n\rangle)=H_{n+1}(X\langle n\rangle)$ and then tensor it by $\Q$. by looking
    at the converging Serre spectral sequence induced by our fibration. It is
    s.t. $E_{ij}^2=H_i(X,H_j(K(\Z/k\Z,n-1)))\Rightarrow H_{i+j}(X\langle
    n\rangle)$.
    
    Since
    $\Q$ is a flat $\Z$-module, $-\otimes\Q$ is an exact functor, not to mention
    a left adjoint, hence it commutes with taking quotients and direct sums (and
    finite products, since these are also direct sums in an abelian category).

    It follows that we may start from the page $E_2$ we get by tensoring every
    group $E^{ij}_2$ with $\Q$ and it will converge to the $E_\infty$ page we
    get by tensoring the groups $E^{ij}_\infty$ with $\Q$.

    We want to compute $E_{ij}^\infty$ for $i+j=n+1$. Since a $\Q$-module is a
    $\Q$-vector space, we will have that $H_{n+1}(X\langle
    n\rangle)=\bigoplus_{i=0}^{n+1}E_{ij}^\infty$.
\end{proof}

\printbibliography

\end{document}


