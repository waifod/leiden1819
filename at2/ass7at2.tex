\documentclass{article}
\usepackage[T1]{fontenc}
\usepackage{lmodern}
\usepackage[utf8]{inputenc}
\usepackage[british]{babel}
\usepackage{geometry}
\usepackage{color}
\usepackage{amsthm}
\usepackage{amsmath,amssymb}
\usepackage{graphicx}
\usepackage{mathtools}
\usepackage{listings}
\usepackage{newlfont}
\usepackage{tikz-cd}
\usepackage{rotating}
\usepackage[backend=biber]{biblatex}
\addbibresource{~/math/references.bib}

\newcommand{\numberset}{\mathbb}
\newcommand{\N}{\numberset{N}}
\newcommand{\Z}{\numberset{Z}}
\newcommand{\R}{\numberset{R}}
\newcommand{\Q}{\numberset{Q}}
\newcommand{\K}{\numberset{K}}
\newcommand{\F}{\numberset{F}}
\newcommand{\n}{\mathcal{N}}
\newcommand{\aid}{\mathfrak{a}}
\newcommand{\bid}{\mathfrak{b}}
\newcommand{\pid}{\mathfrak{p}}
\newcommand{\qid}{\mathfrak{q}}
\newcommand{\mi}{\mathfrak{m}}
\newcommand{\I}{\mathbb{I}}
\newcommand{\V}{\mathbb{V}}
\newcommand{\A}{\mathbb{A}}
\newcommand{\Ps}{\mathbb{P}}
\newcommand{\exercise}[1]{\noindent {\bf Exercise #1}}

\DeclareMathOperator{\im}{im}
\DeclareMathOperator{\coker}{coker}
\DeclareMathOperator{\Id}{Id}
\DeclareMathOperator{\GL}{GL}
\DeclareMathOperator{\Mat}{Mat}
\DeclareMathOperator{\Ext}{Ext}
\DeclareMathOperator{\Tor}{Tor}
\DeclareMathOperator{\Hom}{Hom}


\begin{document}

\title{Algebraic Topology II - Assignment 7}

\author{Matteo Durante, s2303760, Leiden University}

\maketitle

\exercise{2}

\begin{proof}
    $(a)$ It is sufficient to notice that, for any element
    $[f]\in\pi_n(S^n)\cong\Z$, we have by definition that
    $h_{S^n}([f])=f_*([\alpha])=\deg(f)\cdot [\alpha]$. Since
    $[\Id_{S^n}]\in\pi_n(S^n)$ is s.t. $\Id_{S^n}$ has degree 1 because it
    induces the identity isomorphism on $H_n(S^n)\cong\Z$, we have then the
    surjectivity.
\end{proof}

\begin{proof}
    $(c)$ The two maps trivially agree up to sign, for they are isomorphisms
    from $\pi_n(S^n)\cong\Z$ to $H_n(S^n)\cong\Z$.
\end{proof}


~\\
\exercise{3}

\begin{proof}    
    By the usual argument about cellular maps, $\pi_m(X)=0=\pi_m(X)\otimes\Q$
    for $m<n$.
    
    By~\cite[thm. 12.1]{HM19}, all of the homotopy groups of $X$ are abelian and
    finitely generated, hence they can be described as
    $\pi_k(X)=\Z^r\oplus\pi_k(X)^{tors}$ for some $r\in\N$. Also,
    $\pi_k(X)\otimes\Q=\Q^r$. We will then work with the Hurewicz
    theorem$\mod\mathcal{C}$, where $\mathcal{C}$ is the class of torsion
    abelian groups.
    
    First of all, we shall compute $H_n(X)\otimes\Q$ for all $n$ and $k$.

    Using the description of $X$ as a finite CW-complex, we see that its
    homology corresponds to the homology of the cellular chain complex
    $(C_\bullet,\partial)$, where $C_0=\Z$, $C_n=\Z$, $C_{n+1}$ and
    $C_{n+1}\xrightarrow{\partial_n}C_n$ is given by $m\mapsto km$. It follows
    that $H_n(X)=\Z/k\Z\in\mathcal{C},\ H_0(X)=\Z,\ H_m(X)=0$ for $m\neq 0,n$
    and $H_0(X)\otimes\Q=\Q,\ H_t(X)\otimes\Q=0$ for all other $t$.

    By Hurewicz, $\pi_n(X)=H_n(X)=\Z/k\Z$ and $\pi_n(X)\otimes\Q=0$.

    We also have that $P_nX$ is a $K(\Z/k\Z,n)$. We may then consider the
    fibration sequence $X\langle n\rangle\rightarrow X\rightarrow K(\Z/k\Z,n)$,
    which then gives us the following one: $\Omega
    K(\Z/k\Z,n)=K(\Z/k\Z,n-1)\rightarrow X\langle n\rangle\rightarrow X$.

    By~\cite[lemma 13.16]{HM19}, $H_n(K(\Z/k\Z,m))\in\mathcal{C}$ for all
    $m\in\N_{>0}$ and by~\cite[lemma 13.15]{HM19} the same goes for
    $\pi_{n+1}(X\langle n\rangle)=\pi_{n+1}(X)$. It follows that
    $\pi_{n+1}(X)\otimes\Q=0$.
\end{proof}

\printbibliography

\end{document}


