\documentclass{article}
\usepackage[T1]{fontenc}
\usepackage{lmodern}
\usepackage[utf8]{inputenc}
\usepackage[british]{babel}
\usepackage{geometry}
\usepackage{color}
\usepackage{amsthm}
\usepackage{amsmath,amssymb}
\usepackage{graphicx}
\usepackage{mathtools}
\usepackage{listings}
\usepackage{newlfont}
\usepackage{tikz-cd}
\usepackage{rotating}
\usepackage[backend=biber]{biblatex}
\addbibresource{~/math/references.bib}

\newcommand{\numberset}{\mathbb}
\newcommand{\N}{\numberset{N}}
\newcommand{\Z}{\numberset{Z}}
\newcommand{\R}{\numberset{R}}
\newcommand{\Q}{\numberset{Q}}
\newcommand{\K}{\numberset{K}}
\newcommand{\F}{\numberset{F}}
\newcommand{\n}{\mathcal{N}}
\newcommand{\aid}{\mathfrak{a}}
\newcommand{\bid}{\mathfrak{b}}
\newcommand{\pid}{\mathfrak{p}}
\newcommand{\qid}{\mathfrak{q}}
\newcommand{\mi}{\mathfrak{m}}
\newcommand{\I}{\mathbb{I}}
\newcommand{\V}{\mathbb{V}}
\newcommand{\A}{\mathbb{A}}
\newcommand{\Ps}{\mathbb{P}}
\newcommand{\RP}{\R P^\infty}
\newcommand{\exercise}[1]{\noindent {\bf Exercise #1}}

\DeclareMathOperator{\im}{im}
\DeclareMathOperator{\coker}{coker}
\DeclareMathOperator{\Id}{Id}
\DeclareMathOperator{\GL}{GL}
\DeclareMathOperator{\Mat}{Mat}
\DeclareMathOperator{\Ext}{Ext}
\DeclareMathOperator{\Tor}{Tor}
\DeclareMathOperator{\Hom}{Hom}


\begin{document}

\title{}

\author{Matteo Durante, s2303760, Leiden University}

\maketitle

\exercise{2}

    We will use the fact that we are working with characteristic 2 to avoid
    distinguishing between the signs of the terms, s.t. the Leibniz rule and the
    cup products will be easier to write down.

\begin{proof}
    Let's consider the path fibration $K(\Z/2\Z,1)\rightarrow
    PK(\Z/2\Z,1)\rightarrow K(\Z/2\Z,2)$. Since $PK(\Z/2\Z,1)$ is contractible,
    we know that $E^{ij}_2=H^i(K(\Z/2\Z,2),H^j(K(\Z/2\Z,1),\Z/2\Z))\Rightarrow
    H^{i+j}(PK(\Z/2\Z,1),\Z/2\Z)$ by~\cite[thm. 9.5]{HM19}, hence the
    $E_\infty$-page is 0 everywhere but at $(0,0)$, where there is $\Z/2\Z$.

    We have that $K(\Z/2\Z,1)\cong\RP$ with $H^*(\RP,\Z/2\Z)=(\Z/2\Z)[a]$ for an
    element $a$ of degree 1 and $H^j(\RP,\Z/2\Z)=\Z/2\Z\cdot a^j$ for all
    $j\in\N$. It follows that $E^{ij}_2=H^i(K(\Z/2\Z,2),\Z/2\Z)\cdot a^j$.

    Fixed $i$, we will be computing each $E^{ij}_2$ by determining $E^{i0}_2$
    and then we will move on to the following integer.

    We start by computing $E^{0j}_2$, which is actually already given as
    $H^0(K(\Z/2\Z,2),\Z/2\Z)\cdot a^j=\Z/2\Z\cdot a^j$.

    Let now $i=1$.
    
    No arrows will ever go into the $(1,0)$ position and all arrows from there
    will end up below the $x$-axis for $d\geq 2$, hence
    $E^{10}_2=E^{10}_\infty=0$. It follows that $H^i(K(\Z/2\Z,2),\Z/2\Z)=0$ and
    therefore $E^{1j}_2=0$ for all $j\in\N$.

    Let now $i=2$.

    Again, there are no arrows into the $(2,0)$-position and for $d>2$ all of
    the ones from there end up below the $x$-axis, hence
    $E^{01}_2\xrightarrow{d_2}E^{20}_2$ has to be surjective for
    $\coker(d_2)=E^{20}_3=E^{20}_\infty=0$. Since this is the only arrow from
    the $(0,1)$-position which does not end up below the $x$-axis, by the same
    reasoning it has to be also injective,
    thus it is an isomorphism $(*)$. Let $x\in E^{20}_2$ be the generating
    element s.t. $d_2(a)=x$. We then have that $E^{2j}_2=\Z/2\Z\cdot xa^j$.
    Also, $d_2(x)=0$ as $d_2(E^{20}_2)=0$.
    
    Let now $i=3$.

    All of the arrows from the $(3,0)$-position end up below the $x$-axis and
    there are no arrows going to the $(3,0)$-position besides $d_2$ and $d_3$.
    However, $d_2$ has as domain $E^{11}_2=0$, thus $E^{30}_2=E^{30}_3$.

    Let's compute $E^{02}_3=\ker(E^{02}_2\xrightarrow{d_2}E^{21}_2)$. We know
    that $E^{02}_2=\Z/2\Z\cdot a^2$ and $d_2(a^2)=d_2(a)\cdot a+a\cdot
    d_2(a)=2a\cdot d_2(a)=0$, thus $E^{02}_3=E^{02}_2$. By a previous argument
    $(*)$, it follows that $d_3$ is an isomorphism. Let $y\in E^{30}_3$ be the
    generating element s.t. $d_3(a^2)=y$. It follows that
    $E^{3j}_2=E^{3j}_3=\Z/2\Z\cdot ya^j$ for all $j$.

    Let now $i=4$.

    Observe that, for $r>2$, no arrow goes into the $(2,1)$-position and all of
    the ones from there end up below the $x$-axis, hence
    $E^{21}_3=E^{21}_\infty=0$. By definition, this means that
    $\ker(E^{21}_2\xrightarrow{d_2}E^{40}_2)=\im(E^{02}_2\xrightarrow{d_2}E^{21}_2)$,
    and, since $E^{02}_2\xrightarrow{d_2}E^{21}_2$ is the zero-map,
    $E^{21}_2\xrightarrow{d_2}E^{40}_2$ is injective.

    By definition, $E^{40}_3=E^{40}_2/\im(E^{21}_2\xrightarrow{d_2}E^{40}_2)$.
    Also, $E^{40}_5=E^{40}_4/\im(E^{03}_4\xrightarrow{d_4}E^{40}_4)$. We will
    compute $E^{03}_4$.

    $d_2(a^3)=d_2(a^2)\cdot a+a\cdot d_2(a^2)=d_2(a)\cdot a^2=xa^2$, hence
    $E^{03}_2\xrightarrow{d_2}E^{22}_2$ is an isomorphism. It follows that
    $E^{03}_3=E^{03}_4=0$.
    
    Also, $\im(E^{03}_4\xrightarrow{d_4}E^{40}_4)=0$. Since for $r>4$ no
    arrow goes into the $(4,0)$-position and any arrow from there ends up below
    the $x$-axis, we have that
    $E^{40}_4=E^{40}_4/\im(E^{03}_4\xrightarrow{d_4}E^{40}_4)=
    E^{40}_5=E^{40}_\infty=0$. Since $E^{12}_3=0$, this means that
    $0=E^{40}_4=E^{40}_3/\im(E^{12}_3\xrightarrow{d_3}E^{40}_3)=E^{40}_3$, which
    implies that $E^{21}_2\xrightarrow{d_2}E^{40}_2$ is also surjective and
    therefore an isomorphism.

    Observe that $E^{21}_2=\Z/2\Z\cdot xa$ and $d_2(ax)=d_2(x)\cdot a+x\cdot
    d_2(a)=x^2$, thus $E^{40}_2=\Z/2\Z\cdot x^2$ and
    $E^{4j}_2=\Z/2\Z\cdot x^2a^j$ for all $j\in\N$.

    Let now $i=5$.

    By definition, $E^{50}_3=E^{50}_2/\im(E^{31}_2\xrightarrow{d_2}E^{50}_2)$,
    $E^{50}_4=E^{50}_3/\im(E^{22}_3\xrightarrow{d_3}E^{50}_2)$,
    $E^{50}_5=E^{50}_4/\im(E^{13}_4\xrightarrow{d_4}E^{50}_4)=E^{50}_4$,
    $E^{50}_6=E^{50}_5/\im(E^{04}_5\xrightarrow{d_5}E^{50}_5)$. Since there are
    no other non-zero arrows to and from the $(5,0)$-position, we have that
    $E^{50}_6=E^{50}_\infty=0$, hence $d_5$ is surjective.

    By the same reasoning,
    $0=E^{31}_\infty=E^{31}_4=E^{31}_3/\im(E^{03}_3\xrightarrow{d_3}E^{31}_3)$,
    which means that $d_3$ is surjective. Since $E^{03}_2=\Z/2\Z\cdot
    a^3\xrightarrow{d_2} E^{22}_2=\Z/2\Z\cdot xa^2$ is an isomorphism as
    $d_2(a^3)=d_2(a)\cdot a^2+a\cdot d_2(a^2)=xa^2$, it follows that
    $E^{03}_3=0$ and therefore $E^{31}_3=0$.

    By definition, we have that
    $0=E^{31}_3=\ker(E^{31}_2\xrightarrow{d_2}E^{50}_2)/
    \im(E^{12}_2\xrightarrow{d_2}E^{31}_2)
    =\ker(E^{31}_2\xrightarrow{d_2}E^{50}_2)$, thus
    $E^{31}_2\xrightarrow{d_2}E^{50}_2$ is injective.
    
    Remember that $E^{31}_2=\Z/2\Z\cdot ya,\ E^{30}_2=\Z/2\Z\cdot y$,
    $d_2(E^{30}_2)=0$ (it falls below the $x$-axis) 
    and therefore $d_2(y)=0$, hence $d_2(ya)=d_2(y)\cdot a+y\cdot d_2(a)=yx=xy$.
    By the injectivity of $E^{31}_2\xrightarrow{d_2}E^{50}_2$, it follows that
    $0\neq d_2(ya)=xy\in E^{50}_2$ and $E^{50}_3=E^{50}_2/(\Z/2\Z\cdot xy)$.

    As shown earlier, $d_2(a^3)=xa^2$. Also, $d_2(xa^2)=d_2(x)\cdot a^2+x\cdot
    d_2(a^2)=d_2(d_2(a^2))\cdot a^2=0$, thus
    $E^{22}_3=\ker(E^{22}_2=\Z/2\Z\cdot xa^2\xrightarrow{d_2}E^{41}_2)/
    \im(E^{03}_2=\Z/2\Z\cdot a^3\xrightarrow{d_2}E^{22}_2=\Z/2\Z\cdot
    xa^2)=\Z/2\Z\cdot xa^2/\Z/2\Z\cdot xa^2=0$. This implies that
    $E^{50}_3=E^{50}_4$, which is also $E^{50}_5$.

    Now, $d_2(a^4)=d_2(a^2)\cdot a^2+a^2\cdot d_2(a^2)=0$ and therefore
    $E^{04}_3=\ker(E^{04}_2=\Z/2\Z\cdot a^4\xrightarrow{d_2}E^{23}_2)=E^{04}_2$.

    We know that $E^{02}_3=\Z/2\Z\cdot a^2$, thus $d_3(a^4)=d_3(a^2)\cdot
    a^2+a^2\cdot d_3(a^2)=2a^2\cdot d_3(a^2)=0$, hence
    $E^{04}_4=\ker(E^{04}_3=\Z/2\Z\cdot
    a^4\xrightarrow{d_3}E^{32}_3)=\Z/2\Z\cdot a^4$. 
    
    Also,
    $E^{04}_5=\ker(E^{04}_4\xrightarrow{d_4}E^{41}_4)=\Z/2\Z\cdot a^4$ because
    $0=E^{41}_\infty=E^{41}_5=E^{41}_4/\im(E^{04}_4\xrightarrow{d_4}E^{41}_4)$,
    and $E^{41}_4=0$ ($E^{41}_3=\ker(E^{41}_2\xrightarrow{d_2}E^{60}_2)/
    \im(E^{22}_2\xrightarrow{d_2}E^{41}_2)=0$ because $d_2(x^2a)=d_2(x^2)\cdot
    a+x^2\cdot d_2(a)=x^3\neq 0$ $(**)$).
    
    Notice that $E^{04}_5\xrightarrow{d_5}E^{50}_5$ is an isomorphism, for this
    is the last non-zero arrow from or to the $(0,4)$ and the $(5,0)$-positions.
    It follows that $E^{50}_2=\Z/2\Z\cdot z$, where $z=d_5(a^4)$. We then have
    that $E^{50}_2/\Z/2\Z\cdot xy=E^{50}_3=E^{50}_4=E^{50}_5=\Z/2\Z\cdot z$,
    which implies that $E^{50}_2=\Z/2\Z\cdot xy\oplus\Z/2\Z\cdot z$ because we
    are working with $\F_2$-vector spaces.

    Finally, $E^{5j}_2=\Z/2\Z\cdot xya^j\oplus\Z/2\Z\cdot za^j$ for every
    $j\in\N$.

    Let now $i=6$.

    By definition, $E^{60}_3=E^{60}_2/\im(E^{41}_2\xrightarrow{d_2}E^{60}_2)$,
    $E^{60}_4=E^{60}_3/\im(E^{32}_3\xrightarrow{d_3}E^{60}_3)$,
    $E^{60}_5=E^{60}_4/\im(E^{23}_4\xrightarrow{d_4}E^{60}_4)$,
    $E^{60}_6=E^{60}_5/\im(E^{14}_5\xrightarrow{d_5}E^{60}_5)$,
    $0=E^{60}_\infty=E^{60}_7=E^{60}_6/\im(E^{05}_6\xrightarrow{d_6}E^{60}_6)$.

    We know that
    $0=E^{41}_4=E^{41}_3/\im(E^{13}_3\xrightarrow{d_3}E^{41}_3)=E^{41}_3$ and
    $E^{41}_3=\ker(E^{41}_2\xrightarrow{d_2}E^{60}_2)/
    \im(E^{22}_2\xrightarrow{d_2}E^{41}_2)=\ker(E^{41}_2\xrightarrow{d_2}E^{60}_2)$
    because $E^{22}_2=\Z/2\Z\cdot xa^2$ and $d_2(xa^2)=0$.

    It follows that $\ker(E^{41}_2\xrightarrow{d_2}E^{60}_2)=0$,
    $\im(E^{41}_2=\Z/2\Z\cdot x^2a\xrightarrow{d_2}E^{60}_2)=\Z/2\Z\cdot x^3$ as
    $d_2(x^2a)=d_2(x^2)\cdot a+x^2\cdot d_2(a)=d_2(d_2(a^4))+x^3=x^3$ and
    $E^{60}_3=E^{60}_2/(\Z/2\Z\cdot x^3)$. $(**)$ Keep in mind that $x^3\neq 0$
    because the map is injective (the group has to vanish because $E^{41}_4=0$
    and the only other possibly non-zero arrow to
    or from the $(4,1)$ position is $E^{13}_3\xrightarrow{d_3}E^{41}_3$, which
    is however 0 because $E^{13}_3=0$; on the other hand, the map $E^{22}_2
    \xrightarrow{d_2}E^{41}_2$ is zero because $d_2(xa^2)=d_2(x)\cdot a^2+x\cdot
    d_2(a^2)=0$, hence it does not contribute to killing $E^{41}_2$).

    Let's compute $E^{23}_4$. We know that $E^{23}_2=\Z/2\Z\cdot xa^3$,
    $\ker(E^{23}_2\xrightarrow{d_2}E^{42}_2)/\im(E^{04}_2=\Z/2\Z\cdot a^4
    \xrightarrow{d_2}E^{23}_2)=\ker(E^{23}_2\xrightarrow{d_2}E^{42}_2)
    E^{23}_2=0$ as $d_2(xa^3)=d_2(x)\cdot a^3+x\cdot d_2(a^3)=3x^2a^2=a^2x^2$,
    which means that $E^{23}_2\xrightarrow{d_2}E^{42}_2$ is
    an isomorphism and $E^{23}_3=E^{42}_3=0$. It follows that $E^{23}_4=0$,
    hence $E^{60}_5=E^{60}_4/\im(E^{23}_4\xrightarrow{d_4}E^{60}_4)=E^{60}_4$.

    We see that
    $E^{60}_6=E^{60}_5/\im(E^{14}_5\xrightarrow{d_5}E^{60}_5)=E^{60}_5$ because
    $E^{14}_5=0$.

    $E^{05}_2=\Z\cdot a^5$,
    $E^{05}_3=\ker(E^{05}_2\xrightarrow{d_2}E^{24}_2)=0$ as
    $d_2(a^5)=5a^4\cdot d_2(a)=xa^4$ and therefore $d_2$ is again an
    isomorphism. It follows that $E^{50}_5=0$, thus $E^{60}_5=E^{60}_6=0$.

    So far we have shown that $0=E^{60}_5=E^{60}_4$, hence
    $\im(E^{32}_3\xrightarrow{d_3}E^{60}_3)=E^{60}_3$. We know that
    $E^{32}_2=\Z/2\Z\cdot ya^2$, $d_2(ya^2)=d_2(y)\cdot a^2+y\cdot d_2(a^2)=0$
    and the map $E^{13}_2\xrightarrow{d_2}E^{32}_2$ is zero, hence
    $E^{32}_3=E^{32}_2=\Z/2\Z\cdot ya^2$. We have that $d_3(ya^2)=d_3(y)\cdot
    a^2+y\cdot d_3(a^2)=d_3(d_3(a^2))+y^2=y^2$, hence $E^{60}_4=\Z/2\Z\cdot
    y^2$. Also, since there are no more non-zero arrows into or from the
    $(3,2)$-position, the map has to be injective and have that $y^2\neq 0$.
    
    It follows that $E^{60}_2=\Z/2\Z\cdot x^3\oplus\Z/2\Z\cdot y^2$ as we
    are still working with $\F_2$-vector spaces. We get 
    $E^{6j}_2=\Z/2\Z\cdot x^3a^j\oplus\Z/2\Z\cdot y^2a^j$ for all $j\in\N$.

    We can conclude that:
    \begin{itemize}
        \item $H^0(K(\Z/2\Z),\Z/2\Z)=E^{00}_2=\Z/2\Z$;
        \item $H^1(K(\Z/2\Z),\Z/2\Z)=E^{10}_2=0$;
        \item $H^2(K(\Z/2\Z),\Z/2\Z)=E^{20}_2=\Z/2\Z\cdot x$, where
            $x=d_2(a)$, with $a$ the generator of $E^{01}_2$;
        \item $H^3(K(\Z/2\Z),\Z/2\Z)=E^{30}_2=\Z/2\Z\cdot y$, where
            $y=d_3(a^2)$;
        \item $H^4(K(\Z/2\Z),\Z/2\Z)=E^{40}_2=\Z/2\Z\cdot x^2$
        \item $H^5(K(\Z/2\Z),\Z/2\Z)=E^{50}_2=\Z/2\Z\cdot xy\oplus\Z/2\Z\cdot
            z$, where $z=d_5(a^4)$;
        \item $H^6(K(\Z/2\Z),\Z/2\Z)=E^{60}_2=\Z/2\Z\cdot
            x^3\oplus\Z/2\Z\cdot y^2$.
    \end{itemize}
\end{proof}

\printbibliography

\end{document}
