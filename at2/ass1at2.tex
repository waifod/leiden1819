\documentclass{article}
\usepackage[T1]{fontenc}
\usepackage{lmodern}
\usepackage[utf8]{inputenc}
\usepackage[british]{babel}
\usepackage{geometry}
\usepackage{color}
\usepackage{amsthm}
\usepackage{amsmath,amssymb}
\usepackage{graphicx}
\usepackage{mathtools}
\usepackage{listings}
\usepackage{newlfont}
\usepackage{tikz-cd}
\usepackage{rotating}
\usepackage[backend=biber]{biblatex}
\addbibresource{~/documents/math/references.bib}

\newcommand{\numberset}{\mathbb}
\newcommand{\N}{\numberset{N}}
\newcommand{\Z}{\numberset{Z}}
\newcommand{\R}{\numberset{R}}
\newcommand{\Q}{\numberset{Q}}
\newcommand{\K}{\numberset{K}}
\newcommand{\F}{\numberset{F}}
\newcommand{\n}{\mathcal{N}}
\newcommand{\aid}{\mathfrak{a}}
\newcommand{\bid}{\mathfrak{b}}
\newcommand{\pid}{\mathfrak{p}}
\newcommand{\qid}{\mathfrak{q}}
\newcommand{\mi}{\mathfrak{m}}
\newcommand{\I}{\mathbb{I}}
\newcommand{\V}{\mathbb{V}}
\newcommand{\A}{\mathbb{A}}
\newcommand{\Ps}{\mathbb{P}}
\newcommand{\exercise}[1]{\noindent {\bf Exercise #1}}

\DeclareMathOperator{\Hom}{Hom}
\DeclareMathOperator{\Ima}{Im}
\DeclareMathOperator{\coker}{coker}
\DeclareMathOperator{\Id}{Id}
\DeclareMathOperator{\GL}{GL}
\DeclareMathOperator{\Ext}{Ext}
\DeclareMathOperator{\Tor}{Tor}

\begin{document}

\title{Algebraic Topology II - Assignment 1}

\author{Matteo Durante, s2303760, Leiden University}

\maketitle


~\\
\exercise{4}

We want to prove that we have a long exact sequence of the following form:
$$\cdots\rightarrow H^n(X)\xrightarrow{(i^*_U,i^*_V)} H^n(U)\oplus
H^n(V)\xrightarrow{j^*_U-j^*_V} H^n(U\cap V)\xrightarrow{\sigma_{X,V}\delta_{U,U
\cap V}} H^{n+1}(X)\rightarrow\cdots$$

Consider the following commutative diagram, where the two rows are given by the long exact
sequences of the pairs $(X,V)$ and $(U,U\cap V)$, the chain homomorphism is
induced by the inclusions and, considered the closed
subset of $X$ given by $W=X\setminus U\subset V$, since $H^n(X,V)\cong
H^n(X\setminus W,V\setminus W)\cong H^n(U,U\cap V)$ by excision, the map
$H^n(X,V)\rightarrow H^n(U,U\cap V)$ is the identity:

\[
    \begin{tikzcd}
        \cdots\arrow{r}
        & H^n(X,V)\arrow{r}{\sigma_{X,V}}\arrow[equal]{d}
        & H^n(X)\arrow{r}{i^*_V}\arrow{d}{i^*_U}
        & H^n(V)\arrow{r}{\delta_{X,V}}\arrow{d}{j^*_V}
        & H^{n+1}(X,V)\arrow{r}\arrow[equal]{d}
        & \cdots \\
        \cdots\arrow{r}
        & H^n(U,U\cap V)\arrow{r}{\sigma_{U,U\cap V}}
        & H^n(U)\arrow{r}{j^*_U}
        & H^n(U\cap V)\arrow{r}{\delta_{U,U\cap V}}
        & H^{n+1}(U,U\cap V)\arrow{r}
        & \cdots
    \end{tikzcd}
\]

Let $x\in H^n(X)$ be sent to $(0,0)$ in the Mayer-Vietoris sequence. Since the
image under $i^*_V$ is 0, by exactness there is a $x'\in H^n(X,V)=H^n(U,U\cap V)$ s.t.
$\sigma_{X,V}(x')=x$ and therefore $i^*_U\sigma_{X,V}(x')=\sigma_{U,U\cap V}(x')
=0$, hence again by exactness we have a $u'\in H^{n-1}(U\cap V)$ s.t.
$\delta_{U,U\cap V}(u')=x'$, thus $\sigma_{X,V}\delta_{U,U\cap
V}(u')=\sigma_{X,V}(x')=x$. 

On the other hand, by commutativity, $(i^*_U,i^*_V)\sigma_{X,V}\delta_{U,U\cap
V}=(i^*_U\sigma_{X,V}\delta_{U,U\cap V},i^*_V\sigma_{X,V}\delta_{U,U\cap
V})=(\sigma_{U,U\cap V}\delta_{U,U\cap V},0\delta_{U,U\cap V})=(0,0)$. We have proved
the exactness at $H^n(X)$ for every $n$.

Let now $(u,v)\in H^n(U)\oplus H^n(V)$ be mapped to 0. By exactness,
$\delta_{U,U\cap V}j^*_U(u)=0$, thus by commutativity $\delta_{X,V}(v)=\delta_{U,U\cap
V}j^*_V(v)=\delta_{U,U\cap V}j^*_U(u)=0$. It follows that exists $x\in H^n(X)$
s.t. $i^*_V(x)=v$. Let $i^*_U(x)=u'$. We have that $j^*_U(u')=j^*_Ui^*_U(x)=
j^*_Vi^*_V(x)=j^*_V(v)=j^*_U(u)$, hence $j^*_U(u-u')=0$ and by exactness there
is a $u''\in H^n(U,U\cap V)$ s.t. $\sigma_{U,U\cap V}(u'')=u-u'$.

Consider now in $H^n(X)$ the element $x+\sigma_{X,V}(u'')$. We see that
$i^*_U(x+\sigma_{X,V}(u''))=i^*_U(x)+i^*_U\sigma_{X,V}(u'')=u'+\sigma_{U,U\cap
V}(u'')=u'+(u-u')=u$ and
$i^*_V(x+\sigma_{X,V}(u''))=i^*_V(x)+i^*_V\sigma_{X,V}(u'')=v+0=v$, hence
$(u,v)$ lies in the image of $(i^*_U,i^*_V)$.

By commutativity and exactness, $(j^*_U-j^*_V)(i^*_U,i^*_V)=j^*_Ui^*_U-j^*_V
i^*_V=0$, thus we have proved the exactness at $H^n(U)\oplus H^n(V)$.

Let now $u\in H^n(U\cap V)$ be mapped to 0 under $\sigma_{X,V}\delta_{U,U\cap
V}$. This implies that $\delta_{U,U\cap V}(u)$ lies in $\ker(\sigma_{X,V})$,
hence there is an element $v\in H^n(V)$ and a $u'=j^*_V(v)$ s.t.
$\delta_{U,U\cap V}(u')=\delta_{U,U\cap
V}j^*_V(v)=\delta_{X,V}(v)=\delta_{U,U\cap V}(u)$, i.e. $\delta_{U,U\cap
V}(u-u')=0$. By exactness, we have a $u''\in H^n(U)$ s.t. $j^*_U(u'')=u-u'$ and,
considering now $(u'',-v)\in H^n(U)\oplus H^n(V)$, we have that
$(j^*_U-j^*_V)(u'',-v)=j^*_U(u'')-j^*_V(-v)=(u-u')-(-u')=u$.

On the other hand, $\sigma_{X,V}\delta_{U,U\cap
V}(j^*_U-j^*_V)=\sigma_{X,V}\delta_{U,U\cap V}j^*_U-\sigma_{X,V}\delta_{U,U\cap
V}j^*_V=\sigma_{X,V}0-\sigma_{X,V}\delta_{X,V}=0$ by commutativity and
exactness. We have now proved the thesis.


~\\
\exercise{5}

$(a)$ First of all, consider the homomorphism of rings
$(\Z/2\Z)[y]\xrightarrow{f}(\Z/2\Z)[x]/(x^2-1)$ s.t. $y\mapsto x+1$. It is
clearly surjective as $y-1\mapsto x$ and, since $\ker(f)=(y^2)$,
$(\Z/2\Z)[y]/(y^2)\cong (\Z/2\Z)[x]/(x^2-1)$. Considering this isomorphism, we
view $\Z/2\Z$ as a $(\Z/2\Z)[y]/(y^2)$-module, where $y$ acts as $x-1$ and hence
0. From now on we will denote $(\Z/2\Z)[y]/(y^2)$ as $A$ thanks to the
isomorphism.

Consider the following short exact sequence:
$$0\rightarrow\Z/2\Z\xrightarrow{\phi}(\Z/2\Z)[y]/(y^2)\rightarrow
\Z/2\Z\rightarrow 0$$
where the first $A$-module homomorphism sends 1 to y, the second one 1 to 1 and
$y$ to $0$. Applying the $\Ext^n_A(-,\Z/2\Z)$ functor, we get the following long
exact sequence:
\begin{align*}
    0 &\rightarrow \Hom_A(\Z/2\Z,\Z/2\Z)\rightarrow \Hom_A(A,\Z/2\Z)\rightarrow
    \Hom_A(\Z/2\Z,\Z/2\Z)\rightarrow \\
    &\rightarrow \Ext^1_A(\Z/2\Z,\Z/2\Z)\rightarrow \Ext^1_A(A,\Z/2\Z)\rightarrow
    \Ext^1_A(\Z/2\Z,\Z/2\Z)\rightarrow\cdots
\end{align*}

We know that $\Ext^0_A(\Z/2\Z,\Z/2\Z)\cong \Hom_A(\Z/2\Z,\Z/2\Z)\cong\Z/2\Z$ and
$\Hom_A(A,\Z/2\Z)\cong\Z/2\Z$.

Since $A$ is a free $A$-module, $\Ext^n_A(A,\Z/2\Z)=0$ for all $n>0$, thus we
have:
\begin{align*}
    & \Z/2\Z\rightarrow\Z/2\Z\rightarrow \Ext^1_A(\Z/2\Z,\Z/2\Z)\rightarrow 0 \\
    & 0\rightarrow \Ext^n_A(\Z/2\Z,\Z/2\Z)\rightarrow
    \Ext^{n+1}_A(\Z/2\Z,\Z/2\Z)\rightarrow 0\textit{ if }n>1
\end{align*}

From the last exact sequence, it follows that $\Ext^n_A(\Z/2\Z,\Z/2\Z)\cong
\Ext^1_A(\Z/2\Z,\Z/2\Z)$ for every $n>1$, while from the previous one
$\Ext^1_A(\Z/2\Z,\Z/2\Z)\cong\coker(\Z/2\Z\rightarrow\Z/2\Z)$.

Now, an element of $\Hom_A(\Z/2\Z,\Z/2\Z)$ is defined by the 
image of the unit, hence for any element of $\Hom_A(A,\Z/2\Z)$ we only
have to check where the unit of $\Z/2\Z$ is sent by it of $\Hom_A(A,\Z/2\Z)$ 
precomposed with the $\phi$. Remember that the unit of $\Z/2\Z$ is sent to
$y$. We have then that, for any element $f\in A$,
$\phi^*(f)(1)=f(\phi(1))=f(y)=y\cdot f(1)=0$, thus $\phi^*$ is the
zero-homomorphism and $\Ext^1_A(\Z/2\Z,\Z/2\Z)\cong\Z/2\Z$.

$(b)$ First of all, consider the ring epimorphism $R[y]\twoheadrightarrow A$
s.t. $y\mapsto x+1$. Its kernel is given by $(y(y-2))$, hence we get an
isomorphism $R[y]/(y(y-2))\cong A$, which turns $R$ into a
$R[y]/(y(y-2))$-module where $y$ acts as $x+1$, i.e. as 2. From now on, thanks
to this isomorphism, we will call $A$ the ring $R[y]/(y(y-2))$.

Consider the following short exact sequences:
$$0\rightarrow R'\xrightarrow{g} A\xrightarrow{f} R\rightarrow 0$$
$$0\rightarrow R\xrightarrow{g'} A\xrightarrow{f'} R'\rightarrow 0$$

Here, $g$ is given by $r\mapsto r(y-2)$, $f$ by $r\mapsto r$, $y\mapsto 2$ and
$R'$ is the $A$-module whose underlying abelian group is $R$ and s.t. $y$ acts
on it as 0.

On the other hand, $g'$ is given by $r\mapsto ry$, $f'$ by $r\mapsto r$,
$y\mapsto 0$.

It is straightforward to check that all of these are $A$-module homomorphisms and the
chains are short exact sequences.

Let now $\phi:=gf'$, $\psi:=g'f$. By composing these chains, we get the 
following free resolution of $R$:
$$\cdots\rightarrow A\xrightarrow{\phi} A\xrightarrow{\psi} A\xrightarrow{\phi}
A\xrightarrow{f} R\rightarrow 0$$

Now we apply the functor $\Hom_A(-,R)$:
$$0\rightarrow\Hom_A(R,R)\xrightarrow{f^*}\Hom_A(A,R)\xrightarrow{\phi^*}
\Hom_A(A,R)\xrightarrow{\psi^*}\Hom_A(A,R)\xrightarrow{\phi^*}
\Hom_A(A,R)\rightarrow\cdots$$

Thanks to the isomorphism $\Hom_A(A,R)\cong R$, $f\mapsto f(1)$, we will identify
each element of this group with the element of $R$ the unit is mapped to.

Furthermore, we see that $\Ext^n_A(R,R)\cong\ker(\psi^*)/\Ima(\phi^*)$ if $n$ is
odd and $\Ext^n_A(R,R)\cong\ker(\phi^*)/\Ima(\psi^*)$ if $n$ is even and $>0$. 
Clearly, $\Ext^0_A(R,R)\cong\ker(\phi^*)$.

Let $h\in\Hom_A(A,R)$ and notice that $\phi^*(h)(1)=h(\phi(1))=h(g(f'(1)))=
h(g(1))=h(y-2)=(y-2)\cdot h(1)=0$. It follows that $\phi^*$ is a zero-homomorphism
and $\ker(\phi^*)\cong R$, hence $\Ext^0_A(R,R)\cong R$.

Let now $h\in Hom_A(A,R)$. We have that $\psi^*(h)(1)=h(\psi(1))=h(g'(f(1))=
h(g'(1))=h(y)=y\cdot h(1)=2 h(1)$ and, since $h(1)$ can be mapped anywhere,
$\Ima(\psi^*)\cong 2R$, thus $\Ext^n(R,R)\cong R/2R$ for $n$ even and $>0$.

By the same reasoning, the elements of $\ker(\psi^*)$ are those s.t. $2h(1)=0$,
i.e. $\ker(\psi^*)\cong\Tor_2(R)$ and therefore $\Ext^n_A(R,R)\cong\Tor_2(R)$ for
$n$ odd.


\printbibliography

\end{document}
