\documentclass{article}
\usepackage[T1]{fontenc}
\usepackage{lmodern}
\usepackage[utf8]{inputenc}
\usepackage[british]{babel}
\usepackage{geometry}
\usepackage{color}
\usepackage{amsthm}
\usepackage{amsmath,amssymb}
\usepackage{graphicx}
\usepackage{mathtools}
\usepackage{listings}
\usepackage{newlfont}
\usepackage{tikz-cd}
\usepackage{rotating}
\usepackage[backend=biber]{biblatex}
\addbibresource{~/math/references.bib}

\newcommand{\numberset}{\mathbb}
\newcommand{\N}{\numberset{N}}
\newcommand{\Z}{\numberset{Z}}
\newcommand{\R}{\numberset{R}}
\newcommand{\Q}{\numberset{Q}}
\newcommand{\K}{\numberset{K}}
\newcommand{\F}{\numberset{F}}
\newcommand{\n}{\mathcal{N}}
\newcommand{\aid}{\mathfrak{a}}
\newcommand{\bid}{\mathfrak{b}}
\newcommand{\pid}{\mathfrak{p}}
\newcommand{\qid}{\mathfrak{q}}
\newcommand{\mi}{\mathfrak{m}}
\newcommand{\I}{\mathbb{I}}
\newcommand{\V}{\mathbb{V}}
\newcommand{\A}{\mathbb{A}}
\newcommand{\Ps}{\mathbb{P}}
\newcommand{\exercise}[1]{\noindent {\bf Exercise #1}}

\DeclareMathOperator{\im}{im}
\DeclareMathOperator{\coker}{coker}
\DeclareMathOperator{\Id}{Id}
\DeclareMathOperator{\GL}{GL}
\DeclareMathOperator{\Mat}{Mat}
\DeclareMathOperator{\Ext}{Ext}
\DeclareMathOperator{\Tor}{Tor}
\DeclareMathOperator{\Hom}{Hom}


\begin{document}

\title{Algebraic Topology II - Assignment 4}

\author{Matteo Durante, s2303760, Leiden University}

\maketitle


~\\
\exercise{3}

\begin{proof}
    Our strategy will be to construct the space $K(\Z,n)$ from $S^n$ by glueing
    disks of dimension $>n+1$.
    
    Assuming its construction, we will first prove that $H^n(X)\cong[X,S^n]$.
    
    By definition we have that, for $n>0$,
    $H^n(-)\cong \tilde{H}^n(-)\cong [-,K(\Z,n)]$, thus $H^n(X)\cong[X,K(\Z,n)]$
    and, by the cellular
    approximation theorem, any class of maps in $[X,K(\Z,n)]$ is represented by
    a cellular map. Since by assumption $X$ is a CW-complex of dimension $n$, we
    have that the image of this map is contained in $S^n\subset K(\Z,n)$,
    therefore it factors uniquely through $S^n$. This gives us a map
    $[X,K(\Z,n)]\rightarrow[X,S^n]$.
    
    $(*)$ This association is well defined, for if two maps (which we may assume
    cellular) $X\xrightarrow{f,g}K(\Z,n)$ are homotopic we have a homotopy $X\times I
    \xrightarrow{H'} K(\Z,n)$ among them. Since $X\times I$ is a CW-complex of
    dimension $n+1$ and there are no
    $(n+1)$-cells in $K(\Z,n)$, being $f,g$ cellular maps, it corresponds to a
    cellular homotopy $H$ between $f,g$ whose image is again in $S^n\subset
    K(\Z,n)$. By factorizing $H$ through $S^n$, it follows that this homotopy
    induces a homotopy between $f$ and $g$ seen as maps $X\rightarrow S^n$.

    Viceversa, any equivalence class of $[X,S^n]$ induces naturally a
    class of maps $X\rightarrow K(\Z,n)$ thanks to the composition with the
    natural inclusion $S^n\xhookrightarrow{i} K(\Z,n)$. We will now check that
    even this association is well defined.
    
    Let $f,g$ be homotopic maps $X\rightarrow S^n$. If there is a homotopy
    $X\times I\xrightarrow{H}S^n$ among them, we
    may naturally turn it into a homotopy between $i\circ f$ and $i\circ g$ by
    considering $i\circ H$, hence we are done.

    The association is injective, for if two maps $f,g$ are extended to
    homotopic maps $i\circ f,i\circ g$, then we may apply the same reasoning as
    before $(*)$ to deduce that $f$ and $g$ are homotopic as well.

    In the same way, if we have two (cellular) maps $X\xrightarrow{f,g} K(\Z,n)$
    inducing homotopic maps $X\rightarrow S^n$, then we may extend the homotopy
    to a map $X\times I\rightarrow K(\Z,n)$ through the inclusion and get
    another between $f$ and $g$.

    We see that the two associations are naturally inverse to each other, hence
    we have a bijection and it follows that $H^n(X)\cong[X,K(\Z,n)]\cong[X,S^n]$
    for every $CW$-complex of dimension $n$.

    We will now construct the aforementioned Eilenberg-MacLane space.

    First of all, observe that we can choose $M(\Z,n)=S^n$. Indeed, $\pi_kS^n=0$
    for $k<n$ by the cellular approximation theorem, which tells us that maps
    $S^k\rightarrow S^n$ are homotopic to the constant map because $S^n$ can be
    constructed using only a 0-cell and a $n$-cell. Furthermore, $\pi_nS^n=\Z$
    by~\cite[cor. 15.7]{Sag17} and the well-known result about $n=1$. Also, this
    fact is stated in~\cite[ex. 8.8]{HM19}.

    By the proof of~\cite[thm. 8.9]{HM19}, $K(\Z,n)^{st}=P_n^{st}(S^n)$ is a
    space with the desired properties. Notice that in its construction,
    given in~\cite[lemmaa 8.4]{HM19}, no $(n+1)$-cells are attached to $S^n$,
    hence we are done.
\end{proof}


\printbibliography

\end{document}


