\documentclass{article}
\usepackage[T1]{fontenc}
\usepackage{lmodern}
\usepackage[utf8]{inputenc}
\usepackage[british]{babel}
\usepackage{geometry}
\usepackage{color}
\usepackage{amsthm}
\usepackage{amsmath,amssymb}
\usepackage{graphicx}
\usepackage{mathtools}
\usepackage{listings}
\usepackage{newlfont}
\usepackage{tikz-cd}
\usepackage{rotating}
\usepackage[backend=biber]{biblatex}
\addbibresource{~/math/references.bib}

\newcommand{\numberset}{\mathbb}
\newcommand{\N}{\numberset{N}}
\newcommand{\Z}{\numberset{Z}}
\newcommand{\R}{\numberset{R}}
\newcommand{\Q}{\numberset{Q}}
\newcommand{\K}{\numberset{K}}
\newcommand{\F}{\numberset{F}}
\newcommand{\n}{\mathcal{N}}
\newcommand{\aid}{\mathfrak{a}}
\newcommand{\bid}{\mathfrak{b}}
\newcommand{\pid}{\mathfrak{p}}
\newcommand{\qid}{\mathfrak{q}}
\newcommand{\mi}{\mathfrak{m}}
\newcommand{\I}{\mathbb{I}}
\newcommand{\V}{\mathbb{V}}
\newcommand{\A}{\mathbb{A}}
\newcommand{\Ps}{\mathbb{P}}
\newcommand{\exercise}[1]{\noindent {\bf Exercise #1}}

\DeclareMathOperator{\im}{im}
\DeclareMathOperator{\coker}{coker}
\DeclareMathOperator{\Id}{Id}
\DeclareMathOperator{\GL}{GL}
\DeclareMathOperator{\Mat}{Mat}
\DeclareMathOperator{\Ext}{Ext}
\DeclareMathOperator{\Tor}{Tor}
\DeclareMathOperator{\Hom}{Hom}


\begin{document}

\title{Algebraic Topology II - Assignment 4}

\author{Matteo Durante, s2303760, Leiden University}

\maketitle


~\\
\exercise{3}

\begin{proof}
    Our strategy will be to construct the space $K(\Z,n)$ from $S^n$ by glueing
    disks of dimension $>n+1$.
    
    Assuming its construction, we will first prove that $H^n(X)\cong[X,S^n]$.
    
    By definition we have that, for $n>0$,
    $H^n(-)\cong \tilde{H}^n(-)\cong [-,K(\Z,n)]^\bullet\cong [-,K(\N,n)]$, thus
    $H^n(X)\cong[X,K(\Z,n)]$ and, by the cellular
    approximation theorem, any class of maps in $[X,K(\Z,n)]$ is represented by
    a cellular map. Since by assumption $X$ is a CW-complex of dimension $n$, we
    have that the image of this map is contained in $S^n\subset K(\Z,n)$,
    therefore it factors through $S^n$. This gives us a map
    $[X,K(\Z,n)]\rightarrow[X,S^n]$.
    
    $(*)$ This association is well defined, for if two cellular maps
    $X\xrightarrow{f,g}K(\Z,n)$ are homotopic we have a homotopy $X\times I
    \xrightarrow{H'} K(\Z,n)$ among them. Since $X\times I$ is a CW-complex of
    dimension $n+1$ and there are no
    $(n+1)$-cells in $K(\Z,n)$, being $f,g$ cellular maps, it corresponds to a
    cellular homotopy $H$ between $f,g$ whose image is again in $S^n\subset
    K(\Z,n)$. By factorizing $H$ through $S^n$, it follows that this homotopy
    induces a homotopy between $f$ and $g$ seen as maps $X\rightarrow S^n$.

    Viceversa, any equivalence class of $[X,S^n]$ induces naturally a
    class of maps $X\rightarrow K(\Z,n)$ thanks to the composition with the
    natural inclusion $S^n\xhookrightarrow{i} K(\Z,n)$. We will now check that
    even this association is well defined.
    
    Let $f,g$ be homotopic maps $X\rightarrow S^n$. If there is a homotopy
    $X\times I\xrightarrow{H}S^n$ among them, we
    may naturally turn it into a homotopy between $i\circ f$ and $i\circ g$ by
    considering $i\circ H$, hence we are done.

    The association is injective, for if two maps $f,g$ are extended to
    homotopic maps $i\circ f,i\circ g$, then we may apply the same reasoning as
    before $(*)$ to deduce that $f$ and $g$ are homotopic as well.

    In the same way, if we have two (cellular) maps $X\xrightarrow{f,g} K(\Z,n)$
    inducing homotopic maps $X\rightarrow S^n$, then we may extend the homotopy
    to a map $X\times I\rightarrow K(\Z,n)$ through the inclusion and get
    another between $f$ and $g$.

    We see that the two associations are naturally inverse to each other, hence
    we have a bijection and it follows that $H^n(X)\cong[X,K(\Z,n)]\cong[X,S^n]$
    for every $CW$-complex of dimension $n$.

    We will now construct the aforementioned Eilenberg-MacLane space.

    First of all, observe that we can choose $M(\Z,n)=S^n$. Indeed, $\pi_kS^n=0$
    for $k<n$ by the cellular approximation theorem, which tells us that maps
    $S^k\rightarrow S^n$ are homotopic to the constant map because $S^n$ can be
    constructed using only a 0-cell and a $n$-cell. Furthermore, $\pi_nS^n=\Z$
    by~\cite[cor. 15.7]{Sag17} and the well-known result about $n=1$. Also, this
    fact is stated in~\cite[ex. 8.8]{HM19}.

    By the proof of~\cite[thm. 8.9]{HM19}, $K(\Z,n)^{st}=P_n^{st}(S^n)$ is an
    Eilenberg-MacLane space for $\tilde{H}^n(-)$. Notice that in its
    construction, given in~\cite[lemmaa 8.4]{HM19}, no $(n+1)$-cells are
    attached to $S^n$, hence we are done.
\end{proof}


~\\
\exercise{4}

\begin{proof}
    $(a)$ We will first show how a map $X\rightarrow F_{p(e_1)}$ induces, with
    the path mentioned, a map $X\rightarrow F_{p(e_2)}$, which we will show to
    be unique up to homotopy.

    Let $X$ be a CW-complex, $s$ the path from $p(e_1)$ to $p(e_2)$ induced
    by the $\gamma$, $X\xrightarrow{f} F_{p(e_1)}$ continuous. Let's look at the
    following commutative diagram, where $\tilde{\phi}$ is given by the
    composition of $f$ with the inclusion $F_{p(e_1)}\hookrightarrow E$, 
    $\Phi(x,t)=s(t)$:
    \[
        \begin{tikzcd}
            X\arrow{r}{\tilde{\phi}}\arrow{d}
            & E\arrow{d}{p} \\
            X\times I\arrow[dotted]{ur}[description]{\tilde{\Phi}}\arrow{r}{\Phi}
            & B
        \end{tikzcd}
    \]

    Since $p$ is a Serre fibration, by~\cite[p. 107, p.110]{FF16}, it induces a
    map $\tilde{\Phi}$ s.t. $p\tilde{\Phi}=\Phi$ and
    $\tilde{\Phi}|_{X\times\{0\}}=\tilde{\phi}$ (which we may pick s.t., for a
    base point $x_0$ of $X$, $\tilde{\Phi}(x_0,t)=\gamma(t)$). We consider now
    the map $h=\tilde{\Phi}|_{X\times\{1\}}$. By construction,
    $ph(X)=p\tilde{\Phi}(X,1)=\Phi(X,1)=s(1)=e_2$ and therefore $h(X)\subset
    p^{-1}(e_2)=F_{p(e_2)}$, hence we may define a map $g$ by restricting the
    codomain of $h$ to $F_{p(e_2)}$. Also, $g(x_0)=e_2$.

    We now show that the homotopy class of $g$ does not depend on the lifting
    $\tilde{\Phi}$ considered or on the choice of the path, as long as the
    latter belongs to the same homotopy class.

    Indeed, let $I\xrightarrow{s'}B$ be defined as $p\gamma'$, where $\gamma'$
    is a path homotopic to $\gamma$ and going from $e_1$ to $e_2$. Let $\Phi',
    \tilde{\Phi}',h'$ and $g'$ be defined from $f$ as their counterparts, this
    time using $s'$.

    Using the homotopy between $s$ and $s'$, we define a map $[-1,1]\times
    I\xrightarrow{S}B$ which is a homotopy between $s*s'^{-1}$ and the constant
    path at $p(e_2)$. Then, we define $(X\times[-1,1])\times I\xrightarrow{\psi}
    B$ as $\psi(x,u,t)=S(u,t)$ and a map
    $X\times[-1,1]\xrightarrow{\tilde{\psi}}E$ as
    $\tilde{\psi}(x,u)=\tilde{\Phi}(x,-u)$ if $u\leq0$, $=\tilde{\Phi}'(x,u)$
    otherwise. Applying the homotopy lifting property of the Serre fibration as
    before, we get a homotopy $(X\times[-1,1])\times
    I\xrightarrow{\tilde{\Psi}}E$, which restricted to $X\times(\{-1\}\times
    I\cup [-1,1]\times\{0\}\cup\{1\}\times I)$ induces a homotopy between $g$
    and $g'$.

    Now, setting $X=S^n$, we get that a map $S^n\xrightarrow{f}F_{p(e_1)}$
    defines a map $S^n\xrightarrow{g}F_{p(e_2)}$ which is unique up to homotopy
    and depends only on the homotopy class of $\gamma$, hence we have an
    association $\pi_n(F_{p(e_1)},e_1)\xrightarrow{\alpha_{\gamma}}
    \pi_n(F_{p(e_2)},e_2)$.

    We want to prove that, given two paths $I\xrightarrow{\gamma,\gamma'}E$,
    $\alpha_{\gamma*\gamma'}=\alpha_{\gamma'}\circ\alpha_{\gamma}$ when
    $\gamma(1)=\gamma'(0)$.

    Let $S^n\xrightarrow{f}F_{p(e_2)},\Phi,\tilde{\Phi}$ be the maps
    constructed from $S^n\xrightarrow{r}F_{p(e_1)}$ using $\gamma$,
    $S^n\xrightarrow{f'}F_{p(e_3)},\Phi',\tilde{\Phi'}$ the ones constructed
    from $f$ using $\gamma'$ and $S^n\xrightarrow{f''}F_{p(e_3)},\Phi'',
    \tilde{\Phi}''$ the ones created from $f$ using $\gamma*\gamma'$.

    Observing that $\tilde{\Phi}'(x,0)=f(x)=\tilde{\Phi}(x,1)$, we can choose
    choose $\tilde{\Phi}''$ s.t. $\tilde{\Phi}''(x,t)=\tilde{\Phi}(x,2t)$ for
    $t\geq1/2$, $=\tilde{\Phi}(x,2t-1)$ for $t>1/2$ and the diagram will commute
    by construction. The thesis follows as
    $f''(x)=\tilde{\Phi}''(x,1)=\tilde{\Phi}'(x,1)=f'(x)$.
\end{proof}

\begin{proof}
    $(b)$ We want to show that, given a path $I\xrightarrow{\gamma}E$ from $e_0\in
    p^{-1}(b_0)$ to $e_1\in p^{-1}(b_1)$, $\alpha_{\gamma}$ defines a group
    homomorphism $\pi_n(F_{p(e_1)},e_1)\rightarrow\pi_n(F_{p(e_2)},e_2)$ with
    inverse $\alpha_{\gamma^{-1}}$.

    Let $f,f'$ be maps $S^n\rightarrow F_{p(e_0)}$ with $f(x_0)=g(x_0)=e_0$.
    Under $\alpha_{\gamma}$, $[f]$ and $[f']$ are sent to the homotopy classes
    of $g(x)=\tilde{\Phi}(x,1)$ and $g'(x)=\tilde{\Phi}(x,1)$. We can construct
    from $\tilde{\Phi},\tilde{\Phi}'$ the map $\tilde{\Phi}''$ defining the
    image of $[f*g]$ by setting
    $\tilde{\Phi}''(-,t)=\tilde{\Phi}(-,t)*\tilde{\Phi}'(-,t)$ for every $t$.
    We can do this because, for every $t\in I$,
    $\tilde{\Phi}(x_0,t)=\gamma(t)=\tilde{\Phi}'(x_0,t)$, hence they both define
    elements of $\pi_n(E,\gamma(t))$. Also, the resulting map $\tilde{\Phi}''$
    is continuous.

    The fact that this $\tilde{\Phi}''$ is an adequate lifting comes from the
    fact that
    $p\tilde{\Phi}(x,t)=\Phi(x,t)=\gamma(t),\ p\tilde{\Phi}'(x,t)=\Phi'(x,t)=
    \gamma(t)$ and therefore $p\tilde{\Phi}''(x,t)=\gamma(t)=\Phi''(x,t)$ with
    $\tilde{\Phi}''(-,0)=\tilde{\Phi}(-,0)*\tilde{\Phi}'(-,0)=f*g$.

    Finally, by definition $g*g'$ is given by $\tilde{\Phi}(-,1)*\tilde{\Phi}'
    (-,1)$, which is precisely $=\tilde{\Phi}''(-,1)$, that is the map $f*f'$
    is sent to up to homotopy.

    Now we are going to prove that $\alpha_{\gamma^{-1}}$ is inverse to
    $\alpha_{\gamma}$. To do this, it will be enough to check that
    $\alpha_{\gamma^{-1}}\circ\alpha_{\gamma}=\Id_{\pi_n(F_{b_0},e_0)}$ by
    simmetry.

    By what we proved in $(a)$,
    $\alpha_{\gamma^{-1}}\circ\alpha_{\gamma}=\alpha_{\gamma*\gamma^{-1}}$ and
    homotopic paths define the same map, thus, since
    $\gamma*\gamma^{-1}$ is homotopic to $const_{e_0}$,
    $\alpha_{\gamma^{-1}}\circ\alpha_{\gamma}=\alpha_{const_{e_0}}$.

    Now, noticing that under the latter map from an element
    $[f]\in\pi_n(F_{b_0},e_0)$ we can define $\tilde{\Phi}$ simply as
    $\tilde{\Phi}(-,t)=f$ and therefore associate $f$ to
    $\tilde{\Phi}(-,1)=f$, we get that
    $\alpha_{const_{e_0}}=\Id_{\pi_n(F_{b_0},e_0)}$ and we are done.
\end{proof}

\begin{proof}
    $(c)$ Remember that $\Omega B$ with the usual maps is an $H$-space
    and defines a group structure which is naturally $\cong\pi_1(B,b_0)$,
    hence for any $\gamma\in\Omega B$ the map $\Omega
    B\xrightarrow{\gamma_\#}\Omega B$, $\alpha\mapsto
    \gamma*\alpha*\gamma^{-1}$ is continuous and compatible with homotopy
    relations. Since homotopy relations are preserved by compositions among
    continuous maps, we may define the action of $[\alpha]\in\pi_1(B,b_0)$ on
    $\pi_n(\Omega B,const_{b_0})$ by setting, for
    $[f]\in\pi_n(\Omega B,const_{b_0})$, $[\alpha]\cdot [f]=[\alpha_\#\circ f]$.
    We will now check that this is an action as claimed.
    
    First of all, it is well defined because all of the operations considered
    preserve homotopy relations.

    Let now $[\beta]\in\pi_1(B,b_0)$. We see that
    $(\alpha*\beta)_\#(f(x))=(\alpha*\beta)*f(x)*(\alpha*\beta)^{-1}=
\alpha*(\beta*f(x)*\beta^{-1})*\alpha^{-1}=\alpha_\#(\beta_\#(f(x)))$, hence
    $([\alpha]*[\beta])\cdot [f]=[\alpha]\cdot ([\beta]\cdot [f])$.

    In particular, $(const_{b_0})_\#(f(x))=const_{b_0}*f(x)*const_{b_0}^{-1}$.
    Since, making use of the axioms of a $H$-space,
    $const_{b_0}*f(-)*const_{b_0}^{-1}\cong f$ (similarly to the previous
    assignment), it follows that $[const_{b_0}]\cdot [f]=[f]$, which confirms
    that this is a group action as desired.

    Using the fact that $\pi_{n-1}(\Omega B,const_{b_0})\cong\pi_n(B,b_0)$,
    this induces an action of $\pi_1(B,b_0)$ on $\pi_n(B,b_0)$.

    We will now consider the case where $n=1$.
    
    Since the elements of $\pi_0(\Omega B,const_{b_0})$ are homotopy classes
    of pointed maps $S^0\xrightarrow{f}\Omega B$, we have that
    $f(*)=const_{b_0},\ f(*_2)=\alpha\in\Omega
    B$ ($*_2$ is the point in $S^0$ which is not fixed). The
    classes correspond to the path components of $\Omega B$ and homotopies to
    homotopies between based loops in $B$.

    In this case, given $[\alpha]\in\pi_1(B,b_0)$, we see that
    $[\alpha]\cdot[f]=[\alpha_\#\circ f]$ and in particular $(\alpha_\#\circ
    f)(*_2)=\alpha*f(*_2)*\alpha^{-1}$.

    The canonical identification $\pi_0(\Omega
    B,const_{b_0})\cong\pi_1(B,b_0)$ is s.t. $[f]\mapsto [f(*_2)]$, hence the
    induced action of $\pi_1(B,b_0)$ on itself is s.t. for
    any $[\beta]\in\pi_1(B,b_0)$ we have
    $[\alpha]\cdot[\beta]=[\alpha*\beta*\alpha^{-1}]$ because, considered the
    map $S^0\xrightarrow{g}\Omega B$ s.t. $g(*_2)=\beta$, $[\alpha]\cdot[\beta]$
    is given by the element corresponding to $[\alpha]\cdot [g]$, that is
    $[(\alpha_\#\circ g)(*_2)]=[\alpha*\beta*\alpha^{-1}]$. This is
    the conjugation action and defines for every loop $[\alpha]$ an automorphism
    of $\pi_1(B,b_0)$.
\end{proof}

\printbibliography

\end{document}
