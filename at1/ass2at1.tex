\documentclass{article}
\usepackage[T1]{fontenc}
\usepackage{lmodern}
\usepackage[utf8]{inputenc}
\usepackage[british]{babel}
\usepackage{geometry}
\usepackage{color}
\usepackage{amsthm}
\usepackage{amsmath,amssymb}
\usepackage{graphicx}
\usepackage{mathtools}
\usepackage{listings}
\usepackage{newlfont}
\usepackage{tikz-cd}
\usepackage{faktor}

\newcommand{\numberset}{\mathbb}
\newcommand{\N}{\numberset{N}}
\newcommand{\Z}{\numberset{Z}}
\newcommand{\R}{\numberset{R}}
\newcommand{\Q}{\numberset{Q}}
\newcommand{\C}{\numberset{C}}
\newcommand{\K}{\numberset{K}}
\newcommand{\F}{\numberset{F}}
\newcommand{\n}{\mathcal{N}}
\newcommand{\aid}{\mathfrak{a}}
\newcommand{\bid}{\mathfrak{b}}
\newcommand{\pid}{\mathfrak{p}}
\newcommand{\qid}{\mathfrak{q}}
\newcommand{\mi}{\mathfrak{m}}
\newcommand{\I}{\mathbb{I}}
\newcommand{\V}{\mathbb{V}}

\DeclareMathOperator{\Ima}{Im}

\newcommand{\exercise}[1]{\noindent {\bf Exercise #1}}

\begin{document}

\title{Algebraic Topology 1 - Assignment 2}

\author{M. Durante, 2303760, Leiden University\\T.A.H.A Quemener, 2304252, Leiden University}

\maketitle


\exercise{1}

Let $X'=\{N,Z,W\}$ and consider the following long exact sequence:

\begin{align*}
		\ldots & \rightarrow H_2(X',\C)\rightarrow H_2(X,\C)\rightarrow H_2(X,X',\C)\rightarrow \\
		& \rightarrow H_1(X',\C)\rightarrow H_1(X,\C)\rightarrow H_1(X,X',\C)\rightarrow \\
		& \rightarrow H_0(X',\C)\rightarrow H_0(X,\C)\rightarrow H_0(X,X',\C)\rightarrow 0
\end{align*}
		
Now I show that $X'$ is homotopically equivalent to $*$ by exhibiting a retract:

\begin{align*}
		H: X'\times I & \rightarrow X' \\
		(x,t) & \mapsto \begin{cases}
								x \textit{ if } t\neq 1 \\
								W \textit{ otherwise}
						\end{cases}
\end{align*}

$H(-,0)=id_{X'}$ and $H(-,1)$ is the constant map sending $X'$ to $W$. Furthermore, $H$ is continuous, as we may check by taking the preimages of the non-trivial open sets:

\begin{align*}
		H^{-1}(\{N\})=\{N\}\times[0,1) \\
		H^{-1}(\{Z\})=\{Z\}\times[0,1)
\end{align*}

These are products of open sets, hence open sets in the product space.

From this homotopy equivalence, we get that:

$$
H_n(X',\C)\cong H_n(*,\C)=\begin{cases}
									0 \textit{ if } n>0 \\
									\C \textit{ otherwise}
							\end{cases}
$$

Plugging these groups in the previously mentioned exact sequence, we get:

\begin{align*}
		\ldots & \rightarrow 0\rightarrow H_2(X,\C)\rightarrow H_2(X,X',\C)\rightarrow \\
		& \rightarrow 0\rightarrow H_1(X,\C)\rightarrow H_1(X,X',\C)\rightarrow \\
		& \rightarrow \C\rightarrow H_0(X,\C)\rightarrow H_0(X,X',\C)\rightarrow 0
\end{align*}

By exactness, $H_2(X,\C)\cong H_2(X,X',\C)$.

Now, let $Y=\{W\}\subset X'\subset X$. By construction, in $X$ (and hence in $X'$), $\overline{\{W\}}=\{W\}$ and $X'$ is an open set in $X$, thus, by excision, we get that $H_2(X,X',\C)\cong H_2(X\setminus Y,X'\setminus Y,\C)$.

Let's look at the long exact sequence corresponding to $X\setminus Y$.

$$\ldots\rightarrow H_2(X'\setminus Y,\C)\rightarrow H_2(X\setminus Y,\C)\rightarrow H_2(X\setminus Y,X'\setminus Y,\C)\rightarrow H_1(X'\setminus Y,\C)\rightarrow\ldots$$

Observing that $X'\setminus Y=\{N,Z\}$ constitutes a discrete subspace and hence each point defines a path component, we get the following:

$$
H_n(X'\setminus Y,\C)\cong H_n(\{N\},\C)\oplus H_n(\{Z\},\C)=\begin{cases}
		\C^2\textit{ if } n=0 \\
		0\textit{ otherwise}
\end{cases}
$$

From this, we get that $H_2(X'\setminus Y,\C)=H_1(X'\setminus Y,\C)=0$, therefore, by exactness, $H_2(X\setminus Y,\C)\cong H_2(X\setminus Y,X'\setminus Y,\C)$.

Just like before, we show that $X\setminus Y$ is retractible by exhibiting a retract:

\begin{align*}
		H': X\setminus Y\times I & \rightarrow X\setminus Y \\
				(x,t) & \mapsto \begin{cases}
										x \textit{ if } t\neq 1 \\
										O \textit{ otherwise}
								\end{cases}
\end{align*}

In a similar fashion as before, it can be shown that it is continuous and that it is indeed a retract of $X\setminus Y$ to a point, thus $H_2(X\setminus Y,\C)\cong H_2(*,\C)=0$. From this, we get that $H_2(X,\C)\cong H_2(X,X',\C)\cong H_2(X\setminus Y,X'\setminus Y,\C)=0$.

Going back to our previous exact sequence, we get that:

$$0\rightarrow H_1(X,\C)\rightarrow H_1(X,X',\C)\rightarrow\C\rightarrow H_0(X,\C)\rightarrow 0$$

By defintion we have $H_{0}(X,\C)=\faktor{\mathbb{C}\left[S(X)_0\right]}{\Ima(\partial_{1})}$, Where $\mathbb{C}\left[S(X)_0\right]$ is the 4-dimensional $\mathbb{C}$-vector space generated by the functions $\{e_{N},e_{Z},e_{W},e_{O}\}$ which are non-zero only on their respective constant map from $\Delta^{0}$ to a point of $X$.\\
Now, we need to study $\Ima(\partial_{1})$. To do so, we may describe $S(X)_{1}$, but we only need to look at the the ``extremities'' of the functions that lie in $S(X)_{1}$ because these define the image through $\partial_1$.\\

Consider the following maps:

\begin{align*}
		f_{NW}:\Delta^1\rightarrow X,\ (1-t)e_0+te_1\mapsto\begin{cases}
			N\textit{ if } t<1/2 \\
			W\textit{ if } t\ge 1/2
		\end{cases} \\
		f_{NO}:\Delta^1\rightarrow X,\ (1-t)e_0+te_1\mapsto\begin{cases}
			N\textit{ if } t<1/2 \\
			O\textit{ if } t\ge 1/2
		\end{cases} \\
		f_{NZ}:\Delta^1\rightarrow X,\ (1-t)e_0+te_1\mapsto\begin{cases}
			N\textit{ if } t<1/2 \\
			W\textit{ if } t=1/2 \\
			Z\textit{ if } t>1/2
		\end{cases}
\end{align*}

In a similar fashion, we may define other simpleces, but as we will see these are enough.
		
For any map $g:\Delta^1\rightarrow X$, $\partial(1g)=d_0(1g)-d_1(1g)=e_{g(e_1)}-e_{g(e_0)}$, hence their images through $\partial_1$ are $\partial_1(1f_{NW})=e_N-e_W,\ \partial_1(1f_{NO})=e_N-e_O$ and $\partial_1(1f_{NZ})=e_N-e_Z$. The maps which are non-zero only on closed loops map to 0 because $e_{g(e_1)}-e_{g(e_0)}=0$.
		
Notice that, for any map $g:\Delta^1\rightarrow X$, considered $P,Q\neq N$, if $g(e_1)=P,\ g(e_0)=Q$, then $\partial(1g)=e_P-e_Q=(e_P-e_N)+(e_N-e_Q)=\partial(f_{NQ}-f_{NP})$, hence $\{e_N-e_W,e_N-e_O,e_N-e_Z\}$ generates $\Ima(\partial_1)$ and the generators are linearly independent, therefore it has dimension 3 as a $\C$-vector space. It follows that $H_0(X,\C)$ has dimension 1 as a $\C$-vector space, thus $H_0(X,\C)\cong\C$.

		Now we focus on $H_1(X,X',\C)$.

		Making use of the excision, we see that $H_1(X,X',\C)\cong H_1(X\setminus Y,X'\setminus Y,\C)$, hence we hope to gather information about that relative homology group by studying the following long exact sequence in which it is involved (we are only showing the two first levels).
\begin{align*}
		\ldots & \rightarrow H_{1}(X'\setminus Y,\mathbb{C}) \rightarrow H_{1}(X\setminus Y,\mathbb{C}) \rightarrow H_{1}(X\setminus Y,X'\setminus Y,\mathbb{C}) \rightarrow \\
		& \rightarrow H_{0}(X'\setminus Y,\mathbb{C}) \rightarrow H_{0}(X\setminus Y,\mathbb{C}) \rightarrow H_{0}(X\setminus Y,X'\setminus Y,\mathbb{C}) \rightarrow 0
\end{align*}

As shown before, we have that $H_{1}(X'\setminus Y,\mathbb{C})= 0$ and $H_{0}(X'\setminus Y,\mathbb{C})\cong\mathbb{C}^{2}$. \\
Moreover we can show that $X\setminus Y$ is homotopically equivalent to $*$ by exhibiting a retract : \\
\begin{align*}
		H'': &X\setminus Y\times I\rightarrow X\setminus Y \\
		&(x,t)\mapsto\begin{cases}
				x & \textit{ if } t \neq 1 \\
				O & \textit{ otherwise}
		\end{cases}
\end{align*}
\\

Hence, $H_{1}(X\setminus Y,\mathbb{C})=0$ and $H_{0}(X\setminus Y,\mathbb{C})\cong \mathbb{C}$.

We can look at the following exact sequence:
$$0\rightarrow H_1(X\setminus Y,X'\setminus Y,\C)\xrightarrow{\phi}\C^2\xrightarrow{\psi}\C\rightarrow H_0(X\setminus Y,X'\setminus Y,\C)\rightarrow 0$$

Since $\phi$ is injective, $H_1(X\setminus Y,X'\setminus Y,\C)\cong\Ima(\phi)=\ker(\psi)$, hence we only have to study $\psi:H_0(X'\setminus Y,\C)\rightarrow H_0(X\setminus Y,\C)$.

Remember that $H_0(X'\setminus Y,\C)=\faktor{\C[S(X'\setminus Y)_0]}{\Ima(\partial_1)}$ and, since $X'\setminus Y=\{N,Z\}$ is a subspace with the discrete topology, we have that $S(X'\setminus Y)_0=\{\sigma_N,\sigma_Z\}$, where $\sigma_P$ maps $\Delta^0$ to $P$, hence $\C[S(X'\setminus Y)_0]=\{n\sigma_N+z\sigma_Z\ |\ n,z\in\C\}$. Furthermore, $\C[S(X'\setminus Y)_1]=\{n\sigma'_N+z\sigma'_Z\ |\ n,z\in\C\}$, where $\sigma'_P$ maps $\Delta^1$ to $P$ (these two are the only possible maps because of the discrete topology). Both groups have dimension 2 as $\C$-vector space.

Let's compute their images through $\partial_1$.
\begin{align*}
		\partial_1(n\sigma'_N+z\sigma'_Z) & = d_0(n\sigma'_N+z\sigma'_Z)-d_1(n\sigma'_N+z\sigma'_Z) \\
		& = (n\sigma_N+z\sigma_Z)-(n\sigma_N+z\sigma_Z) \\
		& = 0
\end{align*}

It follows that $\Ima(\partial_1)=0$, hence $H_0(X'\setminus Y,\C)=\C[S(X'\setminus Y)_0]$ and it has dimension 2 as a $\C$-vector space.

Given that the dimensions of $\C[S(X\setminus Y)_0]$ and $\C[S(X'\setminus Y)_0]$ are equal to $|S(X\setminus Y)_0|=|X\setminus Y|=3$ and $|S(X'\setminus Y)_0]|=|X'\setminus Y|=2$, $C_0(X\setminus Y,X'\setminus Y,\C)$ has dimension 1 as a $\C$-vector space and therefore it is isomorphic to $\C$.

Consider the following singular simplex:
\begin{align*}
		\mu:\Delta^1 &\rightarrow X\setminus Y \\
		(1-t)e_0+te_1 &\rightarrow\begin{cases}
				N\textit{ if } t\in[0,1/2) \\
				O\textit{ if } t\ge 1/2
		\end{cases}
\end{align*}

This is such that $\mu\in S(X\setminus Y)_1\setminus S(X'\setminus Y)_1$ (it's true that this is given by the fact that the codomain differs, but what I mean is that the latter can't be refined without altering the domain in order to see $\mu$ as a map to $X'\setminus Y$ and therefore it can't be in the image of the canonical inclusion $i:S(X'\setminus Y)_1\rightarrow S(X\setminus Y)_1$), hence the class of $1\mu$ is not 0 in $C_1(X\setminus Y,X'\setminus Y,\C)$. The image of this class under the induced boundary homomorphism is the class of $\partial_1(1\mu)=1\mu\delta_0-1\mu\delta_1=e_O-e_N$ in $C_0(X\setminus Y,X'\setminus Y,\C)$ (and again it is $\neq 0$ because $e_O\in S(X\setminus Y)_0\setminus S(X'\setminus Y)_0)$, thus the image of the induced boundary homomorphism has dimension $\ge 1$, therefore it is surjective.

Using the excision, $H_0(X,X',\C)\cong H_0(X\setminus Y,X'\setminus Y,\C)\cong 0$.

Being the sequence exact, $\psi$ is surjective. $\dim_{\C}\ker(\psi)=\dim_{\C} (\C^2)-\dim_{\C}(\Ima(\psi))=2-1=1$, hence $H_1(X,X',\C)\cong H_1(X\setminus Y,X'\setminus Y,\C)\cong\ker(\psi)\cong\C$. Let's go back to our exact sequence, which now looks like this:

$$0\rightarrow H_1(X,\C)\xrightarrow{f}\C\xrightarrow{g}\C\xrightarrow{h}\C\rightarrow 0$$

By exactness, $H_1(X,\C)\cong\Ima(f)=\ker(g)\cong\C$ because, being $\C$ a 1-dimensional $\C$-vector space, either a morphism is bijective or it is the 0-map. Given that $h$ is bijective, $g$ is the 0-map and therefore $f$ is bijective.




~\\
\exercise{2.4}

Let's set the following maps of sets:

\begin{align*}
		\partial_2 : \Q[\{T_j\ |\ j\in\{0,\ldots,3\}] & \rightarrow \Q[\{E_{j,k}\ |\ 1\leq j<k\leq 3\}] \\
			1T_0 & \mapsto 1E_{1,2}-1E_{1,3}+1E_{2,3} \\
			1T_1 & \mapsto 1E_{0,2}-1E_{0,3}+1E_{2,3} \\
			1T_2 & \mapsto 1E_{0,1}-1E_{0,3}+1E_{1,3} \\
			1T_3 & \mapsto 1E_{0,1}-1E_{0,2}+1E_{1,2} \\
		\partial_1 : \Q[\{\{E_{j,k}\ |\ 0\leq j<k\leq 3\}] & \rightarrow \Q[\{V_j\ |\ j\in\{0,\ldots,3\}\}] \\
			1E_{0,1} & \mapsto 1V_1-1V_0 \\
			1E_{0,2} & \mapsto 1V_2-1V_0 \\
			1E_{0,3} & \mapsto 1V_3-1V_0 \\
			1E_{1,2} & \mapsto 1V_2-1V_1 \\
			1E_{1,3} & \mapsto 1V_3-1V_1 \\
			1E_{2,3} & \mapsto 1V_3-1V_2
\end{align*}

We have defined the $\partial_i$ only on the functions sending a single element of the domain to $1\in\Q$ and all of the others to 0 because they form a basis (they are linearly independent and every other element by linear combination) of their respective $\Q$-vector spaces, which have respectively dimensions $4$, $6$ and $4$ (the dimensions are equal to the cardinality of the sets which we linearize because they are finite). A $\Q$-homomorphisms defined on a set of generators will extend uniquely on the remaining elements.

We see that, for any $h\in\{0,\ldots,3\}$, given $0\leq i<j<k\leq 3$, $i,j,k\neq h$, we have:

\begin{align*}
		(\partial_1\circ\partial_2)(1T_h) & = \partial_1(\partial_2(1T_h)) \\
		& = \partial_1(1E_{i,j}-1E_{i,k}+1E_{j,k}) \\
		& = \partial_1(1E_{i,j})-\partial_1(1E_{i,k})+\partial_1(1E_{j,k}) \\
		& = (1V_j-1V_i)-(1V_k-1V_i)+(1V_k-1V_j) \\
		& = 0
\end{align*}

We see that$$\partial_1(aE_{0,1}+bE_{0,2}+cE_{0,3}+dE_{1,2}+eE_{1,3}+fE_{2,3})=(-a-b-c)V_0+(a-d-e)V_1+(b+d-f)V_2+(c+e+f)V_3,$$hence the coefficients of an element in $\ker(\partial_1)$ will satisfy the following system of 4 equations over $\Q$:

$$
\begin{cases}
		-a-b-c=0 \\
		a-d-e=0 \\
		b+d-f=0 \\
		c+e+f=0 \\
\end{cases}
$$

Considering the associated matrix, we may observe that the submatrix obtained by removing the first two columns is $4\times 4$ and has determinant 2, hence we have that the matrix has rank 4, $\dim_{\Q}(\ker(\partial_1))=6-4=2$ and $\dim_{\Q}(\Ima(\partial_1))=4=\dim_{\Q}(\Q[\{V_j\ |\ j\in\{0,\ldots,3\}\}])$. This implies that $\partial_1$ is surjective, hence $H_0(C)=0$ and $\dim_{\Q}(H_0(C))=0$.

Furthermore, since $\Ima(\partial_2)\subset\ker(\partial_1)$, which has dimension 2, being $\partial_2(1T_0)$ linearly independent from $\partial_2(1T_1)$, we get that $\Ima(\partial_2)=\ker(\partial_1)$, hence $\dim_{\Q}(\Ima(\partial_2))=2$, $\dim_{\Q}(\ker(\partial_2))=4-2=2$, $H_1(C)=0$, $\dim_{\Q}(H_1(C))=0$ and $H_2(C)=\ker(\partial_2)$.

Now we compute the Euler characteristic of $C$, which is $\sum_{n\geq 0} (-1)^n\dim_{\Q}(H_n(C))=0-0+2=2$

\end{document}
