\documentclass{article}
\usepackage[T1]{fontenc}
\usepackage{lmodern}
\usepackage[utf8]{inputenc}
\usepackage[british]{babel}
\usepackage{geometry}
\usepackage{color}
\usepackage{amsthm}
\usepackage{amsmath,amssymb}
\usepackage{graphicx}
\usepackage{mathtools}
\usepackage{listings}
\usepackage{newlfont}
\usepackage{tikz-cd}

\newcommand{\numberset}{\mathbb}
\newcommand{\N}{\numberset{N}}
\newcommand{\Z}{\numberset{Z}}
\newcommand{\R}{\numberset{R}}
\newcommand{\Q}{\numberset{Q}}
\newcommand{\C}{\numberset{C}}
\newcommand{\K}{\numberset{K}}
\newcommand{\F}{\numberset{F}}
\newcommand{\n}{\mathcal{N}}
\newcommand{\aid}{\mathfrak{a}}
\newcommand{\bid}{\mathfrak{b}}
\newcommand{\pid}{\mathfrak{p}}
\newcommand{\qid}{\mathfrak{q}}
\newcommand{\mi}{\mathfrak{m}}
\newcommand{\I}{\mathbb{I}}
\newcommand{\V}{\mathbb{V}}
\newcommand{\Ps}{\mathbb{P}}

\DeclareMathOperator{\Id}{Id}
\DeclareMathOperator{\Ima}{Im}
\DeclareMathOperator{\sgn}{sgn}

\newcommand{\exercise}[1]{\noindent {\bf Exercise #1}}

\begin{document}

\title{Algebraic Topology 1 - Assignment 11}

\author{M. Durante, s2303760, Leiden University\\M. Fruttidoro, s2287129, Leiden University\\I. Prosepe, s2290162, Leiden University}

\maketitle


\exercise{12.1}

Let $m\leq n$ and consider a class $\alpha\in\pi_m(X,*\in A)$. This is represented by a map $(S^m,*)\xrightarrow{f} (X,*)$ and, seeing $(S^m,*)$ as a CW complex, since $f(\{*\})=\{*\}\subset A$, $f$ is a map of relative CW complexes. It follows by~\cite[thm. 12.1]{sag} that it is homotopic relative to $\{*\}$ to a cellular map $(S^m,*)\xrightarrow{g} (X,A)$, $g(*)=*$, which will again represent $\alpha$.

Given this, since $g(S^m)\subset X_m\subset X_n$, we can factor $g$ uniquely through $i_n$, which gives us a map $S^m\xrightarrow{\tilde{g}} X_n$ s.t. $g=i_n\circ\tilde{g}$. This implies that $(i_n)_*([\tilde{g}])=[i_n\circ\tilde{g}]=[g]=\alpha$, where $[\tilde{g}]\in\pi_n(X,*)$. The surjectivity of $(i_n)_*$ follows.

Consider the case where $m<n$. We shall prove its injectivity.

Let $\alpha,\beta\in\pi_m(X_n,*)$ and represent them by cellular maps $S^m\xrightarrow{f} X_n$ and $S^m\xrightarrow{g} X_n$ (the procedure to produce them is the same one we used earlier).

Assume that we can find a homotopy $S^m\times I\xrightarrow{H} X$ between $f$ and $g$, i.e. they represent the same element in $\pi_m(X,*)$. Since $I$ and $\partial I$ are finite CW complexes, $S^m\times I$ is a CW complex and the subspace $S^m\times\partial I$ is a subcomplex by~\cite[cor. 12.9]{sag}.

By~\cite[thm. 12.1]{sag}, we can find a cellular map $S^m\times I\xrightarrow{H'} X$ which restricts to $f$ and $g$ on the boundary components.

This implies that the map factors through the inclusion $i_n$, giving a homotopy $S^m\times I\xrightarrow{\tilde{H}} X_n$ between $f$ and $g$. It follows that $\alpha=\beta$.

\begin{thebibliography}{9}
  \bibitem{sag}
    S. Sagave,
    \textit{Algebraic Topology},
    2017
\end{thebibliography}

\end{document}
