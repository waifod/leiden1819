\documentclass{article}
\usepackage[T1]{fontenc}
\usepackage{lmodern}
\usepackage[utf8]{inputenc}
\usepackage[british]{babel}
\usepackage{geometry}
\usepackage{color}
\usepackage{amsthm}
\usepackage{amsmath,amssymb}
\usepackage{graphicx}
\usepackage{mathtools}
\usepackage{listings}
\usepackage{newlfont}
\usepackage{tikz-cd}
\usepackage{rotating}
\usepackage[backend=biber]{biblatex}
\addbibresource{~/math/references.bib}

\newcommand{\numberset}{\mathbb}
\newcommand{\N}{\numberset{N}}
\newcommand{\Z}{\numberset{Z}}
\newcommand{\R}{\numberset{R}}
\newcommand{\Q}{\numberset{Q}}
\newcommand{\K}{\numberset{K}}
\newcommand{\F}{\numberset{F}}
\newcommand{\n}{\mathcal{N}}
\newcommand{\aid}{\mathfrak{a}}
\newcommand{\bid}{\mathfrak{b}}
\newcommand{\pid}{\mathfrak{p}}
\newcommand{\qid}{\mathfrak{q}}
\newcommand{\mi}{\mathfrak{m}}
\newcommand{\I}{\mathbb{I}}
\newcommand{\V}{\mathbb{V}}
\newcommand{\A}{\mathbb{A}}
\newcommand{\Ps}{\mathbb{P}}
\newcommand{\exercise}[1]{\noindent {\bf Exercise #1}}

\DeclareMathOperator{\Ima}{Im}
\DeclareMathOperator{\coker}{coker}
\DeclareMathOperator{\Id}{Id}
\DeclareMathOperator{\GL}{GL}
\DeclareMathOperator{\Mat}{Mat}
\DeclareMathOperator{\Ext}{Ext}
\DeclareMathOperator{\Tor}{Tor}
\DeclareMathOperator{\Spec}{Spec}

\begin{document}

\title{Algebraic Geometry - Assignment 1}

\author{Matteo Durante, s2303760, Leiden University}

\maketitle

\exercise{1}

Throughout the exercise we will call $V$ the closed subsets of $\Spec(R)$.

\begin{proof}
$(i)$ Let $[\pid]\in\Spec(R)$. Then we have the following by definition:
$$\overline{\{[\pid]\}}=\bigcap_{[\pid]\in V} V=\bigcap_{\alpha\subset\pid}
V(\alpha)$$

Since $\pid\subset\pid$, clearly $\bigcap_{\alpha\subset\pid}
V(\alpha)=V(\pid)\cap (\bigcap_{\alpha\subsetneq\pid} V(\alpha))\subset
V(\pid)$. On the other hand, by definition $[\qid]\in V(\pid)$ is equivalent to
$\pid\subset\qid$, hence for every $\alpha\subset\pid$ we have that $[\qid]\in
V(\alpha)$ and therefore $V(\pid)\subset V(\alpha)$, thus
$V(\pid)\subset\bigcap_{\alpha\subset\pid} V(\alpha)$.$\quad\square$

\proof $(ii)$ Let $[\pid]\in V(\pid)\subset V\cup V'$, $V,V'$ closed. Then we
have that $[\pid]\in V$ or $[\pid]\in V'$, let's say the former. Since $V$ is
closed, it will contain $\overline{\{[\pid]\}}=V(\pid)$, hence we have the
thesis.

We will now prove that $[\pid]$ is the unique generic point of $V(\pid)$.

First of all, we know that it is one by $(i)$. Let $[\qid]$ be another. Then,
$\overline{\{[\qid]\}}=V(\qid)=V(\pid)$, hence $[\pid]\in V(\qid),[\qid]\in
V(\pid)$. By definition, $\qid\subset\pid$ and $\pid\subset\qid$, hence we are
done.
\end{proof}

\begin{proof}
$(iii)$ Consider $V$, a closed subset of $\Spec(R)$. It will be defined
by an ideal $\alpha$ of $R$. Clearly, since
$\sqrt{\alpha}=\bigcap_{\alpha\subset\pid} \pid$, a prime ideal $\pid$ of $R$
contains $\alpha$ if and only if it contains $\sqrt{\alpha}$, hence
$V(\alpha)=V(\sqrt{\alpha})$. On the other hand, if for some ideal $\beta$ we
have that $\beta\not\subset\sqrt{\alpha}$, then there exists an element 
$x\in\beta\setminus\sqrt{\alpha}\subset\beta\setminus\alpha$ and a prime ideal
$\pid\supset\alpha$ s.t. $x\not\in\pid$, i.e. $\beta\not\subset\pid$ and
therefore $[\pid]\not\in V(\beta)$. Also, it is clear that if $V(\alpha)=
V(\beta)$ we have that $\sqrt{\alpha}=\sqrt{\beta}$ by applying the earlier
result in both directions. We have shown that any ideal defining a closed
subset is contained in a unique
radical one defining the same subset. Furthermore, if $V(\beta)\supset
V(\alpha)$ with $\alpha$ radical, since any prime ideal $\pid\supset\alpha$ is
s.t. $\beta\subset\pid$, we get that $\beta\subset\alpha$.

Suppose now that $\alpha$ is an ideal of $R$ s.t. the closed subset $V(\alpha)$ is
irreducible; let $\beta,\gamma$ be ideals of $R$ s.t.
$\beta\gamma\subset\sqrt{\alpha}$, i.e. $V(\beta)\cup V(\gamma)=V(\beta\gamma)\supset
V(\sqrt{\alpha})=V(\alpha)$.
We know by irreducibility that $V(\beta)\supset V(\sqrt{\alpha})$ or
$V(\gamma)\supset V(\sqrt{\alpha})$, hence
$\beta\subset\sqrt{\alpha}$ or $\gamma\subset\sqrt{\alpha}$. Since this holds
for every pair of ideals whose product is contained in $\sqrt{\alpha}$, the
latter is a prime ideal.
\end{proof}


~\\
\exercise{2}

\begin{proof}
    Consider for an element $(a_P)_{[P]\in U}\in\Gamma(U,\mathcal{O}_X)$ a
    principal open covering $(X_f)_f$ of $U$, $f\in R$, s.t. for every $f$ there
    is a $a_f\in R_f$ with the property that, for every $[P]\in X_f$,
    $a_f\mapsto (a_f)_P=a_P$ under the canonical homomorphism $R_f\rightarrow
    R_P$. We want to check that $(a_P)_{[P]\in V}\in\Gamma(V,\mathcal{O}_X)$.

    To do this, consider a principal open covering $(X_g)_g$ of $V$, $g\in R$.
    We know that for every $f,g$ we have that $X_f\cap X_g=X_{fg}$ and, since
    $\bigcup_{f,g} X_{fg}=\bigcup_f\bigcup_g (X_f\cap X_g)=\bigcup_f
    (X_f\cap\bigcup_g X_g)=\bigcup_f (X_f\cap V)=U\cap V=V$, $(X_{fg})_{f,g}$ is
    a new principal open covering of $V$.
    
    Consider now for every $X_{fg}$ the element $(a_f)_{fg}=a_{fg}$, image of $a_f$
    under the canonical homomorphism $R_f\rightarrow R_{fg}$. We know that, for
    every $[P]\in X_{fg}\subset X_f$, the following diagram commutes:
    \[
        \begin{tikzcd}
            R_f\arrow{rr}{(-)_{fg}}\arrow[swap]{dr}{(-)_P}
            && R_{fg}\arrow{dl}{(-)_P} \\
            & R_P
        \end{tikzcd}
    \]
    It follows that $(a_{fg})_P=((a_f)_{fg})_P=(a_f)_P=a_P$, hence we are done.
\end{proof}


~\\
\exercise{3}

\begin{proof}
    Remember that, for any $x\in X$, the elements $f_x$ of $\mathcal{O}_{X,x}$,
    the stalk of $\Gamma(-,\mathcal{O}_X)$ at $x$, are elements $f\in\Gamma(U,
    \mathcal{O}_X)$, with $U\subset X$ open and containing $x$, under the
    equivalence relation where $(f,U)\sim (g,V)$ if and only if there exists
    an open $W\subset U\cap V$ s.t. $x\in W$ and $f|_W=g|_W$.

    Suppose now that $f_x=[(U,f)]\in\mathcal{O}_{X,x}$, $f_x\neq 0_x$, is
    nilpotent. This means that $(f_x)^n=0_x$ for some $n>1$ and, since for every
    open $U$ containing $x$ the map
    $\Gamma(U,\mathcal{O}_X)\xrightarrow{(-)_x}\mathcal{O}_{X,x}$ is a ring
    homomorphism, $(f^n)_x=0_x$. It follows that there exists an
    open $W\subset U\cap X$, $x\in W$, s.t. $f^n|_W=0|_W$. Since the restriction
    $\Gamma(U,\mathcal{O}_X)\xrightarrow{|_W}\Gamma(W,\mathcal{O}_X)$ is again a
    ring homomorphism, this means that $(f|_W)^n=0|_W$. On the other hand, $f|_W
    \neq 0|_W$, for otherwise $f_x=[(U,f)]=[(W,f|_W)]=0_x$. It follows that
    $f|_W\in\Gamma(W,\mathcal{O}_X)$ is nilpotent.
    
    Let $U\subset X$ be an open s.t. $\Gamma(U,\mathcal{O}_X)$ has nilpotents.
    This means that there is a non-zero element $(a_x)_{x\in U}\in\Gamma(U,
    \mathcal{O}_X)\subset\Pi_{x\in U}\mathcal{O}_{X,x}$ and a $n>1$ for which
    $((a_x)_{x\in U})^n=(a_x^n)_{x\in U}=(0)_{x\in U}=0|_U$. Since $(a_x)_{x\in U}
    \neq 0$, we have for some $x\in U$ that $a_x\neq 0$ and $a_x^n=0$, i.e.
    $a_x\in\mathcal{O}_{X,x}$ is nilpotent.
    
    Let now $x\in X$ lie in $U_i$. It will be identified with $[P]\in
    \Spec(R_i)$. We know that the identification
    of an open neighbourhood of $x$ with an affine scheme is an isomorphism of
    locally ringed spaces and that under this isomorphism stalks are carried
    isomorphically from one scheme to the other. In particular, the stalk of
    $\Spec(R_i)$ at $[P]$, which is $(R_i)_P$ by~\cite[p. 100]{Mum88}, is mapped
    isomorphically to the one of $X$ at $x$ i.e.
    $\mathcal{O}_{X,x}\cong (R_i)_P$. Since the latter has no nilpotents, so
    does the former, hence $X$ is a reduced scheme by what we have proved
    earlier.
\end{proof}


~\\
\exercise{4}

We will take for granted that any maximal ideal in $\Z[X]$ is given by $(p,f)$,
where $p\in\Z$ is prime and $f\in\Z[X]$ is s.t. $[f]_p\in\F_p[X]$ is
irreducible.

Consider two distinct primes $p,q\in\Z$. Then, for some $n,m\in\Z\setminus\{0\}$
we have that $np+mq=1$. It follows that $(p)+(q)=(1)$ in $\Z[X]$, hence
$V((p))\cap V((q))=V((p)+(q))=V((1))=\emptyset$.

Consider now a prime $p\in\Z$ and an irreducible polynomial $f\in\Z[X]$. Either
they are s.t. $(p,f)=(1)$, and therefore $V((p))\cap V((f))=V((p,f))=V((1))=
\emptyset$, or the only prime ideals containing it are the
maximal ones: indeed, if $q\in\Z$ is prime, $(p,f)\subset (q)$ would mean that
$q|f$, which is impossible because $f$ is primitive, while if $g\in\Z[X]$ is an
irreducible polynomial and $(p,f)\subset (g)$ we get that $g|p$, which is again
not possible.

Now, if $(p,f)\neq (1)$, then the maximal ideals containing it correspond
bijectively to the ones of $\Z[X]/(p,f)\cong\F_p[X]/([f]_p)$, i.e. to the ones
in $\F_p[X]$ containing $[f]_p$. Since $\F_p[X]$ is a PID, the irreducible
polynomials generate prime ideals, which to contain $[f]_p$ have to be generated
by an element dividing it. It follows that in $\F_p[X]/([f]_p)$ the only prime
ideals are given by the irreducible factors of $[f]_p\in\F_p[X]$ and
they are maximal by what we have shown earlier, hence we find that $V((p))\cap
V((f))=V((p,f))=\{[(p,g)]\ |\ [g]_p\in\F_p[X]\textit{ is irreducible and }
[g]_p|[f]_p\}$.

Both cases are possible: for the former consider $p=2,\ f=2X+1$, while for the
latter $p=2,\ f=X$. Furthermore, the intersection is always finite since
$[f]_p$ has finitely many irreducible factors in $\F_p[X]$.

Finally, consider two irreducible polynomials $f,g\in\Z[X]$ s.t. $(f)\neq
(g)$ (and therefore coprime). We know that, for any prime element $a\in\Z[X]$,
$(f,g)\not\subset (a)$, for otherwise $a|f,g$ and therefore $a|\gcd(f,g)$.

It follows that either $(f,g)=(1)$ or the only primes containing it are maximal
ideals $(p,h)$ such that $h$ is irreducible modulo $p$ and $h|f,g\mod p$, i.e.
$[h]_p$ is an irreducible factor of $\gcd([f]_p,[g]_p)$, by an argument similar
to the one previously used and an isomorphism we are about to mention. In particular,
given a prime $p$ s.t. $(p,f,g)$ is proper, we have a bijection between
the maximal ideals containing $(p,f,g)$ and the ones of
$\Z[X]/(p,f,g)\cong\F_p[X]/([f]_p,[g]_p)\cong\F_p[X]/(\gcd([f]_p,[g]_p))$
(remember that $\F_p[X]$ is a PID, hence every ideal is generated by the gcd of
a system of generators).

In the former case, $V((f))\cap V((g))=\emptyset$, while in the latter
$=\{[(p,h)]\ |\ p\in\Z\textit{ is prime},\ [h]_p\in\F_p[X]\textit{ is irreducible}
\textit{ and }[h]_p|\gcd([f]_p,[g]_p)\}$.

We will prove that even in this latest case the intersection is finite.

Since $f,g$ are coprime, there exists a pair of polynomials $h,l\in\Q[x]$
s.t. $hf+lg=1$. Let $n,m\in\Z\setminus\{0\}$
be integers s.t. $nh,mg\in\Z[X]$. We have then that
$(nmh)f+(nml)g=nm\in (f,g)$ for a pair of integer polynomials
$nmh,nml\in\Z[X]$. It follows that a prime $p\in\Z$ defining a maximal ideal
containing $(f,g)$ has to divide $nm$, hence there are finitely many such
primes. Since for every prime $\gcd([f]_p,[g]_p)$ has finitely many irreducible
factors, by the previously mentioned bijection we are done.

Both cases are possible: for the former consider $f=X,\ g=X+1$, while for
the latter $f=X,\ g=X+2$.


\printbibliography

\end{document}
