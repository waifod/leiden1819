\documentclass{article}
\usepackage[T1]{fontenc}
\usepackage{lmodern}
\usepackage[utf8]{inputenc}
\usepackage[british]{babel}
\usepackage{geometry}
\usepackage{color}
\usepackage{amsthm}
\usepackage{amsmath,amssymb}
\usepackage{graphicx}
\usepackage{mathtools}
\usepackage{listings}
\usepackage{newlfont}
\usepackage{tikz-cd}
\usepackage{rotating}
\usepackage[backend=biber]{biblatex}
\addbibresource{~/math/references.bib}

\newcommand{\numberset}{\mathbb}
\newcommand{\N}{\numberset{N}}
\newcommand{\Z}{\numberset{Z}}
\newcommand{\Q}{\numberset{Q}}
\newcommand{\R}{\numberset{R}}
\newcommand{\C}{\numberset{C}}
\newcommand{\K}{\numberset{K}}
\newcommand{\F}{\numberset{F}}
\newcommand{\n}{\mathcal{N}}
\newcommand{\aid}{\mathfrak{a}}
\newcommand{\bid}{\mathfrak{b}}
\newcommand{\pid}{\mathfrak{p}}
\newcommand{\qid}{\mathfrak{q}}
\newcommand{\mi}{\mathfrak{m}}
\newcommand{\I}{\mathbb{I}}
\newcommand{\V}{\mathbb{V}}
\newcommand{\A}{\mathbb{A}}
\newcommand{\Ps}{\mathbb{P}}
\newcommand{\exercise}[1]{\noindent {\bf Exercise #1}}

\DeclareMathOperator{\im}{im}
\DeclareMathOperator{\coker}{coker}
\DeclareMathOperator{\Id}{Id}
\DeclareMathOperator{\GL}{GL}
\DeclareMathOperator{\Mat}{Mat}
\DeclareMathOperator{\Ext}{Ext}
\DeclareMathOperator{\Tor}{Tor}
\DeclareMathOperator{\Hom}{Hom}
\DeclareMathOperator{\Spec}{Spec}


\begin{document}

\title{Algebraic Geometry II - Assignment 4}

\author{Matteo Durante, s2303760, Leiden University}

\maketitle


\exercise{1}

\begin{proof}
    $(i)$ We know that, for any open $U\subset X\setminus K$, we have
    $i_*\mathcal{F}(U)=\mathcal{F}(\emptyset)=0$, hence for any $x\in X\setminus
    K$ we get that $m_x=[(V,m)]\in (i_*\mathcal{F})_x$ is s.t. $m_x=[(V,m)]=
    [(V\setminus K,m|_{V\setminus K})]=[(V\setminus K,0|_{V\setminus K})]=0_x$
    and therefore $(i_*\mathcal{F})_x=0$.

    On the other hand, for any $x\in K$, we have that
    $(i_*\mathcal{F})_x=\varinjlim_{x\in U\subset X}i_*\mathcal{F}(U)=
    \varinjlim_{x\in U\subset X}\mathcal{F}(U\cap K)=\varinjlim_{x\in U\subset
    K}\mathcal{F}(U)=\mathcal{F}_x$.

    Now, given an exact sequence
    $\mathcal{F}\xrightarrow{\phi}\mathcal{G}\xrightarrow{\psi}\mathcal{H}$ of
    sheaves on $K$, consider the induced sequence
    $i_*\mathcal{F}\xrightarrow{i_*\phi}i_*\mathcal{G}\xrightarrow{i_*\psi}
    i_*\mathcal{H}$ of sheaves on $X$. We know that a sequence of sheaves on $X$
    is exact if and only if for every $x\in X$ the induced sequence
    $(i_*\mathcal{F})_x\xrightarrow{(i_*\phi)_x}(i_*\mathcal{G})_x
    \xrightarrow{(i_*\psi)_x}(i_*\mathcal{H})_x$ on the stalks is exact.

    For $x\in X\setminus K$, all of these stalks are 0 and the sequence is
    trivially exact. For $x\in K$, the sequence becomes to
    $\mathcal{F}_x\xrightarrow{\phi_x}\mathcal{G}_x\xrightarrow{\psi_x}
    \mathcal{H}_x$, which is exact because the one we started from is. This
    concludes the proof.
\end{proof}

\begin{proof}
    $(ii)$ Let $U\subset V\subset X$ be open. We have that $V\cap K\subset U\cap
    K$ are open of $K$ and the restriction map
    $i_*\mathcal{F}(V)=\mathcal{F}(V\cap K)\rightarrow
    i_*\mathcal{F}(U)=\mathcal{F}(U\cap K)$ is precisely the
    restriction map $\mathcal{F}(V\cap K)\rightarrow\mathcal{F}(U\cap K)$. Since
    the latter is surjective, so is the first, hence $i_*\mathcal{F}$ is a
    flasque sheaf and the thesis follows.
\end{proof}

\begin{proof}
    $(iii)$ We know that flasque sheaves are $\Gamma$-acyclic, hence, given a
    flasque resolution of $\mathcal{F}$,
    $(0)\rightarrow\mathcal{F}\rightarrow\mathcal{G}^\bullet$, and the induced
    one on $i_*\mathcal{F}$, $(0)\rightarrow i_*\mathcal{F}\rightarrow
    i_*\mathcal{G}^\bullet$, we have that $h^n(\Gamma(K,\mathcal{G}^\bullet))\cong
    R^n\Gamma(K,\mathcal{F})=H^n(K,\mathcal{F})$,
    $h^n(\Gamma(X,i_*\mathcal{G}^\bullet))
    \cong R^n\Gamma(X,i_*\mathcal{F})=H^n(X,i_*\mathcal{F})$ for every $n\in\N$.

    Notice that, given $\mathcal{G}^n\xrightarrow{\phi}\mathcal{G}^{n+1}$, the
    morphism $i_*\mathcal{G}^n\xrightarrow{i_*\phi}i_*\mathcal{G}^{n+1}$ is
    s.t. $i_*\phi(X)=\phi(i^{-1}(X))=\phi(K)$.

    By definition, for $n\in\N_{>0}$:
    \begin{align*}
        h^n(\Gamma(X,i_*\mathcal{G}^\bullet))&=\ker(i_*\mathcal{G}^n(X)\rightarrow
        i_*\mathcal{G}^{n+1}(X))/\im(i_*\mathcal{G}^{n-1}(X)\rightarrow
        i_*\mathcal{G}^n(X)) \\
        &=\ker(\mathcal{G}^n(K)\rightarrow\mathcal{G}^{n+1}(K))/
        \im(\mathcal{G}^{n-1}(K)\rightarrow\mathcal{G}^n(K)) \\
        &=h^n(\Gamma(K,\mathcal{G}^\bullet)
    )\end{align*}

    Similarly, for $n=0$, we have that:
    \begin{align*}
        h^0(\Gamma(X,i_*\mathcal{G}^\bullet))&=\ker(i_*\mathcal{G}^0(X)\rightarrow
    i_*\mathcal{G}^1(X)) \\
        &=\ker(\mathcal{G}^0(K)\rightarrow\mathcal{G}^1(K)) \\
        &=h^0(\Gamma(K,\mathcal{G}^\bullet))
    \end{align*}

    The thesis follows.
\end{proof}


~\\
\exercise{2}

\begin{proof}
    $(i)$ Let $x\in X\setminus\{\eta\}$. We know that $\overline{\{x\}}=Y$ is an
    irreducible closed subscheme of $X$. Since $x\neq\eta$, $Y\subsetneq X$ by
    uniqueness of the generic point.

    Now, since $\dim(X)=1$, we have that $\dim(Y)=0$ and, having
    $\overline{\{y\}}\subset Y$ for any $y\in Y$, being $\overline{\{y\}}$ an
    irreducible closed subscheme of $Y$, we have that $\overline{\{y\}}=Y$, thus
    $y=x$ again by the uniqueness of the generic point and $Y=\{x\}$.

    It follows that $X=|X|\cup\{\eta\}$.
\end{proof}

\begin{proof}
    $(ii)$ Let $K(X)\xrightarrow{\phi}\Pi_{x\in |X|}K(X)/\mathcal{O}_{X,x}$ be
    the map $f\mapsto ([f]_x)_{x\in |X|}$, which is natural and $\K$-linear as
    it is given by the direct product of the natural projections
    $K(X)\rightarrow K(X)/\mathcal{O}_{X,x}$ for $x\in |X|$. We have to check
    that we may restrict its codomain to $\bigoplus_{x\in
    |X|}K(X)/\mathcal{O}_{X,x}$, which is equivalent to proving that, for any
    $f\in K(X)$, $f\not\in\mathcal{O}_{X,x}$ for finitely many $x\in |X|$.

    Let's cover $X$ by finitely many affine open subschemes
    $(U_i=\Spec(R_i))_{i=1}^n$. Each $R_i$ will then be a
    finitely generated $\K$-algebra and a domain because $X$ is an integral
    $\K$-scheme of finite type.
    
    Also, $\eta\in U_i$ for every $i$ by density and it
    corresponds to the zero-ideal of $R_i$, hence either $R_i$ is a field and
    then $U_i=\{\eta\}$ (this actually can't even be), which implies that we may
    discard $U_i$ from the
    covering as it is redundant, or it isn't and therefore it has a maximal
    ideal $\mi$ corresponding to some point $x\in U_i$ s.t. $x\neq\eta$. After
    refining the covering, we may assume that $|U_i|>1$ and therefore, being $x$
    closed, we have a chain of irreducible closed subschemes of $U_i$ given by
    $\emptyset\subsetneq\{x\}\subsetneq U_i$. From this, $\dim(U_i)>0$. On the
    other hand, since $\dim(X)=1$, $\dim(U_i)\leq 1$, hence $\dim(U_i)=1$.
    
    Remember that, for any $x\in X$, the
    fraction field of $\mathcal{O}_{X,x}$ is given precisely by $K(X)$ and
    corresponds to the localization at $(0)$.

    Now, for any $x\in X$ there exists a $i$ s.t. $x\in U_i$ and therefore a
    prime $\pid\subset R_i$ s.t. $(R_i)_\pid=\mathcal{O}_{X,x}$.

    Let $f\in R_i$, $f\neq 0$. We will prove that there are finitely many prime
    ideals $\pid\subset R_i$ s.t. $f\in\pid$.

    Indeed, consider the ideal $(f)$. Since $R_i$ is Noetherian, it admits a
    minimal primary decomposition $(f)=\bigcap_{j=0}^m\qid_j$. Taking radicals,
    we have that $r((f))=\bigcap_{j=0}^mr(\qid_j)=\bigcap_{j=0}^m\pid_j$. If for
    some prime $\pid$ we have $f\in\pid$, then
    $\bigcap_{j=0}^m\pid_j\subset\pid$ and therefore $\pid_j\subset\pid$ for
    some $j$. Since $R_i$ has Krull dimension 1, this means that $\pid_j=\pid$,
    hence we have proved our claim.

    Now, any element $f\in K(X)$, $f\neq 0$, is s.t. it can be written for every
    $i$ as $a_i/b_i$ with $a_i,b_i\in R_i\setminus\{0\}$, where we can assume
    $a_i,b_i$ to be relatively prime and therefore uniquely defined up to
    invertibles because $R_i$ is a domain. Now, since $b_i\in\pid$ only for
    finitely many primes in $R_i$, we have that $f=a_i/b_i\not\in
    (R_i)_\pid=\mathcal{O}_{X,x}$ for finitely many points $x\in U_i$. Since the
    $U_i$ form a finite covering of $X$, we have the thesis.
\end{proof}

\begin{proof}
    $(iii)$ Let's consider for every open $\emptyset\subsetneq U\subset X$ the
    natural map $\mathcal{K}(U)=K(X)\xrightarrow{\psi(U)}
    (\bigoplus_{x\in |X|}i_{x,*}(K(X)/\mathcal{O}_{X,x}))(U)=
    \bigoplus_{x\in |X|\cap U}K(X)/\mathcal{O}_{X,x}$ given by
    $res^X_U\circ\phi$. This by construction defines a morphism of sheaves as,
    for every open $\emptyset\subsetneq V\subset U$,
    $res^U_V\circ\psi(U)=res^U_V\circ res^X_U\circ\phi=res^X_V\circ\phi=\psi(V)=
    \psi(V)\circ res^U_V$.

    Consider the sequence
    $(0)\rightarrow\mathcal{O}_X\rightarrow\mathcal{K}_X\xrightarrow
    {\psi}\bigoplus_{x\in |X|}i_{x,*}(K(X)/\mathcal{O}_{X,x})\rightarrow (0)$.
    We want to show that it is exact on the stalks and therefore exact, which
    will give us the thesis by the uniqueness up to unique isomorphism of the
    cokernel.

    If we can prove that $(\bigoplus_{x\in
    |X|}i_{x,*}(K(X)/\mathcal{O}_{X,x}))_y=K(X)/\mathcal{O}_{X,y}$ for all $y\in
    |X|$ we are almost done as the sequence on the stalks then is
    $0\rightarrow\mathcal{O}_{X,y}\rightarrow\mathcal{K}_{X,y}=K(X)\rightarrow
    K(X)/\mathcal{O}_{X,y}\rightarrow 0$, which is exact ($\psi_y$ is precisely
    the projection map).

    Notice that each $y\in |X|$ is a closed point, hence
    $(i_{x,*}(K(X)/\mathcal{O}_{X,x}))_y=K(X)/\mathcal{O}_{X,x}$ if $x=y$ and
    $=0$ otherwise. Also, since we are working on a Noetherian topological
    space, $(\bigoplus_{x\in
    |X|}i_{x,*}(K(X)/\mathcal{O}_{X,x}))(U)=\bigoplus_{x\in
    |X|}i_{x,*}(K(X)/\mathcal{O}_{X,x})(U)$.
    
    Let $[(U,(f_x)_{x\in |X|\cap U})]\in(\bigoplus_{x\in |X|}i_{x,*}(K(X)/
    \mathcal{O}_{X,x}))_y$ and consider the finite closed subset
    $W=\{x\in |X|\cap U\ |\ f_x\neq 0,\ y\neq x\}$. We see that $[(U,(f_x)_{x\in
    |X|\cap U})]=[(U\setminus W,(f_x)_{x\in (|X|\cap U)\setminus W})]$ and
    $f_x=0$ for all $x\in (|X|\cap U)\setminus W$ s.t. $y\neq x$, thus the stalk
    is contained in $K(X)/\mathcal{O}_{X,y}$. On the other hand, by definition,
    for every open $U$ with $y\in U$ we have $K(X)/\mathcal{O}_{X,y}\subset
    (\bigoplus_{x\in|X|}i_{x,*}(K(X)/\mathcal{O}_{X,x}))(U)$, hence the stalk is
    precisely $K(X)/\mathcal{O}_{X,y}$.
    
    We are left with checking exactness at $y=\eta$. Making use of the same
    procedure, considering this time $W=\{x\in |X|\cap U\ |\ f_x\neq 0\}$, we
    get $[(U,(f_x)_{x\in |X|\cap U})]=[(U\setminus W,(0_x)_{x\in (|X|\cap
    U)\setminus W})]$, thus $(\bigoplus_{x\in |X|}i_{x,*}(K(X)/
    \mathcal{O}_{X,x}))_\eta=0$, and $\mathcal{O}_{X,\eta}=K(X)=
    \mathcal{K}_{X,\eta}$ hence the sequence becomes
    $0\rightarrow K(X)\rightarrow K(X)\rightarrow 0\rightarrow 0$, which is
    exact.
\end{proof}

\begin{proof}
    $(iv)$ First of all, notice that the sheaf given by $K(X)/\mathcal{O}_{X,x}$
    is trivially a flasque on $\{x\}$. Also, by $(1.iii)$ $i_{x,*}$ sends
    flasque sheaves to flasque sheaves for any $x\in |X|$, hence
    $i_{x,*}(K(X)/\mathcal{O}_{X,x})$ is again a flasque sheaf on $X$.

    Clearly, for any open $U\subset X$ we have that $(\bigoplus_{x\in
    |X|}i_{x,*}(K(X)/\mathcal{O}_{X,x}))(U)=\bigoplus_{x\in
    |X|}i_{x,*}(K(X)/\mathcal{O}_{X,x})(U)=\bigoplus_{x\in
    |X|\cap U}K(X)/\mathcal{O}_{X,x}$ and the restriction maps are the obvious
    projections.

    Let $V\subset U$, $(f_x)_{x\in |X|\cap V}\in\bigoplus_{x\in |X|\cap V}K(X)/
    \mathcal{O}_{X,x}$. By setting $g_x=f_x$ for $x\in |X|\cap V$, $g_x=0$ for
    $x\in U\setminus V$, we get an element $(g_x)_{x\in |X|\cap U}\in
    \bigoplus_{x\in |X|\cap U}K(X)/\mathcal{O}_{X,x}$ which is mapped to
    $(f_x)_{x\in |X|\cap V}$ under the restriction map.
    
    It follows that $\mathcal{K}_X/\mathcal{O}_X\cong
    \bigoplus_{x\in |X|}i_{x,*}(K(X)/\mathcal{O}_{X,x})$ is a flasque sheaf.
    Since $\mathcal{K}_X$ is a constant
    sheaf and therefore flasque, being the sequence $(0)\rightarrow\mathcal{O}_X
    \rightarrow\mathcal{K}_X\rightarrow\mathcal{K}_X/\mathcal{O}_X\rightarrow
    (0)$ exact, we have that it is also a flasque resolution of $\mathcal{O}_X$
    with $F_0=\mathcal{K}_X,\ F_1=\mathcal{K}_X/\mathcal{O}_X,\ F_i=(0)$ for
    $i>1$.
\end{proof}

\begin{proof}
    $(v)$ Using the just mentioned flasque resolution, by definition:
    \begin{align*}
        H^0(X,\mathcal{O}_X) &=\ker(\Gamma(X,F_0)\rightarrow\Gamma(X,F_1)) \\
        &=\ker\left(K(X)\rightarrow\bigoplus_{x\in |X|}K(X)/
        \mathcal{O}_{X,x}\right) \\
        &=\bigcap_{x\in |X|}\mathcal{O}_{X,x}
    \end{align*}
    
    This is because the only elements sent by our natural map to $([0]_x)_{x\in
    |X|}$ are the ones which belong to $\mathcal{O}_{X,x}$ for all $x\in |X|$.

    Similarly:
    \begin{align*}
        H^1(X,\mathcal{O}_X)
        &=\ker(\Gamma(X,F_2)\rightarrow\Gamma(X,F_3))/\im(\Gamma(X,F_1)\rightarrow\Gamma(X,F_2)) \\
        &=\left(\bigoplus_{x\in
        |X|}K(X)/\mathcal{O}_{X,x}\right)/\im\left(K(X)\rightarrow
        \bigoplus_{x\in |X|}K(X)/\mathcal{O}_{X,x}\right) \\
        &=\coker\left(K(X)\rightarrow\bigoplus_{x\in
        |X|}K(X)/\mathcal{O}_{X,x}\right)
    \end{align*}
\end{proof}

\begin{proof}
    $(vi)$ Remember that we have the following long exact sequence of cohomology
    groups induced by the previously mentioned short exact sequence of sheaves:
    $$\cdots\rightarrow H^{n-1}(X,\mathcal{K}_X/\mathcal{O}_X)\rightarrow
    H^n(X,\mathcal{O}_X)\rightarrow H^n(X,\mathcal{K}_X)\rightarrow
    H^n(X,\mathcal{K}_X/\mathcal{O}_X)\rightarrow\cdots$$

    Since $\mathcal{K}_X$ and $\mathcal{K}_X/\mathcal{O}_X$ are flasque sheaves,
    $H^n(X,\mathcal{K}_X)=H^{n-1}(X,\mathcal{K}_X/\mathcal{O}_X)=0$ for every
    $n>1$, hence by exactness $H^n(X,\mathcal{O}_X)=0$ for such $n$.
\end{proof}

\begin{proof}
    $(vii)$ We know already know that $X=\Ps^1_\K$ is a Noetherian integral
    scheme and that $\Ps^1_\K=U_0\cup U_1$, where $U_0=\Spec(\K[x_{10}]),\
    U_1=\Spec(\K[x_{01}])$ are open integral affine $\K$-subschemes of finite
    type and Krull dimension 1. It follows that $\Ps^1_\K$ is also an integral
    $\K$-scheme of finite type. We still have to prove that $\dim(X)=1$
    to show that it is a curve.

    Both and $U_0,\ U_1$ are irreducible and contain the unique generic point
    $\eta$. Also, $U_{01}=U_0\cap U_1=\Spec(\K[x_{01},x_{10}])$ is irreducible as
    well as it contains $\eta$.

    Notice that every irreducible closed subset $V\subset X$ has to contain at
    least one point $x\neq\eta$. If $x\in U_{01}$, then it is closed in both
    $U_i$ and therefore closed in $X$. If $x\in X\setminus U_j\subset U_i$,
    then, being closed in $U_i$, we have that $\{x\}=U_i\cap W=
    (X\setminus U_j)\cap W$ for some closed subset $W\subset X$, hence it is
    closed also in $X$. We have then a chain of irreducible closed subsets
    $\{x\}\subset V\subset X$ and a decomposition $X=|X|\cup\{\eta\}$ because
    every point $x\in U_i$ besides $\eta$ is closed.
    
    We know that the intersection of an irreducible closed subset with an open
    subset is still irreducible (or empty), hence $V\cap U_i=U_i$, which
    then implies $V=X$ as $U_i$ is dense in $X$, or $V\cap U_i=\{x\}$ for
    some $x\in |X|$, or $V\cap U_i=\emptyset$, which then implies
    $V=V\cap U_j=\{x\}$ for some $x\in |X|$. In the second case, either
    $V\cap U_j=\emptyset$, and
    therefore $V=\{x\}$, or $V\cap U_j=\{y\}$ for some $y\in |X|\setminus U_i$,
    which gives us $V=\{x,y\}$, which is reducible as $\{x\}\cup\{y\}$ and
    we have then a contradiction.

    We can conclude that either $V=X$ or $V=\{x\}$, hence $\dim(X)=1$.

    Notice that, since we are working with a $\K$-scheme,
    $\K\subset\mathcal{O}_{X,x}$ for every $x\in X$. Clearly:
    $$
        \bigcap_{x\in U_i}\mathcal{O}_{X,x}=
        \bigcap_{\pid\subset\K[x_{ji}]}\K[x_{ji}]_\pid
        \subset\mathcal{O}_{X,\eta}=Q(\K[x_{ji}])= Q(\K[x_{ij}])
    $$

    To make things easier, we will denote $x_{ji}$ by $t$.

    Since $\K$ is a field, $\K[t]$ is a principal ideal domain and therefore
    $\emptyset\subsetneq\pid=(g(t))$ for some irreducible polynomial
    $g(t)\in\K[t]$. Viceversa, every irreducible polynomial $g(t)\in\K[t]$
    defines a prime ideal $\pid=(g(t))$.

    We know that an element $f(t)=b(t)/c(t)\in Q(\K[t])$, where numerator and
    denominator are coprime, is s.t. $f(t)\in\K[t]_{(g(t))}$ if and only if
    $c(t)\not\in (g(t))$, i.e. $g(t)\nmid c(t)$. Since any non-constant
    polynomial can be factored as a product of irreducible ones, if
    $c(t)\not\in\K$ there exists
    some prime $\pid$ s.t. $c(t)\in\pid$. It follows that the only elements of
    $Q(\K[t])$ in $\bigcap_{\pid\subset\K[t]}\K[t]_\pid$ are the ones s.t.
    $c(t)\in\K$, hence $\bigcap_{\pid\subset\K[x_{ji}]}\K[x_{ji}]_\pid=
    \K[x_{ji}]$.

    Finally, by $(2.v)$, since we can identify $x_{ij}$ with $x_{ji}^{-1}$, we
    may write:
    $$
    H^0(X,\mathcal{O}_X)=\bigcap_{x\in |X|}\mathcal{O}_{X,x}
    =\bigcap_{x\in X}\mathcal{O}_{X,x}
    =\left(\bigcap_{x\in U_0}\mathcal{O}_{X,x}\right)\cap
    \left(\bigcap_{x\in U_1}\mathcal{O}_{X,x}\right)
    =\K[x_{10}]\cap\K[x_{01}]
    =\K
    $$
\end{proof}

\printbibliography

\end{document}
