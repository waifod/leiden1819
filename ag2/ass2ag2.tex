\documentclass{article}
\usepackage[T1]{fontenc}
\usepackage{lmodern}
\usepackage[utf8]{inputenc}
\usepackage[british]{babel}
\usepackage{geometry}
\usepackage{color}
\usepackage{amsthm}
\usepackage{amsmath,amssymb}
\usepackage{graphicx}
\usepackage{mathtools}
\usepackage{listings}
\usepackage{newlfont}
\usepackage{tikz-cd}
\usepackage{rotating}
\usepackage[backend=biber]{biblatex}
\addbibresource{~/math/references.bib}

\newcommand{\numberset}{\mathbb}
\newcommand{\N}{\numberset{N}}
\newcommand{\Z}{\numberset{Z}}
\newcommand{\Q}{\numberset{Q}}
\newcommand{\R}{\numberset{R}}
\newcommand{\C}{\numberset{C}}
\newcommand{\K}{\numberset{K}}
\newcommand{\F}{\numberset{F}}
\newcommand{\n}{\mathcal{N}}
\newcommand{\aid}{\mathfrak{a}}
\newcommand{\bid}{\mathfrak{b}}
\newcommand{\pid}{\mathfrak{p}}
\newcommand{\qid}{\mathfrak{q}}
\newcommand{\mi}{\mathfrak{m}}
\newcommand{\I}{\mathbb{I}}
\newcommand{\V}{\mathbb{V}}
\newcommand{\A}{\mathbb{A}}
\newcommand{\Ps}{\mathbb{P}}
\newcommand{\exercise}[1]{\noindent {\bf Exercise #1}}

\DeclareMathOperator{\im}{im}
\DeclareMathOperator{\coker}{coker}
\DeclareMathOperator{\Id}{Id}
\DeclareMathOperator{\GL}{GL}
\DeclareMathOperator{\Mat}{Mat}
\DeclareMathOperator{\Ext}{Ext}
\DeclareMathOperator{\Tor}{Tor}
\DeclareMathOperator{\Hom}{Hom}
\DeclareMathOperator{\Map}{Map}
\DeclareMathOperator{\ord}{ord}
\DeclareMathOperator{\Spec}{Spec}

\begin{document}

\title{Algebraic Geometry II - Assignment 2}

\author{Matteo Durante, s2303760, Leiden University}

\maketitle

\exercise{1}

\begin{proof}
    Remember that, in a category with a terminal object, the fiber product of
    two objects over it is isomorphic to the product in a unique way, hence we
    will define the diagonal morphism $X\xrightarrow{\Delta}X\times_{\Spec(\Z)}X$
    as the unique map s.t. $p_i\circ\Delta=\Id_X$, just like we would do for the
    usual product. Thanks to this, from now on we shall work forgetting that we
    are working using a fiber product and make use instead of the product to
    simplify the diagrams.
    
    Also, notice that the diagonal morphism exists for every product in every
    category as it is the unique factorization of $\Id_X$ through the
    projections $X\times X\xrightarrow{p_1,p_2}X$.

    Consider two schemes $W,Z$ and morphisms $W\xrightarrow{f_1,f_2}Z$. It can
    be shown that there exists an equalizer $W'\xrightarrow{e}W$ relative to
    them. In the proof, however, we will not assume its existence.

    Now, consider $W=X\times X,\ Z=X$, $f_i=p_i$. We will show that
    $\Delta$ is actually the equalizer we mentioned and therefore, for every
    point $y\in Z$, the natural inclusion $\Spec(k(y))\xhookrightarrow{\iota}
    X\times X$ factors uniquely through it as $p_1\circ\iota=p_2\circ\iota$
    by definition of $Z$, which will imply that $y\in\Delta(X)$.

    Indeed, consider in any category an object $X$
    s.t. its product exists and a morphism $Y\xrightarrow{g}X\times X$ s.t.
    $p_1\circ g=p_2\circ g$.

    We can consider the following diagrams:
    \[
    \begin{tikzcd}
        Y\arrow[bend left]{drr}{p_2\circ g}\arrow[bend
        right,swap]{ddr}{p_1\circ g}\arrow{dr}{g} \\
        & X\times X\arrow[swap]{r}{p_2}\arrow{d}{p_1}
        & X \\
        & X
    \end{tikzcd}
    \begin{tikzcd}
        Y\arrow[bend left]{drr}{p_2\circ\Delta\circ p_2\circ g}\arrow[bend
        right,swap]{ddr}{p_1\circ\Delta\circ p_1\circ
        g}\arrow[description]{dr}{\Delta\circ p_i\circ g} \\
        & X\times X\arrow[swap]{r}{p_2}\arrow{d}{p_1}
        & X \\
        & X
    \end{tikzcd}
    \]

    The diagram on the left is commutative and, by definition of $\Delta$,
    $p_1\circ\Delta=p_2\circ\Delta=\Id_X$, hence $p_j\circ\Delta\circ p_i\circ
    g=p_i\circ g=p_j\circ g=p_j\circ\Delta\circ p_j\circ g$, which implies that
    the one on
    the right is commutative as well and equal to the first one for any $i$
    chosen. It follows, by uniqueness of the factorization, that $\Delta\circ
    (p_i\circ g)=g$, i.e. $g$ factors through $\Delta$.

    We still have to prove the uniqueness of the factorization through $\Delta$
    and, to do this, it is enough to show that the diagonal morphism is
    actually a monomorphism.

    Indeed, let $X\times X\xrightarrow{h_1,h_2}Y$ be morphisms s.t.
    $\Delta\circ h_1=\Delta\circ h_2$. Then, $h_1=(p_i\circ\Delta)\circ h_1=
    (p_i\circ\Delta)\circ h_2=h_2$, which proves our claim.

    It follows that $\Delta$ is indeed the equalizer of $p_1,p_2$, hence we have
    the thesis.
    
    The other inclusion comes from the fact that
    $p_1\circ\Delta=\Id_X=p_2\circ\Delta$, hence $p_1$ and $p_2$ coincide at
    every point of $\Delta(X)$.

    We have proved that $\Delta(X)=Z$. By definition, a scheme $X$ is separated
    if and only if $Z$ is closed in $X\times X$, that is if and only if
    $\Delta(X)$ is closed in $X\times X$, hence we may conclude.
\end{proof}


~\\
\exercise{2}

\begin{proof}
    First of all we will prove that in the category of schemes the outer
    diagram obtained by composing two pullback squares is again a pullback
    square. It may be noticed that the proof actually works for any
    category.
    \[
	\begin{tikzcd}
        	T\arrow[bend left]{drrr}{t_2}\arrow[dotted, bend left, swap]{drr}{t'}\arrow[dotted]{dr}{t}\arrow[bend right, swap]{ddr}{t_1} \\
		& X\times_Y (Y\times_Z K)\arrow{r}{q_2}\arrow{d}{q_1}
		& Y\times_Z K\arrow{d}{p_1}\arrow{r}{p_2}
		& K\arrow{d} \\
		& X\arrow{r}{f}
		& Y\arrow{r}{g}
		& Z
	\end{tikzcd}
    \]
	
    First of all, we know that by composing commutative squares we get a
    commutative diagram.
	
    Now, given two arrows $T\xrightarrow{t_1} X,\ T\xrightarrow{t_2} K$
    making the diagram commute, we get a pair of arrows
    $T\xrightarrow{f\circ t_1} Y,\ T\xrightarrow{t_2} K$ which again make
    the diagram commute and, by the universal property of the pullback (we
    are looking at the square on the right), this induces a unique arrow
    $T\xrightarrow{t'} Y\times_Z K$ which factorizes them through the pair
    $p_1,p_2$. In particular, $p_1\circ t'=f\circ t_2$, thus the pair $t',\
    t_2$ makes the left side of the diagram commute.
	
    Doing the same with the pair of arrows $t_1,t'$ with respect to the
    square on the left, we get a unique arrow $T\xrightarrow{t} X\times_Y
    (Y\times_Z K)$ factorizing them through the arrows $q_1,q_2$.
	
    Putting everything together, we have factorized uniquely the pair
    $t_1,t_2$ making the external diagram commute through the pair
    $q_1,p_2\circ q_2$.
	
    Let $T\xrightarrow{t''} X\times_Y (Y\times_Z K)$ be another arrow making
    the diagram commute. Considering $q_2\circ t''$, we see that it factors
    the pair $f\circ t_1,t_2$ through $p_1,p_2$, hence $q_2\circ t''=t'$ by
    the uniqueness of the factorization. Since $t''$ now factors the pair
    $t_1,t'$ through $q_1,q_2$, for the same reason we get that $t''=t$.
	
    It follows that the external diagram we obtained by composing two
    pullbacks is indeed a pullback with respect to the arrows
    $X\xrightarrow{g\circ f} Z,\ K\rightarrow Z$, hence in particular
    $X\times_Y (Y\times_Z K)\cong X\times_Z K$.
	
    Remember that the pullback of two arrows is not the object per se, but
    the arrows from the object to the domains of the arrows we are
    describing the pullback of. This means that the pullback of the pair
    $g\circ f,K\rightarrow Z$ is actually the pair $q_1,p_2\circ q_2$.
	
    We will now prove that the morphism $Y\xrightarrow{g} Z$ is proper
    keeping in mind the previously seen diagram. Since $Y,\ Z$ are separated
    schemes and $g$ is of finite type, we only have to show that it is
    universally closed. By definition this means that, for any morphism
    $K\rightarrow Z$, the morphism $Y\times_Z K\xrightarrow{p_2} K$ is
    closed.
	
    Let now $V\subset Y\times_Z K$ be closed. Being $f$ surjective, $q_2$ is
    surjective as well by~\cite[p. 120, prop. 4]{Mum88}. This implies that
    $p_2(V)=(p_2\circ q_2)(q_2^{-1}(V))$. By continuity, $q_2^{-1}(V)$ is
    closed, hence, since $g\circ f$ is universally closed (proper) and
    therefore $p_2\circ q_2$ is a closed map, $p_2(V)\subset K$ is closed.
\end{proof}


~\\
\exercise{3}

\begin{proof}
    First of all, we will prove that, given an $A$-module $M$ and an affine
    map $X=\Spec(A)\xrightarrow{f}Y=\Spec(B)$, $f_*\tilde{M}\cong
    \tilde{M}_B$, where $M_B$ is the $B$-module induced by the map $f^\#$.

    There is an obvious map $\tilde{M}_B\rightarrow\tilde{M}$ as
    $\Gamma(Y,f_*\tilde{M})=\Gamma(X,\tilde{M})=M$. To verify that
    it is an isomorphism, we only have to verify that sections over an open
    $U\subset Y$ of $\tilde{M}_B$ and the ones over $f^{-1}(U)$ of
    $\tilde{M}$ coincide.

    We will first show that $f^{-1}(D(g))=D(f^\#(g))$.
	
    Indeed, a prime ideal $\pid\subset A$ satisfies $[\pid]\in f^{-1}(Y_g)$
    if and only if $f([\pid])\in Y_g$, i.e. $g\not\in (f^\#)^{-1}(\pid)$
    that is $f^\#(g)\not\in\pid$, which is equivalent to $[\pid]\in
    X_{f^\#(g)}$, which proves our claim.
	
    We can now prove that $f_*\tilde{M}=\tilde{M}_B$.
    
    Since $g$ acts on $M_B$ as multiplication by $f^\#(g)$, we have that
    $(M_B)_g=M_{f^\#(g)}=\Gamma(D(f^\#(g)),\tilde{M})=\Gamma(f^{-1}(D(g)),
    \tilde{M})=\Gamma(D(g),f_*\tilde{M})$. Since the condition
    $\Gamma(U,f_*\tilde{M})\cong\tilde{M}_B(U)$ is verified
    on a basis of the topology of $Y$ and we are working with sheaves it is
    verified on every open subset, thus $f_*\tilde{M}\cong\tilde{M}_B$. 

    Now we claim that, given a map of quasi-coherent
    $\mathcal{O}_X$-modules $\mathcal{F}\xrightarrow{\phi}\mathcal{G}$,
    kernel, image sheaf and cokernel are also quasi-coherent.

    Indeed, on an affine open $U=\Spec(A)\subset X$, we may write
    $\tilde{p}=\phi|_U$, where $M\xrightarrow{p} N$ is a homomorphism of
    $A$-modules $M$ and $N$ with $\mathcal{F}|_U=\tilde{M},\ \mathcal{G}|_U=
    \tilde{N}$. We know that the functor $\sim$ is exact, hence $\ker(\phi|_U)=
    \widetilde{\ker(p)},\ \im(\phi|_U)=\widetilde{\im(p)}$ and
    $\coker(\phi|_U)=\widetilde{\coker(p)}$, hence the thesis.

    Let now $X,Y$ be noetherian schemes, $X\xrightarrow{f} Y$ an affine map.
    We will now prove that the $\mathcal{O}_Y$-module
    $f_*\mathcal{O}_X$ is quasi-coherent. Also, since this is a local
    condition on affine open subsets and the map is affine, we may assume
    without loss of generality that $Y=\Spec(A)$.

    Since $X$ is noetherian, it is quasi-compact and therefore we may cover
    it by finitely many affine open subsets $(U_i)_{i=1}^n$. Also, for every
    pair $U_i,U_j$ we know that $U_i\cap U_j$ is again quasi-compact and
    therefore we may cover it by $(U_{ijk})_{k=1}^m$.

    It follows that, given an open $V\subset Y$, we have the following exact
    sequence:$$0\rightarrow\Gamma(f^{-1}(V),\mathcal{O}_X)\rightarrow\Pi_i
    \Gamma(U_i\cap f^{-1}(V),\mathcal{O}_X)\rightarrow\Pi_{i,j,k}
    \Gamma(U_{ijk}\cap f^{-1}(V),\mathcal{O}_X)$$

    Since the sequence is compatible with the restriction maps induced by
    the inclusions $W\subset V\subset Y$, we get an exact sequence of
    sheaves on $Y$:$$0\rightarrow f_*\mathcal{O}_X\rightarrow\Pi_i
    (f|_{U_i})_*\mathcal{O}_X|_{U_i}\rightarrow\Pi_{i,j,k}(f|_{U_{ijk}})_*
    \mathcal{O}_X|_{U_{ijk}}$$
	
    The sheaves $(f|_{U_i})_*\mathcal{O}_X|_{U_i}$ and
    $(f|_{U_{ijk}})_*\mathcal{O}_X|_{U_{ijk}}$ are both quasi-coherent by
    the affine case and, since our coverings are finite,
    $\Pi_i (f|_{U_i})_*\mathcal{O}_X|_{U_i}$ and
    $\Pi_{i,j,k}(f|_{U_{ijk}})_*\mathcal{O}_X|_{U_{ijk}}$ are
    finite products of quasi-coherent $\mathcal{O}_Y$-modules and therefore
    quasi-coherent.

    Since $f_*\mathcal{O}_X$ is the kernel of a homomorphism of
    quasi-coherent $\mathcal{O}_Y$-modules, by what we have shown earlier it
    is quasi-coherent as well.

    Now we shall prove that the quasi-coherent $\mathcal{O}_Y$-module
    $f_*\mathcal{O}_X$ is coherent if and only if $f$ is a finite morphism.
	
    Let $U$ be an affine open subset of $Y$. By definition of affine
    morphism, $f^{-1}(U)$ is an affine open subset of $X$.
	
    Clearly $\Gamma(f^{-1}(U),\mathcal{O}_X)$ is a finitely generated
    $\Gamma(f^{-1}(U),\mathcal{O}_X)$-module for every affine open subset
    $U$ of $Y$. Seeing it like earlier as a $\Gamma(U,\mathcal{O}_Y)$-module
    through the ring homomorphism
    $\Gamma(U,\mathcal{O}_Y)\xrightarrow{f(U)}\Gamma(f^{-1}(U),
    \mathcal{O}_X)$, we see that by definition the morphism is finite if and
    only if $\Gamma(f^{-1}(U),\mathcal{O}_X)=f_*\mathcal{O}_X(U)$ (we are
    abusing the notation because on the left we have a ring, on the right a
    module) is finitely generated as a $\Gamma(U,\mathcal{O}_Y)$-module for
    every affine open subset $U$ of $Y$, which is equivalent to
    $f_*\mathcal{O}_X$ being a coherent sheaf of $\mathcal{O}_Y$-modules.
\end{proof}


~\\
\exercise{4}

\begin{proof}
    Let $Y=\amalg_{i\in I}\Spec(A_i)$ be a scheme. We want to prove that, given
    $U=\amalg_{i\in I} U_i$, $U_i$ open subset of $\Spec(A_i)$,
    $\Gamma(\amalg_{i\in I} U_i,\mathcal{O}_Y)=\Pi_{i\in
    I}\Gamma(U_i,\mathcal{O}_{\Spec(A_i)})$.

    Of course, $\Gamma(U_i\cap U_j,\mathcal{O}_Y)=0$ for $i\neq j$, hence the
    subset of $\Pi_{i\in I}\Gamma(U_i,\mathcal{O}_Y)\cong\Pi_{i\in
    I}\Gamma(U_i,\Spec(A_i))$ given by the vectors of elements agreeing on the
    overlappings is $\Pi_{i\in I}\Gamma(U_i,\mathcal{O}_Y)$ itself. It follows
    that the ring homomorphism $\Gamma(\amalg_{i\in I}
    U_i,\mathcal{O}_Y)\rightarrow\Pi_{i\in I}\Gamma(U_i,\mathcal{O}_Y)$ given by
    $s\mapsto (s|_{U_i})_{i\in I}$ is an isomorphism by the sheaf axioms, hence
    $\Gamma(\amalg_{i\in I}U_i,\mathcal{O}_Y)\cong\Pi_{i\in
    I}\Gamma(U_i,\Spec(A_i))$.

    Consider now the scheme $X=\amalg_{n\in\N}\Spec(\Z)$. By what we have
    proved, $\Gamma(X,\mathcal{O}_X)=\Pi_{i\in\N}\Z$ and, considered
    $f=(2)_{n\in\N}$, we will show that $X_f=\amalg_{n\in\N}\Spec(\Z[1/2])$
    and the canonical ring homomorphism
    $\Gamma(X,\mathcal{O}_X)_f\rightarrow\Gamma(X_f,\mathcal{O}_{X_f})=\Pi_{n\in\N}\Z[1/2]$
    is not surjective.
	
    By definition, $x\in X_f$ if and only if $f(x)\neq 0$. We know that for
    every $x\in X$ we have that $x\in\Spec(\Z)$ for some $n\in\N$, hence we
    may write $x=[(p)]_n$ for $p\in\Z$ prime or $=0$. We get the following
    commutative diagram describing the map
    $\Gamma(X,\mathcal{O}_X)\rightarrow k(x)$:
    \[
	\begin{tikzcd}
        	\Gamma(X,\mathcal{O}_X)\cong\Pi_{n\in\N}\Z\arrow{rr}{|^X_{\Spec(\Z)}=\pi_n}\arrow{dr}
		&& \Gamma(\Spec(\Z),\mathcal{O}_X)\arrow{dl}\cong\Z \\
		& k(x)\cong Q(\Z/p\Z)
	\end{tikzcd}
    \]
	
    Under these maps, we see that $f(x)=0$ for $x=[(p)]_n$ if and only if
    $p=2$. It follows that $X_f\cap\Spec(\Z)=(\Spec(\Z))_2$ for all $n\in\N$
    and therefore $X_f=\amalg_{n\in\N}(\Spec(\Z))_2\cong\amalg_{n\in\N}
    \Spec(\Z[1/2])$.
	
    By what we proved earlier,
    $\Gamma(X_f,\mathcal{O}_{X_f})=\Pi_{n\in\N}\Z[1/2]$. Of course, we
    have a natural injection
    $\Gamma(X,\mathcal{O}_X)_f=(\Pi_{n\in\N}\Z)_f\rightarrow\Gamma(X_f,
    \mathcal{O}_{X_f})=\Pi_{n\in\N}\Z[1/2]$ given by the restriction map
    $\Gamma(X,\mathcal{O}_X)\xrightarrow{|^X_{X_f}}\Gamma(X_f,
    \mathcal{O}_X)=\Gamma(X_f,\mathcal{O}_{X_f})$,
    which then factors through $\Gamma(X,\mathcal{O}_X)_f$ by the universal
    property of the localization. Also, the injectivity of the map is given
    by the fact that we are localizing with respect to elements which are
    not zero-divisors.

    On the other hand, we have that $(2^n)_{n\in\N}$ is not invertible in
    $\Gamma(X,\mathcal{O}_X)_f$ but its image is invertible in
    $\Gamma(X_f,\mathcal{O}_{X_f})$, hence the map is not an isomorphism and
    in particular it is not surjective, for otherwise the inverse would be
    the image of some element of the domain, which would then be the inverse
    of $(2^n)_{n\in\N}$ in $\Gamma(X,\mathcal{O}_X)_f$ due to the
    injectivity.
	
    We will now give a counterexample to the injectivity.
	
    Consider the scheme $Y=\amalg_{n\in\N}\Spec(\Z[x]/(x^n))$ and the
    element
    $f=(x)_{n\in\N}\in\Gamma(Y,\mathcal{O}_Y)=\Pi_{n\in\N}\Z[x]/(x^n)$. We
    see that $f$ is not nilpotent, for given any $n\in\N$ we have $x^n$ at
    the $(n+1)$th coordinate of $f^n$, which is $\neq 0$ in
    $\Z[x]/(x^{n+1})$.
	
    On the other hand, for any $n\in\N$ we see that
    $Y_f\cap\Spec(\Z[x]/(x^n))=\emptyset$, for the restriction map
    $\Gamma(Y,\mathcal{O}_Y)\xrightarrow{|^Y_{\Spec(\Z[x]/(x^n))}}\Gamma(\Spec(\Z[x]/(x^n)),\mathcal{O}_Y)$
    sends $f$ to $x$, which is nilpotent in $\Z[x]/(x^n)$ and therefore
    $f([y])=0$ for every $[y]$ in $\Spec(\Z[x]/(x^n))$.
	
    In the following diagram describing the map $g\mapsto g([y])$ for
    $y\in\Spec(\Z[x]/(x^n))$ we assume $p\in\Z$ to be either prime or 0 and
    remember that the prime ideals of $\Z[x]/(x^n)$ have to contain $x$,
    thus they correspond bijectively to the ones of $\Z$:
    \[
	\begin{tikzcd}
		\Gamma(Y,\mathcal{O}_Y)\cong\Pi_{n\in\N}\Z[x]/(x^n)\arrow{rr}{|^Y_{\Spec(\Z[x]/(x^n))}=\pi_n}\arrow{dr}
		&& \Gamma(\Spec(\Z[x]/(x^n)),\mathcal{O}_Y)\cong\Z[x]/(x^n)\arrow{dl} \\
		& k([y])\cong Q((\Z[x]/(x^n))/y)\cong Q(\Z/p\Z)
	\end{tikzcd}
    \]
	
    It follows that $Y_f=\emptyset$, thus
    $\Gamma(Y_f,\mathcal{O}_{Y_f})=\Gamma(Y_f,\mathcal{O}_Y)=0$. We only
    have to prove that $\Gamma(Y,\mathcal{O}_Y)_f\neq 0$, which is given by
    the fact that $f$ is not nilpotent and therefore the localization does
    not trivialize the ring.
\end{proof}

\printbibliography

\end{document}
