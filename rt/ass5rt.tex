\documentclass{article}
\usepackage[T1]{fontenc}
\usepackage{lmodern}
\usepackage[utf8]{inputenc}
\usepackage[british]{babel}
\usepackage{geometry}
\usepackage{color}
\usepackage{amsthm}
\usepackage{amsmath,amssymb}
\usepackage{graphicx}
\usepackage{mathtools}
\usepackage{listings}
\usepackage{newlfont}
\usepackage{tikz-cd}
\usepackage{rotating}
\usepackage[backend=biber]{biblatex}
\addbibresource{~/math/references.bib}

\newcommand{\numberset}{\mathbb}
\newcommand{\N}{\numberset{N}}
\newcommand{\Z}{\numberset{Z}}
\newcommand{\Q}{\numberset{Q}}
\newcommand{\R}{\numberset{R}}
\newcommand{\C}{\numberset{C}}
\newcommand{\K}{\numberset{K}}
\newcommand{\F}{\numberset{F}}
\newcommand{\n}{\mathcal{N}}
\newcommand{\aid}{\mathfrak{a}}
\newcommand{\bid}{\mathfrak{b}}
\newcommand{\pid}{\mathfrak{p}}
\newcommand{\qid}{\mathfrak{q}}
\newcommand{\mi}{\mathfrak{m}}
\newcommand{\I}{\mathbb{I}}
\newcommand{\V}{\mathbb{V}}
\newcommand{\A}{\mathbb{A}}
\newcommand{\Ps}{\mathbb{P}}
\newcommand{\exercise}[1]{\noindent {\bf Exercise #1}}

\DeclareMathOperator{\im}{im}
\DeclareMathOperator{\coker}{coker}
\DeclareMathOperator{\Id}{Id}
\DeclareMathOperator{\GL}{GL}
\DeclareMathOperator{\Mat}{Mat}
\DeclareMathOperator{\Ext}{Ext}
\DeclareMathOperator{\Tor}{Tor}
\DeclareMathOperator{\Hom}{Hom}
\DeclareMathOperator{\Aut}{Aut}
\DeclareMathOperator{\Class}{Class}


\begin{document}

\title{Representation Theory of Finite Groups - Assignment 5}

\author{Matteo Durante, s2303760, Leiden University}

\maketitle

\exercise{9.1}

\begin{proof}
    $(1\implies 2)$ This comes from the fact that any irreducible
    $\C$-representation is finite dimensional as the dimension has to divide
    $|G|$.

    $(2\implies 3)$ Any irreducible character $\chi\in X(G)$ corresponds to an
    irreducible $\C$-representation $G\xrightarrow{\rho}\Aut_\C(V)$ and we know
    that $\chi_V=\overline{\chi_{V^*}}$. Also, since $\chi_V=\overline{\chi_V}$,
    we have that $\chi_{V^*}=\chi_V$. This happens if and only if the two
    representations are equivalent, hence we have that $V\cong V^*$.

    $(3\implies 4)$ Let now $g\in G$. We know that, for any character
    $\chi_V\in X(G)$, $\overline{\chi_V(g)}=\chi_V(g^{-1})$ $(*)$. Also, for
    irreducible characters, $\chi_V=\chi_{V^*}=\overline{\chi_V}$ because the
    representations corresponding to $V$ and $V^*$ are isomorphic. This implies
    $\chi_V(g)=\chi_V(g^{-1})$.
    
    Since irreducible characters generate $\Class_\C(G)$, this means that
    $\chi(g)=\chi(g^{-1})$ for any $\chi\in\Class_\C(G)$. If $g\not\sim g^{-1}$,
    then there would be a class function assigning distinct values to the two of
    them, hence $g\sim g^{-1}$.

    $(*)$ Given a $\C$-representation $G\xrightarrow{\rho}\Aut_\C(V)$, for any
    $g\in G$ of finite order $n$ we have $\rho(g)^n=\rho(g^n)=\rho(1)=\Id$, thus
    the characteristic polynomial of $\rho(g)$ divides $X^n-1$ and it has
    distinct roots. It follows that there is a basis $B$ of eigenvectors
    diagonalizing $\rho(g)$. Since changing basis does not affect the trace
    we may then fix this one, which immediately gives that
    $\chi(g)=\sum_i\lambda_i$, where the $\lambda_i$ are the eigenvalues of
    $\rho(g)$. Clearly, with respect to our basis, $\rho(g^{-1})=\rho(g)^{-1}$
    has the $\lambda_i^{-1}$ on the diagonal and therefore
    $\chi(g^{-1})=\sum_i\lambda_i^{-1}$. Since $\lambda_i$ is a root of $X^n-1$,
    it is a root of unity and therefore $\lambda_i^{-1}=\overline{\lambda_i}$.
    It follows that $\chi(g^{-1})=\sum_i\lambda_i^{-1}=
    \sum_i\overline{\lambda_i}=\overline{\sum_i\lambda_i}=\overline{\chi(g)}$.

    $(4\implies 1)$ As shown earlier, $\chi(g^{-1})=\overline{\chi(g)}$
    for any character $\chi\in\Class_\C(G)$. Also, since $g\sim g^{-1}$,
    $\chi(g)=\chi(g^{-1})$, hence $\chi(g)=\overline{\chi(g)}$. This immediately
    implies that $\chi$ is real valued.
\end{proof}


~\\
\exercise{9.5}

\begin{proof}
    Let $S_4$ act on the set $Y$ of the faces of a cube by permuting the
    diagonals and consider the induced group homomorphism
    $S_4\xrightarrow{\rho}W_Y=\C^Y\cong\C^6$. Let now $\chi\in X(S_4)$ be the
    associated representation. Looking at
    the fixed points of the elements of $S_4$, we will determine
    $\chi(s)$ for all $s\in S_4$. Indeed, as we are about to show, $\chi(s)$
    is given by the cardinality of the set of fixed points in $Y$.

    Let the linear transformation $\rho(s)$ be represented by a matrix with
    respect to the canonical basis. This is a permutation matrix, i.e. it has
    one non-zero entry in each row and column, which are indexed by the elements
    of $Y$. We see that $a_{bc}=1$ if and only if $s\cdot c=b$ and it is 0
    otherwise. From this it is clear that the non-zero diagonal entries
    correspond to the elements in $Y$ fixed by $s$. Their number will then
    correspond to $\chi(s)$.

    Remember that, since $\chi$ is a class function, it is constant on the
    conjugacy classes, which we have already described in a previous assignment.
    
    Clearly, $\rho(\Id_{S_4})$ fixes every face of the cube and therefore
    $\chi(1)=6$. On the other hand, $\rho(i\ j)$ and $\rho(i\ j\ k)$ do not fix
    any for distinct $i,j,k$, while $\rho((h\ i)(j\ k))$ and $\rho(h\ i\ j\ k)$
    both fix two faces.

    It follows that we may write $\chi=6[\Id_{S_4}]+2[(h\ i)(j\ k)]+2[(h\ i\ j\
    k)]$ and we know that there is a unique way to describe it as a linear
    combination of irreducible characters $\chi=\sum_ia_i\chi_i$. Look at the
    following system of equations given by $\chi(s)=\sum_ia_i\chi_i(s)$ as $s$
    ranges over the 5 conjugacy classes:
    \begin{align*}
        a_1+a_2+2a_3+3a_4+3a_5 &=6\quad [\Id_{S_4}] \\
        a_1-a_2+a_4-a_5 &=0\quad [(i\ j)] \\
        a_1+a_2-a_3 &=0\quad [(i\ j\ k)] \\
        a_1-a_2-a_4+a_5 &=2\quad [(h\ i)(j\ k)] \\
        a_1+a_2+2a_3-a_4-a_5 &=2\quad [(h\ i\ j\ k)]
    \end{align*}

    Solving this system of equations we find the solution $(1,0,1,0,1)$, hence
    $\chi=\chi_1+\chi_3+\chi_5$, $W_Y=\bigoplus_{i=1}^3V_i$ and
    $S_4\xrightarrow{\rho=\bigoplus_{i=1}^3\rho_i}
    \bigoplus_{i=1}^3\Aut_\C(V_i)\subset\Aut_\C(W_Y)$.
\end{proof}


~\\
\exercise{10.5}

\begin{proof}
    Let $f=\sum_{h\in G} c_hh\in\C[G]$. Noticing that multiplying $|G|e$ by
    $h\in G$ on any side we are just permuting the terms, we have the following:
    \begin{align*}
        e\cdot f &= (\frac{1}{|G|}\sum_{g\in G} g)(\sum_{h\in G} c_hh) \\
        &= \frac{1}{|G|}\sum_{g\in G} (\sum_{h\in G} c_hgh) \\
        &= \frac{1}{|G|}\sum_{h\in G} (\sum_{g\in G} c_hgh) \\
        &= \frac{1}{|G|}\sum_{h\in G} c_h (\sum_{g\in G} gh) \\
        &= \frac{1}{|G|}\sum_{h\in G} c_h (\sum_{g\in G} hg) \\
        &= \sum_{h\in G} c_hh (\frac{1}{|G|}\sum_{g\in G} g) \\
        &= (\sum_{h\in G} c_hh)(\frac{1}{|G|}\sum_{g\in G} g) \\
        &= f\cdot e
    \end{align*}
    It follows that $e\in Z(\C[G])$.

    Observing that $e^2=\frac{1}{|G|^2}\sum_{h\in G}\sum_{g\in G}
    hg=\frac{1}{|G|^2}\sum_{h\in G} h(\sum_{g\in G}
    g)=\frac{1}{|G|^2}|G|\sum_{g\in G}g=\frac{1}{|G|}\sum_{g\in G}g=e$, we have
    that $e$ is a root of the polynomial $p(X)=X^2-X\in\Z[X]$, hence it is
    integral over $\Z$.
\end{proof}


~\\
\exercise{10.8}

\begin{proof}
    $(a)$ We know that $\#(S_{3_/\sim})=3$ and therefore
    $Z(\C[S_3])\cong\C\times\C\times\C$. We will make the isomorphism explicit.

    Let's call $C_j$ the equivalence class of the elements of order $j$ in
    $S_3$.

    We already have an isomorphism
    $\C[S_3]\xrightarrow{\rho}\Pi_{i=1}^3\Mat_{n_i}(\C)$ given by extending
    $s\mapsto(\rho_i(s))_{i=1}^3$, where $n_i$ is the dimension of the $i$th
    irreducible representation and therefore $n_1=n_2=1,\ n_2=2$. By restricting
    it to $Z(\C[S_3])$ we will have the desired isomorphism.

    We know that the elements of $Z(\C[S_3])$ have the form $\sum_ja_j\sum_{s\in
    C_j}s$. Also, $\rho_1$ is the final representation
    and therefore, on the first coordinate, $\sum_ja_j\sum_{s\in
    C_j}s\mapsto\sum_ja_j\sum_{s\in C_j}1=\sum_j|C_j|a_j=
    a_1+3a_2+2a_3$. Likewise, $\rho_2$ is the final representation, hence
    $\sum_ja_j\sum_{s\in C_j}s\mapsto\sum_ja_j\sum_{s\in C_j}(-1)^{j+1}=
    a_1-3a_2+2a_3$.

    Finally, $\rho_3$ is the permutation representation, which is given by
    taking the subspace $V$ of $\C^3$ spanned by $e_1-e_2,\ e_1-e_3$ and setting
    $\rho_3(s)(e_i)=e_{s(i)}$.

    We have the following:
    \begin{align*}
        \rho_3(1\ 2)&=\begin{bmatrix}
            -1 & -1 \\
            0 & 1
        \end{bmatrix},\quad
        \rho_3(1\ 3)=\begin{bmatrix}
            1 & 0 \\
            -1 & -1
        \end{bmatrix},\quad
        \rho_3(2\ 3)=\begin{bmatrix}
            0 & 1 \\
            1 & 0
        \end{bmatrix}, \\
        \rho_3(1\ 2\ 3)&=\begin{bmatrix}
            -1 & -1 \\
            1 & 0
        \end{bmatrix},\quad
        \rho_3(1\ 3\ 2)=\begin{bmatrix}
            0 & 1 \\
            -1 & -1
        \end{bmatrix}
    \end{align*}

    As before, we have on the third coordinate $\sum_ja_j\sum_{s\in C_j}s\mapsto
    a_1\cdot\Id_V-a_3\cdot\Id_V$.

    Consider the following system of equations as $k$ ranges from $1$ to $3$:
    \begin{align*}
        a_1+3a_2+2a_3=\delta_{1k} \\
        a_1-3a_2+2a_3=\delta_{2k} \\
        a_1-a_3=\delta_{3k}
    \end{align*}

    Solving them, we find the solution $(1/6,1/6,1/6)$ for $k=1$,
    $(1/6,-1/6,1/6)$ for $k=2$ and $(2/3,0,1/2)$ for $k=3$, which give us the
    unique elements mapped to $(\delta_{1k},\delta_{2k},\delta_{3k})$. Since
    their images are $\C$-linearly independent they are linearly independent
    themselves. Also, being $Z(\C[S_3])$ a 3-dimensional $\C$-vector space,
    it follows that they generate the whole space. Finally, seeing the ring
    homomorphism $Z(\C[S_3])\rightarrow\C\times\C\times\Mat_2(\C)$ given by
    restricting the domain of the previously mentioned isomorphism as a
    $\C$-linear application between $\C$-vector spaces, it becomes clear that
    the elements lying in its image have the form $(a,b,c\cdot\Id_V)$ for
    $a,b,c\in\C$, hence $Z(\C[S_3])\cong\C\times\C\times\C$.
\end{proof}

\begin{proof}
    $(b)$ The previously mentioned isomorphism is naturally a
    $\overline{\Z}$-algebra isomorphism, hence we may simply find the integral
    closure of $\Z$ in $\C\times\C\times\C$, which will give us the desired
    isomorphism by restricting the codomain.

    Let $p\in\Z[X]$ be a monic polynomial. An element
    $a=(a_i)_{i=1}^3\in\C\times\C\times\C$ is a zero
    of this polynomial if and only if $p(a_i)=0$ for each $i$, which is
    equivalent to saying that $a_i\in\overline{Z}\subset\C$. It follows that the
    integral closure of $\Z$ in $\C\times\C\times\C$ is contained in
    $\overline{\Z}\times\overline{\Z}\times\overline{\Z}$.

    On the other hand, let
    $a\in\overline{\Z}\times\overline{\Z}\times\overline{\Z}$. For each $a_i$,
    let $p_i\in\Z[X]$ be the associated minimum polynomial and set
    $p=p_1p_2p_3$. We see that $p\in\Z[X]$ is still monic and $p(a)=0$, thus $a$
    belongs to the integral closure of $\Z$ in $\C\times\C\times\C$.

    To find the generators of the algebraic closure as a
    $\overline{\Z}$-submodule of $Z(\C[S_3])$, it is enough to look at the
    preimages of $(1,0,0)$, $(0,1,0)$, $(0,0,1)$ under our isomorphism. By our
    earlier computations, these are respectively $\frac{1}{6}\sum_{s\in S_3}s$,
    $\sum_j\frac{(-1)^{j+1}}{6}\sum_{s\in C_j}s$,
    $\frac{2}{3}-\frac{1}{2}\sum_{s\in C_3}s$.
\end{proof}



\printbibliography

\end{document}
