\documentclass{article}
\usepackage[T1]{fontenc}
\usepackage{lmodern}
\usepackage[utf8]{inputenc}
\usepackage[british]{babel}
\usepackage{geometry}
\usepackage{color}
\usepackage{amsthm}
\usepackage{amsmath,amssymb}
\usepackage{graphicx}
\usepackage{mathtools}
\usepackage{listings}
\usepackage{newlfont}
\usepackage{tikz-cd}
\usepackage{rotating}
\usepackage[backend=biber]{biblatex}
\addbibresource{~/math/references.bib}

\newcommand{\numberset}{\mathbb}
\newcommand{\N}{\numberset{N}}
\newcommand{\Z}{\numberset{Z}}
\newcommand{\R}{\numberset{R}}
\newcommand{\Q}{\numberset{Q}}
\newcommand{\K}{\numberset{K}}
\newcommand{\F}{\numberset{F}}
\newcommand{\n}{\mathcal{N}}
\newcommand{\aid}{\mathfrak{a}}
\newcommand{\bid}{\mathfrak{b}}
\newcommand{\pid}{\mathfrak{p}}
\newcommand{\qid}{\mathfrak{q}}
\newcommand{\mi}{\mathfrak{m}}
\newcommand{\I}{\mathbb{I}}
\newcommand{\V}{\mathbb{V}}
\newcommand{\A}{\mathbb{A}}
\newcommand{\Ps}{\mathbb{P}}
\newcommand{\exercise}[1]{\noindent {\bf Exercise #1}}

\DeclareMathOperator{\Ima}{Im}
\DeclareMathOperator{\coker}{coker}
\DeclareMathOperator{\Id}{Id}
\DeclareMathOperator{\GL}{GL}
\DeclareMathOperator{\Aut}{Aut}
\DeclareMathOperator{\Mat}{Mat}

\begin{document}

\title{Representation Theory of Finite Groups - Assignment 2}

\author{Matteo Durante, s2303760, Leiden University}

\maketitle


~\\
\exercise{3.5}

$(a)$ Let's define the map $N\xrightarrow{f}\Pi_{i\in I} M_i$ as $n\mapsto
(f_i(n))_{i\in I}$. We will show that this is a $R$-module homomorphism making
the desired diagrams commute.
\[
    \begin{tikzcd}
        N\arrow{r}{f}\arrow[swap]{dr}{f_j}
        & \Pi_{i\in I} M_i\arrow{d}{p_j} \\
        & M_j
    \end{tikzcd}
\]
The commutativity is trivial since, for any $n\in N$, $(p_j\circ f)(n)=p_j((f_i
(n))_{i\in I})=f_j(n)$.

We want to prove that it is indeed an $R$-module homomorphism.

First of all, it is a group homomorphism because for every $n,n'\in N$ we have
that
\begin{align*}
    f(n+n') & =(f_i(n+n'))_{i\in I} \\
    & =(f_i(n)+f_i(n'))_{i\in I} \\
    & =(f_i(n))_{i\in I}+(f_i(n'))_{i\in I} \\
    & =f(n)+f(n')
\end{align*}

Furthermore, let $r\in R,\ n\in N$. We see that:
\begin{align*}
    f(r\cdot n) & =(f_i(r\cdot n))_{i\in I} \\
    & =(r\cdot f_i(n))_{i\in I} \\
    & =r\cdot (f_i(n))_{i\in I} \\
    & =r\cdot f(n)
\end{align*}

Let now $N\xrightarrow{f'}\Pi_{i\in I} M_i$, $n\mapsto (f'_i(n))_{i\in I}$ be 
another $R$-module homomorphism making the diagrams commute. Then, $f'_i(n)=p_i
(f'(n))=(p_i\circ f')(n)=f_i(n)$, i.e. $f'$ coincides with $f$ in every
component and therefore $f=f'$.

~\\
$(b)$ Let's define the map $\bigoplus_{i\in I} M_i\xrightarrow{g} N$ as
$(m_i)_{i\in I}\mapsto\sum_{i\in I} g_i(m_i)$. This map is clearly well defined
as there are finitely many $i\in I$ s.t. $m_i\neq 0$ (and hence $g_i(m_i)=0$),
thus the one we are considering is a finite sum (we may disregard all of the 
$m_i$ which are 0).

We will show that this is a $R$-module homomorphism making the desired diagrams
commute.
\[
    \begin{tikzcd}
        M_j\arrow{r}{h_j}\arrow[swap]{dr}{g_j}
        & \bigoplus_{i\in I} M_i\arrow{d}{g} \\
        & N
    \end{tikzcd}
\]
The commutativity is trivial since, for any $m_j\in M_j$, $(g\circ
h_j)(m_j)=g((m_i)_{i\in I})=\sum_{i\in I} g_i(m_i)=g_j(m_j)$, where $m_j$ is
mapped by $h_j$ to the element of $\bigoplus_{i\in I} M_i$ having a
0 at every coordinate $i\neq j$ and $m_j$ at the coordinate $j$.

It is a group homomorphism because for every $(m_i)_{i\in I}, (m'_i)_{i\in I}\in
\bigoplus_{i\in I} M_i$ we have that:
\begin{align*}
    g((m_i)_{i\in I}+(m'_i)_{i\in I}) & =g((m_i+m'_i)_{i\in I}) \\
    & =\sum_{i\in I} g_i(m_i+m'_i) \\
    & =\sum_{i\in I} (g_i(m_i)+g_i(m'_i)) \\
    & =\sum_{i\in I} g_i(m_i)+\sum_{i\in I} g_i(m'_i) \\
    & =g((m_i)_{i\in I})+g((m'_i)_{i\in I})
\end{align*}

Furthermore, let $r\in R,\ (m_i)_{i\in I}\in\bigoplus_{i\in I} M_i$. We see
that:
\begin{align*}
    g(r\cdot (m_i)_{i\in I}) & =g((r\cdot m_i)_{i\in I}) \\
    & =\sum_{i\in I} g_i(r\cdot m_i) \\
    & =\sum_{i\in I} r\cdot g_i(m_i) \\
    & =r\cdot\left(\sum_{i\in I} g_i(m_i)\right) \\
    & =r\cdot g((m_i)_{i\in I})
\end{align*}

Let now $\bigoplus_{i\in I} M_i\xrightarrow{g'} N$ be another $R$-module
homomorphism making the diagrams commute. Then, considered an element
$(m_i)_{i\in I}$ s.t. $m_i=0$ for every $i\neq j$, $g'( (m_i)_{i\in
I}))=g'(h_j(m_j))=(g'\circ h_j)(m_j)=g_j(m_j)=g(h_j(m_j))=g((m_i)_{i\in I})$.
Since these elements generate $\bigoplus_{i\in I} M_i$ and $g'$ coincides with
$g$ on them, $g=g'$.

~\\
$(c)$ For the first correspondence consider the morphism given by $f\mapsto
(f\circ h_i)_{i\in I}$. We see that it is surjective for, given any collection
of $R$-module homomorphisms $M_i\xrightarrow{f_i} N$, we have a $R$-module 
homomorphism $\bigoplus_{i\in I} M_i\xrightarrow{f} N$ s.t. $f\circ h_i=f_i$ for
all $i\in I$ by $(b)$. On the other hand, the map factorizing all of the $f_i$
through the $h_i$ is
uniquely defined, thus if $\bigoplus_{i\in I} M_i\xrightarrow{f,f'} N$ are two $R$-module 
homomorphisms s.t. $(f\circ h_i)_{i\in I}=(g_i)_{i\in I}=(f'\circ h_i)_{i\in I}$,
since the two of them factorize the same collection of $R$-module homomorphisms,
we have that $f=f'$ again by $(b)$.

For the second correspondence consider the morphism given by $f\mapsto
(p_i\circ f)_{i\in I}$. We see that it is surjective for, given any collection
of $R$-module homomorphisms $N\xrightarrow{f_i} M_i$, we have a $R$-module 
homomorphism $N\xrightarrow{f}\Pi_{i\in I} M_i$ s.t. $p_i\circ f=f_i$ for
all $i\in I$ by $(a)$. On the other hand, the map factorizing all of the $f_i$
through the $p_i$ is
uniquely defined, thus if $N\xrightarrow{f,f'} \Pi_{i\in I} M_i$ are two $R$-module 
homomorphisms s.t. $(p_i\circ f)_{i\in I}=(g_i)_{i\in I}$ $(p_i\circ f')_{i\in I}$,
since the two of them factorize the same collection of $R$-module homomorphisms,
we have that $f=f'$ again by $(a)$.


~\\
\exercise{3.11}

$(a)$ Consider the map $\Mat(n,\K)\xrightarrow{f} \bigoplus_{i=1}^n V$ given by
$A\mapsto ((a_{i,j})_{i=1}^n)_{j=1}^n$. We will prove that it is an $R$-module
isomorphism.

It is clearly a well defined group homomorphism, hence we start from checking
that it is $\Mat(n,\K)$-linear.
\begin{align*}
    f(A\cdot B) & =f\left(\left(\sum_{k=1}^n
    a_{i,k}b_{k,j}\right)_{i,j=1}^n\right) \\
    & =\left(\left(\sum_{k=1}^n a_{i,k}b_{k,j}\right)_{i=1}^n\right)_{j=1}^n \\
    & =(A\cdot (b_{k,j})_{k=1}^n)_{j=1}^n \\
    & =A\cdot ((b_{k,j})_{k=1}^n)_{j=1}^n \\
    & =A\cdot f(B)
\end{align*}

Now we have to check that it is an isomorphism, which is trivial because the map
is clearly surjective and the only matrix mapped to the $n$-tuple of zero-vectors
is the null one.

~\\
$(b)$ Let $M$ be a simple $\Mat(n,\K)$-module. By~\cite[prop. 9.7]{Tor10}, $M
\cong V$ because $\Mat(n,\K)\cong \bigoplus_{i=1}^n V$ as a left
$\Mat(n,\K)$-module and $V$ is a simple $R$-module.


~\\
\exercise{4.5}

Throughout the exercise, we will use $\cdot$ to denote the action of an element,
while $gv(-)$ will be the function induced by $v(g^{-1}\cdot -)$.

$(a)$ We will begin by proving that, given any $g\in G$, the map which has been
defined is a $\K$-automorphism of $\K^X$.

Let $v,w\in\K^X$. For any $x\in X$, $\lambda,\mu\in\K$,
$g(\lambda\cdot v+\mu\cdot w)(x)=(\lambda\cdot v+\mu\cdot w)(g^{-1}\cdot x)=
\lambda\cdot v(g^{-1}\cdot x)+\mu\cdot w(g^{-1}\cdot x)=\lambda\cdot gv(x)+\mu\cdot gw
(x)$, i.e. $g$ defines a $\K$-vector space endomorphism.

It is clearly bijective, for $g(g^{-1}v)(x)=g^{-1}v(g^{-1}\cdot x)=v(
(g^{-1})^{-1}\cdot (g^{-1}\cdot x))=v((gg^{-1})\cdot x)=v(x)$, i.e. $g\circ g^{-1}=\Id_{\K^X}$ and in
the same way $g^{-1}\circ g=\Id_{\K^X}$ (here we are abusing the notation by
calling $g$ the function it defines).

It follows that $g$ defines an element of $\Aut_{\K}(\K^X)$.

We want to prove that the function $G\xrightarrow{\phi}\Aut_{\K}(\K^X)$ sending
$g\in G$ to the automorphism defined by $g^{-1}$ is actually a group homomorphism.

Let $g,h\in G$, $v\in\K^X$, $x\in X$. We see that:
\begin{align*}
    \phi(gh)(v)(x) & =(gh)^{-1}v(x) \\
    & =v(gh\cdot x) \\
    & =v(g\cdot (h\cdot x)) \\
    & =g^{-1}v(h\cdot x) \\
    & =\phi(g)(h^{-1}v)(x) \\
    & =\phi(g)(\phi(h)(v))(x) \\
    & =(\phi(g)\circ\phi(h))(v)(x)
\end{align*}

~\\
$(b)$ Remember that the $\K[G]$ module structure of $\K^X$ is given by $v\mapsto
g\cdot v:=\phi(g)(v)$.

We will now prove the $\K[G]$-linearity.

Let $v_g\in\K^X,\ x\in X,\ \lambda_g\in\K$ with $\lambda_g\neq 0$ for finitely
many $g\in G$. Then:
\begin{align*}
    f^*\left(\sum_{g\in G}\lambda_g g\cdot v_g\right)(x) & = \left(\left(\sum_{g\in G}\lambda_g
    g\cdot v_g\right)\circ f\right)(x)=\left(\sum_{g\in
    G}\lambda_g\phi(g)(v_g)\right)(f(x)) \\
    & = \sum_{g\in G}\lambda_g\phi(g)(v_g)(f(x))=\sum_{g\in G}\lambda_g g^{-1}
    v_g(f(x)) \\
    & = \sum_{g\in G}\lambda_g v_g(g\cdot f(x))=\sum_{g\in G}\lambda_g
    v_g(f(g\cdot x))\textit{ by equivariance} \\
    & = \sum_{g\in G}\lambda_g (v_g\circ f)(g\cdot x)=\sum_{g\in G}\lambda_g
    g^{-1}(v_g\circ f)(x) \\
    & = \sum_{g\in G}\lambda_g\phi(g)(f^*(v_g))(x)=\sum_{g\in G}(\lambda_g
    g\cdot f^*(v_g))(x) \\
    & = \left(\sum_{g\in G}\lambda_g g\cdot f^*(v_g)\right)(x)
\end{align*}

This concludes the proof.

~\\
$(c)$ We only have to prove that the mapping preserves the identities (i.e.
$\Id_X\mapsto\Id_{F(X)}=\Id_{\K^X}$) and the compositions, reversing the arrows
($g\circ f\mapsto (g\circ f)^*=f^*\circ g^*$).

Let $X$ be a $G$-set. We have that, for any $v\in\K^X,\ x\in X$,
$\Id_X^*(v)(x)=(v\circ\Id_X)(x)=v(\Id_X(x))=v(x)$, i.e. $\Id_X^*=\Id_{\K^X}$.

Let now $X,Y,Z$ be $G$-sets, $X\xrightarrow{f}Y\xrightarrow{g}Z$ two 
$G$-equivariant maps. For any $v\in\K^X$, we have:
\begin{align*}
    (f^*\circ g^*)(v) & = f^*(g^*(v)) \\
    & = (v\circ g)\circ f \\
    & = v\circ (g\circ f) \\
    & = (g\circ f)^*(v)
\end{align*}

It follows that $F$ is indeed a contravariant functor.


~\\
\exercise{4.8}

Consider a pair of ring homomorphisms $\Z/m\Z\xrightarrow{f}R,\
\Z/n\Z\xrightarrow{g}R,\ k=char(R)$. We know that $k|m,n$, the characteristics
of our domains, for otherwise we would not have at least one among the two ring
homomorphisms from $\Z/m\Z,\Z/n\Z$ to $R$, hence $k|gcd(m,n)=d$.

It follows that $f([d]_m)=g([d]_n)=0$, thus by the universal property both ring
homomorphisms factor uniquely through $i$ and $j$ as $(\Z/m\Z)/(d\Z/m\Z)\cong
\Z/d\Z\cong(\Z/n\Z)/(d\Z/n\Z)$.

We still want to check that the two factorizations through $\Z/d\Z$ given by the
canonical projections induce the same ring homomorphism $\Z/d\Z\xrightarrow{h} R$.

However, this is trivial, for a ring homomorphism must map the unit of the
domain to the unit of the codomain, i.e. $[1]_d\mapsto 1_R$, and, since $[1]_d$
generates $\Z/d\Z$, this uniquely defines the ring homomorphism, thus there
exists only one ring homomorphism from $\Z/d\Z$ to $R$.
\[
    \begin{tikzcd}
        & \Z/m\Z\arrow{d}{i}\arrow[bend left]{ddr}{f} \\
        \Z/n\Z\arrow{r}{j}\arrow[bend right]{drr}{g}
        & \Z/d\Z\arrow[dotted]{dr}[description]{\exists! h} \\
        && R
    \end{tikzcd}
\]

\printbibliography

\end{document}
